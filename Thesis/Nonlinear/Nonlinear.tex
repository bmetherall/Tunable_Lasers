% !TeX root = ../Thesis.tex

\chapter{Solution of the Nonlinear Model}
\section{Code}
With the inclusion of the nonlinearity ($b > 0$) the model becomes too difficult to solve analytically, and we must resort to solving it numerically. 

Appendix \ref{chap:code}

\subsection{Validation}
The code can be validated by comparing to the results of the linear model.

\begin{figure}[htbp]
\centering
\input{./Figures/Variance}
\caption{Simulation and analytic equilibrium variance, chirp, and phase shift as a function of $s$.}
\label{fig:}
\end{figure}



\begin{figure}[htbp]
\centering
\input{./Figures/LinearPlot}
\caption{Equilibrium energy and peak power of the pulse as a function of $s$.}
\label{fig:}
\end{figure}

\section{Nonlinear Model}


\begin{figure}[htbp]
\centering
\input{./Figures/Stability}
\caption{}
\label{fig:}
\end{figure}

\begin{figure}[htbp]
\centering
\input{./Figures/StabilityZoom}
\caption{}
\label{fig:}
\end{figure}

\begin{figure}[htbp]
\centering
\input{./Figures/StabilitySwitch}
\caption{}
\label{fig:}
\end{figure}

\begin{figure}[htbp]
\centering
\input{./Figures/StabilitySwitchZoom}
\caption{}
\label{fig:}
\end{figure}

In the linear model, the pulse converges to a Gaussian as long as $a h^2 \zeta > 1$ so that the equilibrium energy from \eqref{eq:equilenergy} is positive. However, this is not necessarily the case with the inclusion of the nonlinearity. The fibre adds a phase shift proportional to the power of the pulse, this in turn can inject higher frequency oscillations due to the dispersion which then reinforces the oscillations. These oscillations are then intensified with each trip around the cavity until the envelope of the pulse becomes mangled. \\

Figure \ref{fig:pulse} shows the envelope, Fourier transform, and chirp for a stable wave as well as a broken wave. In the case of the stable wave the envelope and the Fourier transform are Gaussian-esque, and the chirp is a very smooth function in close agreement with \cite{chen}. However, in the case of the broken wave the envelope and Fourier transform are oscillatory. Furthermore, the chirp is highly erratic. \\

The breaking of the wave is not as predictable as one may expect---the structure of the boundary is quite rich. Note that in Figure \ref{fig:pulse} the wave breaks for the smaller of the two $b$ values. This structure is highlighted in Figure \ref{fig:DMmap} where the error is calculated by
\begin{align*}
\frac{\|A(\text{iteration } 40) - A(\text{iteration } 39) \|_2}{\|A(\text{iteration } 39) \|_2}.
\end{align*}
Additionally, Figure \ref{fig:energy} shows the energy of the pulse within the parameter space. \\

One interesting feature is that in the region $0.25 < s < 0.55$ the wave does not break regardless of $b$. This is because $b|A|^2$ is approximately constant for a particular $s$ value, and so the phase shift added by the nonlinearity is constant, thus the wave is stable. \\

Since some of the individual mappings are nonlinear they do not commute---changing the order of the components should lead to different results. The nonlinearity should still follow the gain component since that is where it is most dominant. Furthermore, the loss is a linear mapping and so it will commute with both dispersion and modulation. Therefore, the only other unique case to consider is if dispersion follows modulation. \\

Figure \ref{fig:switch} shows the effect of switching these two components. Overall, the large scale structure is unchanged with the exception of $0.3 < s < 0.55$ and $20 < b$. In this region the waveform appears to have reached a period 2 equilibrium, that is, $\mathcal{L}(\mathcal{L}(A)) = A$.
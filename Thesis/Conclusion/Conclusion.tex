% !TeX root = ../Thesis.tex

\chapter{Conclusion}
By expanding upon the ideas originally proposed by Cutler \cite{cutler}, and Kuizenga and Siegman \cite{kuizenga1970, kuizenga1970a, siegman}, we developed a nonlinear functional model for tuneable lasers. By omitting the nonlinearity, we obtain similar linearly chirped Gaussian solutions. However, a consequence of showing that Gaussians span $L^2(\mathbb{R})$ is that the linear model has been solved regardless of the choice of the modulation function. \\

In contrast, with the inclusion of the nonlinearity we were able to demonstrate wave breaking, and found a peculiar boundary of stability. This phenomenon has been demonstrated in a laboratory setting \cite{agrawal2013, anderson, finot, rothenberg, tomlinson}, but, previous mathematical models have been unable to capture it thus far. The nonlinearity induces \gls{spm} which injects higher frequency modes to the pulse. This causes the Fourier transform to become bimodal instead of Gaussian. As the nonlinearity increases, so too does the separation between these two modes. If this effect becomes too strong, the wave becomes incoherent and deformed, and ultimately unsustainable---the wave has broken. \\

The shape of this boundary between a stable pulse, and an unstable broken pulse is quite intriguing, and surprisingly sharp. At a critical $s$ value, about 0.3 in our simulations, the boundary becomes periodic in the parameter space---this arises from the nature of the nonlinearity. In a sense the nonlinearity is periodic in $b$, and thus, the stability of the solution is as well. \\

One of the issues experimentalists have is designing their tuneable lasers. They are uncertain how their laser will behave---and if the laser will lase at all---until they actually construct it. This model serves to provide better design principles, and how to troubleshoot should they encounter wave breaking. For example, from our results it is clear that to fix a broken wave $b$ must be decreased, or $s$ increased. This could be achieved by reducing the length of the optical fibre between the gain fibre and the output coupler, or by increasing $\beta_2 L_D$ by adding a second \gls{cfbg} in series, respectively\footnote{See \eqref{eq:ndparam}.}. Additionally, one could decrease  the modulation time, $T_M$, to move through the parameter space along a ray outward from the origin. Although this would increase $b$, it also could increase $s$ enough to move out of the unstable region. \\

\section{Future Work}
There are a few avenues to explore as a continuation of this project. The first is to compare to experimental data to validate the findings. However, due to the ultrashort duration of the pulses it becomes difficult to accurately measure the envelope and spectrum---autocorrelation methods must be used. This makes it troublesome to make the distinction between a Gaussian envelope or the generalized Gaussian envelope we've found. \\

An analytic development of interest is to find the asymptotic expansion for small values of the nonlinearity parameter---that is, $b \rightarrow 0$. This is likely to provide insight to how the nonlinearity  impacts the linearized solution to better understand the manifestation of wave breaking. Furthermore, to probe the underlying structure shown in Figure \ref{fig:error} a bifurcation diagram could be constructed or analytic continuation could be used to observe the sensitivity to the parameters. \\


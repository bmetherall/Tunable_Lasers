% !TeX root = ../Thesis.tex

\chapter{Conclusion}
By expanding upon the ideas originally proposed by Cutler \cite{cutler}, and Kuizenga and Siegman \cite{kuizenga1970, kuizenga1970a, siegman}, we developed a nonlinear functional model for tuneable lasers. By omitting the nonlinearity, we obtain similar linearly chirped Gaussian solutions. However, a consequence of showing Gaussians span $L^2(\mathbb{R})$ is that this model has been solved regardless of the choice of the modulation function. \\

In contrast, with the inclusion of the nonlinearity we were able to demonstrate wave breaking, and found a peculiar boundary of stability. This phenomenon has been demonstrated in a laboratory setting \cite{agrawal2013, anderson, finot, rothenberg, tomlinson}, but, mathematical models have been unable to thus far. The nonlinearity induces self phase modulation which injects higher frequency modes to the pulse. This causes the Fourier transform to become a bimodal distribution instead of Gaussian. As the nonlinearity increases so too does the separation between these two modes. If this effect becomes too strong, the wave becomes incoherent and deformed, and ultimately unsustainable---the wave has broken. \\

The shape of this boundary between a stable pulse, and an unstable broken pulse is quite intriguing, and surprisingly sharp. At a critical $s$ value, about 0.3 in our simulations, the boundary becomes periodic---this arises from the nature of the nonlinearity. In a sense the nonlinearity is periodic in $b$, and thus, the stability of the solution is as well. \\


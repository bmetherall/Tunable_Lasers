% !TeX root = ../Thesis.tex

\chapter{A New Model}
\label{chap:linear}
In this chapter we shall derive our new model and solve it analytically in the linear case. To accomplish this we shall use the ideas of the previous functional models \cite{cutler, siegman, kuizenga1970a, kuizenga1970b, kuizenga1970, martinez1984, martinez1985, burgoyne2014}. To alleviate some of the pitfalls mentioned in the previous chapter, we shall include all five processes involved in the modification of the pulse within the laser cavity (gain, nonlinearity, loss, dispersion, and modulation). In addition to this, the functional operations associated with each component will be derived from \eqref{eq:nlse}, with the exception of the modulation in which we consider the exact functional form to be determined by the laser operator. \\

\section{Components}
\label{sec:comp}
We shall begin our analysis with the derivation of the functional operators for the five components. \\

\subsection{Gain}
\label{chap:gain}
Within the Er-doped gain fibre, the gain term is dominant, and equation~\eqref{eq:nlse} reduces to
\begin{align}
\label{eq:gainde}
\pdiff{A}{z} &= \frac{1}{2} g(A) A,
\end{align}
where $g(A)$ takes the form \cite{bohun, burgoyne2014, hausbook, haus1975, haus1991, haus1992, haus2000, kartner, peng, shtyrina, silfvast, usechak, yarutkina}
\begin{align}
\label{eq:energy}
	g(A) &= \frac{g_0}{1 + E / \Es},& E &= \int_{-\infty}^\infty |A|^2 \, \df T,
\end{align}
where $g_0$ is a small signal gain, $E$ is the energy of the pulse, and $\Es$ is the energy at which the gain begins to saturate.
Multiplying~\eqref{eq:gainde} by $\bar{A}$, the complex conjugate of $A$, yields
\begin{align*}
	\bar{A} \pdiff{A}{z} = \frac{1}{2}\frac{g_0 |A|^2}{1 + E / \Es},
\end{align*}
adding this to its complex conjugate gives
\begin{align}
\pdiff{|A|^2}{z} &= \frac{g_0 |A|^2}{1 + E / \Es}.
\label{eq:diffea}
\end{align}
After integrating this becomes
\begin{align}
\diff{E}{z} &= \frac{g_0 E}{1 + E / \Es}.
\label{eq:diffez}
\end{align}
For $E \ll \Es$ the energy will grow exponentially, whereas for $E \gg \Es$ the gain has saturated and so the growth is linear. To obtain a closed form solution, \eqref{eq:diffez} is integrated over a gain fibre of length $L_g$, the length of the gain fibre, and assuming the energy increases from $E$ to $E_{\textrm{out}}$, then
\begin{align*}
	g_0 L_g = \log\frac{E_{\textrm{out}}}{E} + \frac{E_{\textrm{out}}-E}{\Es},
\end{align*}
and by exponentiating, rearranging, and applying $W$, the Lambert $W$ function\footnote{See Appendix \ref{chap:lambertw}.},
\begin{align*}
W \left(\frac{E}{\Es} \textrm{e}^{E/\Es} \textrm{e}^{g_0 L_g}\right) &= W\left(\frac{E_{\textrm{out}}}{\Es} \textrm{e}^{E_{\textrm{out}}/\Es}\right) = \frac{E_{\textrm{out}}}{\Es},
\end{align*}
by \eqref{eq:lambertw}. This results in the closed form expression
\begin{align*}
E_{\textrm{out}} = \Es W \left( \frac{E}{\Es} \textrm{e}^{E/\Es} \textrm{e}^{g_0 L_g} \right).
\end{align*}
Notice from \eqref{eq:diffea} and \eqref{eq:diffez} that
\begin{align*}
\frac{1}{|A|^2}\pdiff{|A|^2}{z} &= \frac{1}{E}\diff{E}{z}
\end{align*}
and so, $E \sim |A|^2$, therefore the gain in terms of the amplitude is given by
\begin{align*}
G(A;E) &= \left(\frac{E_{\textrm{out}}}{E}\right)^{1/2}A = \left( \frac{\Es}{E} W \left( \frac{E}{\Es} \textrm{e}^{E/\Es} \textrm{e}^{g_0 L_g} \right) \right)^{1/2} A
\end{align*}
after passing through the gain fibre once.

\subsection{Fibre Nonlinearity}
The nonlinearity of the fibre arises from the parameter $\gamma$; in regions where this effect is dominant expression~\eqref{eq:nlse} becomes
\begin{align}
\label{eq:fibrediff}
	\pdiff{A}{z} - i \gamma |A|^2 A = 0.
\end{align}
This expression can be manipulated in a similar manner to the gain to show $\frac{\partial}{\partial z} |A|^2 = 0$, suggesting the \acrshort{wkb} ansatz $A(z,T) = A_0(T) e^{i \varphi(T,z)}$. Substituting this representation into~\eqref{eq:fibrediff} and setting $\varphi(T,0)=0$ gives $\varphi(T,z) = \gamma |A|^2 z$. For a fibre of length $L_f$ the effect of the nonlinearity is given by the map $A \mapsto F(A)$, where
\begin{align*}
F(A) &= A \textrm{e}^{i \gamma |A|^2 L_f}.
\end{align*}
This is also frequently called the Kerr nonlinearity or Kerr effect after John Kerr who discovered the effect in 1875. As we shall see in Section \ref{chap:energy}, this is what is responsible for \acrlong{spm} and ultimately the degradation of the pulse \cite{desurvire, dunlop, martinez1984, tamura1996}.

\subsection{Loss}
Two sources of loss exist within the laser circuit: the loss due to the output coupler and the optical loss due to absorption and scattering. The first case is simply a scalar multiplication depending on the reflectivity of the output coupler. The loss due to absorption and scattering can be derived from \eqref{eq:nlse},
\begin{align*}
\pdiff{A}{z} = - \alpha A,
\end{align*}
and gives $\textrm{e}^{-\alpha L_T}A$, which by multiplying by the amount of the signal lost to the output coupler gives
\begin{align*}
L(A) &= (1 - R) \textrm{e}^{- \alpha L_T}A,
\end{align*}
where $R$ is the reflectivity of the output coupler, and $L_T$ is the total length of the laser circuit as the effect of the losses\footnote{Depending on the experimental setup the loss may take the form $L(A) = R \textrm{e}^{- \alpha L_T}A$.}.

\subsection{Dispersion}
Within the laser cavity, the dispersion is dominated by the \gls{cfbg}. In comparison, the dispersion due to the fibre is negligible. The dispersive terms of \eqref{eq:nlse} give
\begin{align}
\label{eq:dispde}
	\pdiff{A}{z} = -i \frac{\beta_2}{2} \pdiff[2]{A}{T},
\end{align}
and since dispersion acts in the frequency domain it is convenient to take the Fourier transform\footnote{We use $\displaystyle \FT{f} = (2 \pi)^{-1/2} \int_\mathbb{R} f(x) \textrm{e}^{i \omega x} \df x$ as the definition of the Fourier transform.} of \eqref{eq:dispde} \cite{debnath, gradshteyn}, giving
\begin{align*}
	\pdiff{}{z}\FT{A} &= i\frac{\omega^2}{2}\beta_2 \FT{A}.
\end{align*}
The effect of dispersion is then given by the map
\begin{align*}
D(A) &= \FTi{\textrm{e}^{i \omega^2 L_D\beta_2/2} \FT{A}},
\end{align*}
where $L_D$ is the length of the dispersive medium. \\

\subsection{Modulation}
In \eqref{eq:meml}, the amount of modulation is characterized by the parameter $M_s$ through the term $\frac{1}{2} M_s T^2 A$. In this new model, the modulation is considered to be applied externally through its action on the spectrum. For simplicity the representation is taken as the Gaussian
\begin{align*}
M(A) &= \textrm{e}^{-T^2 / 2 T_M^2} A,
\end{align*}
where $T_M$ is a characteristic width of the modulation. \\

Despite assuming a Gaussian modulation, the solution presented in Section \ref{sec:linear} can be generalized to any modulation function in $L^2(\mathbb{R})$ since Gaussians in fact span $L^2(\mathbb{R})$ as shown in Appendix \ref{chap:gauss}. \\

%% !TeX root = ../Thesis.tex
%\chapter{Span of Gaussians in $L^2$}
%\label{chap:gauss}
\section{Span of Gaussians in $L^2$}
In order to show that Gaussians span $L^2$, we shall start our analysis with the span of the Hermite polynomials in $L^2$. Typically the Hermite polynomials are recursively defined as \cite{conway, courant, teuwen}
\begin{align*}
H_n(x) := (-1)^n \textrm{e}^{x^2} \diff[n]{}{x} \textrm{e}^{-x^2},
\end{align*}
with the inner product\footnote{The complex conjugate of $g$ is omitted since the functions dealt with are real.}
\begin{align*}
\left< f, g \right> = \int_\mathbb{R} f(x) g(x) \textrm{e}^{-x^2} \df x,
\end{align*}
where $\textrm{e}^{-x^2}$ is the weighting function. This is so that 
\begin{align*}
\left< H_m, H_n \right> = \sqrt{\pi} 2^n n! \delta_{mn}
\end{align*}
and, therefore, the Hermite polynomials form an orthogonal set \cite{courant, hochstrasser, kreyszig, szego, teuwen}. Consider instead, the Gaussian--Hermite polynomials
\begin{align*}
\Her{n}{x} &:= \textrm{e}^{-x^2/2} H_n(x),
\end{align*}
with the inner product
\begin{align*}
\left< f, g \right> = \int_\mathbb{R} f(x) g(x) \, \df x,
\end{align*}
notice that the weighting function has been absorbed into the Hermite polynomials. We shall now show that the set of Gaussian--Hermite polynomials spans $L^2$. \\

\begin{theorem}
\label{thm:hermite}
The set of Gaussian--Hermite polynomials, $\mathcal{G}(\mathbb{R}) = \{ \Her{n}{x} : x \in \mathbb{R}, n \in \mathbb{N} \}$, spans $L^2(\mathbb{R})$.
\end{theorem}
\begin{proof}
The Gaussian--Hermite polynomials are an orthogonalization of the set $\Xi = \{x^n \textrm{e}^{-x^2/2} : x \in \mathbb{R}, n \in \mathbb{N}\}$, and therefore, have the same span \cite{kreyszig, szego}. Thus, it is sufficient to show that $\Xi$ spans $L^2$.
Suppose there exists an $f \in L^2$ such that $\left< f, \widetilde{H}_n \right> = 0$ for all $n$, or equivalently, that $\left< f, g_n \right> = 0$ for $g_n(x) = x^n \textrm{e}^{-x^2/2}$ for all $n$ so that it is not in the span of $\mathcal{G}(\mathbb{R})$. \\

Let us now consider the function \cite{courant, szego, teuwen}
\begin{align*}
F(z) = \frac{1}{\sqrt{2\pi}}\int_\mathbb{R} f(x) \textrm{e}^{zx} \textrm{e}^{-x^2/2} \, \df x,
\end{align*}
noting that $f(x) \textrm{e}^{zx} \textrm{e}^{-x^2/2} \in L^2(\mathbb{R})$ by the Cauchy--Schwarz inequality \cite{griffel, kolmogorov}, so $F$ converges, and that $F$ is holomorphic. The first exponential can be expanded into its Maclaurin series, then with Fubini's theorem we obtain
\begin{align*}
F(z) = \frac{1}{\sqrt{2\pi}} \sum_{n = 0}^{\infty} \frac{z^n}{n!} \int_\mathbb{R} f(x) x^n \textrm{e}^{-x^2/2} \, \df x.
\end{align*}
Recall, that by assumption
\begin{align*}
\int_\mathbb{R} f(x) x^n \textrm{e}^{-x^2/2} \, \df x &= \left< f, g_n \right> = 0,
\end{align*}
for all $n$ and so $F(z) \equiv 0$. We now notice that
\begin{align*}
0 &= F(z) = F(i \omega) = \FT{f \textrm{e}^{-x^2/2}},
\end{align*}
and now it is clear that $f(x) \textrm{e}^{-x^2/2} = 0$ almost everywhere, therefore, $f(x) = 0$ almost everywhere. \\
\end{proof}

Alternatively, without much difficulty it can be shown that the Gaussian--Hermite polynomials satisfy
\begin{align}
\label{eq:paracyl}
\diff[2]{\Her{n}{x}}{x} - \left( x^2 - 2n - 1 \right) \Her{n}{x} = 0.
\end{align}
This has the form of the parabolic cylinder functions \cite{miller}, and indeed the Gaussian--Hermite polynomials can be expressed in terms of parabolic cylinder functions \cite{hochstrasser, miller}. Moreover, \ref{eq:paracyl} can also be written in Sturm--Liouville form as
\begin{align*}
\diff{}{x} \left( 1 \cdot \diff{\Her{n}{x}}{x} \right) + (1 - x^2) \Her{n}{x} = -2n \Her{n}{x}.
\end{align*}
By the spectral theorem \cite{griffel, higson, kreyszig}, this suggests that the Gaussian--Hermite polynomials form an orthogonal set with the weighting function $w(x) = 1$, and are complete in $L^2(\mathbb{R})$. Using this result, we can now show that Gaussian functions also span $L^2(\mathbb{R})$ with the following theorem. \\

\begin{figure}[tbp]
\begin{subfigure}{0.5\textwidth}
\input{./Figures/GS1}
\caption{$N = 20$, $h = 0.1$}
\end{subfigure}
\begin{subfigure}{0.5\textwidth}
\input{./Figures/GS2}
\caption{$N = 10$, $h = 0.1$}
\end{subfigure}
\caption{Two examples of Gaussian series.}
\label{fig:gs}
\end{figure}

\begin{theorem}
\label{thm:gaussian}
Gaussians of a single variance span $L^2(\mathbb{R})$.
\end{theorem}
\begin{proof}
Theorem \ref{thm:hermite} showed that any square integrable function can be expressed as
\begin{align*}
f(x) = \sum_{n = 0}^\infty a_n \Her{n}{x}.
\end{align*}
Using a similar idea to \cite{calcaterra2, calcaterra}, we shall now expand the Gaussian--Hermite polynomials using their definition
\begin{align*}
f(x) = \sum_{n = 0}^\infty a_n (-1)^n \textrm{e}^{x^2/2} \diff[n]{}{x} \textrm{e}^{-x^2}.
\end{align*}
Finally, we can rewrite the derivatives using central differences \cite{zwillinger} so that
\begin{align}
f(x) = \sum_{n = 0}^\infty a_n (-1)^n \textrm{e}^{x^2/2} \left[ \lim_{h \rightarrow 0} \frac{1}{h^n} \sum_{i = 0}^n (-1)^i \binom{n}{i} \exp\left( -\left( x + h\left( \frac{n}{2} - i \right) \right)^2 \right) \right].
\label{eq:gs}
\end{align}
\end{proof}

\begin{figure}[tbp]
\input{./Figures/GaussMod}
\caption[Gaussian series for a particular modulation function.]{$N = 20$, $h = 0.1$}
\label{fig:gaussmod}
\end{figure}

Two examples of Gaussian series from \eqref{eq:gs} are shown in Figure \ref{fig:gs}. These examples are taken from \cite{calcaterra2, calcaterra}, however, we achieve similar or better approximations with either fewer terms, or with an $h$ value an order of magnitude larger. There are two reasons for this, first, we use central difference as opposed to backwards difference yielding a convergence of $\bigO{h^2}$ instead of $\bigO{h}$. Furthermore, because \cite{calcaterra2, calcaterra} used backwards differences, the means of the Gaussians are all non-negative, whereas with central differences, our Gaussians' means are both positive and negative---this leads to smaller coefficients, and better numerical stability. Moreover, Figure \ref{fig:gaussmod} shows the Gaussian series for the modulation function in \cite{bohun, burgoyneemail}.



\section{Non-Dimensionalization}

%\begin{figure}[tbp]
%\centering
%\begin{subfigure}[]{\columnwidth}
%\input{Shape}
%\caption{Envelope}
%\label{fig:envelope}
%\end{subfigure} \\
%\begin{subfigure}[]{\columnwidth}
%\input{Transform}
%\caption{Fourier Transform}
%\label{fig:fourier}
%\end{subfigure} \\
%\begin{subfigure}[]{\columnwidth}
%\input{Chirp}
%\caption{Chirp}
%\end{subfigure}
%\caption{Simulation with $s = 0.09$, $a = 8 \times 10^3$, $h = 0.04$, and $E_0 = 0.1$ with $A_0 = \Gamma \sech{(2T)} \textrm{e}^{i \pi / 4}$. $\Gamma$ is chosen such that $\int_{-\infty}^\infty |A|^2 \, \textrm{d}T = E_0$, and hyperbolic secant is chosen since it is a common soliton. In the stable case $b = 1.32$, whereas for the broken case $b = 1.25$. The pulses are shown after 25 iterations.}
%\label{fig:pulse}
%\end{figure}

The structure of each process of the laser can be better understood by re-scaling the time, energy, and amplitude. Nominal values for tuneable lasers are shown in Table~\ref{tab:values}. Knowing experimental durations and energies, the table suggests the convenient scalings:
\begin{align}
	T &= T_M \widetilde{T},& E &= \Es \widetilde{E},& A &= \left( \frac{\Es}{T_M} \right)^{1/2} \widetilde{A}.
\end{align}
Revisiting each process map shows each process has a characteristic non-dimensional parameter. The new mappings, after dropping the tildes, are
\begin{subequations}
\label{eq:effects}
\begin{align}
\label{eq:gain}
G(A) &= \left(E^{-1} W \left( a E \textrm{e}^{E}\right) \right)^{1/2} A, \\
\label{eq:fibre}
F(A) &= A \textrm{e}^{i b |A|^2}, \\
\label{eq:loss}
L(A) &= h A, \\
\label{eq:disp}
D(A) &= \FTi{\textrm{e}^{i s^2 \omega^2} \FT{A}}, \\
\label{eq:mod}
M(A) &= \textrm{e}^{-T^2 / 2} A,
\end{align}
\end{subequations}
with the four dimensionless parameters, as defined by the values in Table \ref{tab:values},
\begin{equation}
\begin{aligned}
	a &= \textrm{e}^{g_0 L_g} \sim 8 \times 10^3,& \qquad h &= (1 - R) \textrm{e}^{-\alpha L} \sim 0.04, \\
	b &= \gamma L_f \frac{\Es}{T_M} \sim 1,& \qquad s &= \sqrt{\frac{\beta_2 L_D}{2 T_M^2}} \sim 0.2,
\label{eq:ndparam}
\end{aligned}
\end{equation}
which characterize the behaviour of the laser. Notice that the modulation is characterized by $T_M$, and each other process has its own independent non-dimensional parameter.

\begin{table}[tbp]
\centering
\begin{tabular}{lcll}
Parameter & Symbol & Value & Sources \\
\noalign{\global\arrayrulewidth=1.5pt}\hline
Absorption of Fibre\footnotemark & $\alpha$ & $10^{-4}$--$0.3\text{ m}^{-1}$  & \cite{burgoyneemail, shtyrina, usechak, tomlinson, yarutkina} \\
Fibre Dispersion & $\beta_2^f$ & $-50$--$50 \text{ ps}^2/ \text{km}$ & \cite{agrawal2002, agrawal2013, burgoyne2014, litchinitser, peng, yarutkina} \\
Fibre Nonlinearity & $\gamma$ & $0.001$--$0.01 \text{ W}^{-1} \text{m}^{-1}$ & \cite{agrawal2013, finot, usechak, yarutkina} \\
Grating Dispersion & $\beta_2^g L_D$ & $10$--$2000 \text{ ps}^2$ & \cite{agrawal2002, agrawal2013, burgoyne2014, li} \\
Length of Cavity & $L_T$ & $10$--$100 \text{ m}$ & \cite{burgoyneemail, peng, tamura1996} \\
Length of Fibre & $L_f$ & $0.15$--$1 \text{ m}$ & \cite{burgoyneemail} \\
Length of Gain Fibre & $L_g$ & $2$--$3 \text{ m}$ & \cite{burgoyne2014, peng, shtyrina, tamura1993, yarutkina} \\
Modulation Time & $T_M$ & $15$--$150 \text{ ps}$ & \cite{bohun, burgoyne2014, burgoyneemail} \\
Reflectivity of Optical Coupler & $R$ & $0.1$--$0.9$ & \cite{burgoyneemail, peng, li, tamura1993, tamura1996, yamashita} \\
Saturation Energy & $\Es$ & $10^3$--$10^4 \text{ pJ}$ & \cite{burgoyneemail, usechak, yarutkina} \\
Small Signal Gain & $g_0$ & $1$--$10 \text{ m}^{-1}$ & \cite{burgoyneemail, yarutkina} \\
\hline
\end{tabular}
\caption{Range of variation of various parameters.}
\label{tab:values}
\end{table}

\footnotetext{Fibre loss is typically reported as $\sim0.5$ dB/km.}

\section{Combining the Effects of Each Block of the Model}
\label{sec:effects}
In this model the pulse is iteratively passed through each process, the order of which must now be considered. We are most interested in the output of the laser cavity, and so we shall start with the loss component. Next the pulse is passed though the \gls{cfbg}, as well as the modulator. Finally, the pulse travels through the gain fibre to be amplified, and then we consider the effect of the nonlinearity since this is the region where the power is maximized. Note that in general the functional operators of the components do not commute, and therefore the order of the components is indeed important---in contrast to the previous models. This is especially the case of dispersion as realized through the Fourier transform. The pulse after one complete circuit of the laser cavity is then passed back in to restart the process. Functionally this can be denoted as
\begin{align}
	\mathcal{L}(A) = F(G(M(D(L(A))))),
\end{align}
where $\mathcal{L}$ is one loop of the laser. A steady solution to this model is one in which the envelope and chirp are unchanged after traversing every component in the cavity, that is, such that $\mathcal{L}(A) = A \textrm{e}^{i \phi}$---for some $\phi \in \mathbb{R}$. The phase shift is allowable because it is immeasurable in the laboratory---only the envelope, chirp, and variance can be measured.

\section{Solution to the Linear Model}
\label{sec:linear}
In reality the nonlinearity parameter, $b$, is of order unity by \eqref{eq:ndparam}, but we shall start our investigation with the simpler case of $b = 0$---which shall be referred to as the linear model. In this case, a solution can be found analytically. It is expected the solution will take the form of a Gaussian. There are a few reasons for this; the solution to the previous models were Gaussian \cite{cutler, siegman, kuizenga1970a, martinez1984, martinez1985} (see Section \ref{chap:meml}), the equilibrium shape will be highly correlated to the shape of the modulation function, and since a Gaussian is a fixed point of the Fourier transform \cite{gradshteyn} dispersion will not alter the envelope. \\

\begin{figure}[tbp]
\centering
\input{./Figures/Sample_Gauss}
\caption{Sample realization of \eqref{eq:A0}.}
\label{fig:samplegauss}
\end{figure}

To compute $\mathcal{L}(A)$, consider the initial pulse
\begin{align}
	A_0 = \sqrt{P} \exp \left( -(1 + iC) \frac{T^2}{2 \sigma^2} \right) \textrm{e}^{i \phi_0},
\label{eq:A0}
\end{align}
where $P$ is the peak power, $C$ is the chirp\footnote{As discussed in Section \ref{sec:fbg}.}, $\sigma^2$ is the variance, and $\phi_0$ is the initial phase. An example of such a pulse is shown in Figure \ref{fig:samplegauss}. This form is chosen because it matches the envelope and linear chirping found by experiment. After passing through the optical coupler the pulse will simply decay to $A_1 = L(A_0) = h A_0$. The pulse then enters the \gls{cfbg}, where it will maintain its Gaussian shape, however, it will spread \cite{agrawal2013, ferreira, silfvast}. This can be written as
\begin{align}
A_2 = D(A_1) = \sqrt{P} h \zeta \exp \left( -(1 + i \widetilde{C}) \frac{T^2}{2 \widetilde{\sigma}^2} \right) \textrm{e}^{i(\phi_0 + \phi)},
\end{align}
where $\widetilde{\sigma}^2$ denotes the resulting variance, $\widetilde{C}$ denotes the resulting chirp, $\zeta$ is the reduction of the amplitude caused by the spread, and $\phi$ is the inherited phase shift from dispersion. At this point $\widetilde{\sigma}^2$, $\widetilde{C}$, $\zeta$, and $\phi$ remain undetermined, and are used as placeholders for convenience. Next, the pulse is modulated:
\begin{align}
A_3 = M(A_2) = \sqrt{P} h \zeta \exp \left( -(1 + i \widetilde{C}) \frac{T^2}{2 \widetilde{\sigma}^2} - \frac{T^2}{2} \right) \textrm{e}^{i(\phi_0 + \phi)}.
\end{align}
Finally, the pulse travels through the gain fibre where it is amplified to
\begin{align}
A_4 = G(A_3) = \mathcal{L}(A_0) = \sqrt{P} \left( \frac{W(a E \textrm{e}^E)}{E} \right)^{1/2} h \zeta \exp \left( -(1 + i \widetilde{C}) \frac{T^2}{2 \widetilde{\sigma}^2} - \frac{T^2}{2} \right) \textrm{e}^{i(\phi_0 + \phi)},
\label{eq:A4}
\end{align}
with $E$ the energy of the pulse as it enters the gain fibre. \\

In equilibrium it must be that $A_0 = A_4 \textrm{e}^{-i \phi} = \mathcal{L}(A_0) \textrm{e}^{- i \phi}$ so that after a single loop of the cavity the variance, chirp, and envelope remain unchanged. More explicitly, this gives three conditions:
\begin{subequations}
\begin{align}
\label{eq:energycond}
1 &= \left( \frac{W(a E \textrm{e}^E)}{E} \right)^{1/2} h \zeta, \\
\label{eq:varcond}
\frac{1}{\sigma^2} &= \frac{1}{\widetilde{\sigma}^2} + 1, \\
\label{eq:chirpcond}
\frac{C}{\sigma^2} &= \frac{\widetilde{C}}{\widetilde{\sigma}^2}.
\end{align}
\label{eq:linsys}
\end{subequations}

\subsection{Spread of the Pulse Due to Dispersion}
Each of these processes has a relatively straight forward effect, with the exception of dispersion.
% !TeX root = ../Thesis.tex

\chapter{Spread Due to Dispersion}

\cite{integrals}


\subsection{Equilibrium Shape of the Pulse}
Now that the effect of dispersion is know analytically, the system \eqref{eq:linsys} can be solved. The out-going variance and chirp can be eliminated from this system of equations using \eqref{eq:varcond} and \eqref{eq:chirpcond}:
\begin{align*}
\widetilde{\sigma}^2 &= \frac{\sigma^2}{1 - \sigma^2}, & \widetilde{C} &= C \frac{1}{1 - \sigma^2}.
\end{align*}
Combining this first expression with \eqref{eq:varout} and expanding the square, we have
\begin{align*}
\frac{\sigma^4}{1 - \sigma^2} = \sigma^4 + 4 C^2 s^4 + 4 C s^2 \sigma^2 + 4s^4,
\end{align*}
or written as a polynomial in $\sigma$,
\begin{align}
0 = \sigma^6 + 4 C s^2 \sigma^4 + \left( 4s^4 (C^2 + 1) - 4 C s^2 \right) \sigma^2 - 4s^4 (C^2 + 1).
\label{eq:midvar}
\end{align}
The chirp can now be eliminated using \eqref{eq:chirpout}, the $1+C^2$ can be reduced in order by noticing that
\begin{align*}
1 + C^2 &= \frac{\sigma^4}{2s^2(1 - \sigma^2)} C.
\end{align*}
Solving for $C$ gives
\begin{align}
\label{eq:equilchirp}
C &= \frac{\sigma^4}{2s^2(1 - \sigma^2)} \pm \sqrt{\frac{\sigma^8}{16s^4(1 - \sigma^2)^2} - 1} = \frac{\sigma^4 \pm \sqrt{\sigma^8 - 16s^4(1 - \sigma^2)^2}}{4s^2(1 - \sigma^2)},
\end{align}
and after simplifying \eqref{eq:midvar} algebraically, we arrive at
\begin{align}
\frac{\sigma^6}{2 - \sigma^2} &= \mp \sqrt{\sigma^8 - 16s^4(1 - \sigma^2)^2}.
\label{eq:varroot}
\end{align}
As we shall see, $\sigma$ is strictly less than $1$ at equilibrium, and so, notice that the left hand side of this expression is strictly positive. Therefore, only the negative root of \eqref{eq:equilchirp} will yield a solution. Moreover, Martinez, Fork, and Gordon \cite{martinez1984} found two solutions to their discrete model (Section \ref{sec:discrete}) analytically, but showed that one was unstable---consistent with experiments. After squaring each side of \eqref{eq:varroot} we obtain the biquartic equation
\begin{align}
\label{eq:var}
\left( \sigma^2 \right)^4 + 4 s^4 \left( \sigma^2 \right)^3 - 20 s^4 \left( \sigma^2 \right)^2 + 32 s^4 \left( \sigma^2 \right) - 16 s^4 = 0.
\end{align}
Since this is a quartic in $\sigma^2$ this can be solved analytically; the (positive, real) solution is
\begin{align}
\label{eq:equilvar}
\sigma^2 &= \sqrt{2} s \left( s^6 + 3s^2 + \sqrt{4 + s^4} \left( 1 + s^4 \right) \right)^{1/2} - s^4 - s^2 \sqrt{4 + s^4}.
\end{align}
A sample solution for the linear model is shown in Figure \ref{fig:linear}. To reiterate, the envelope at equilibrium is Gaussian, and therefore, so too is the Fourier transform. Also, the pulse is linearly chirped. \\

\begin{figure}[tbp]
\centering
\begin{subfigure}{0.5\textwidth}
\centering
\input{./Figures/Linear_Shape}
\caption{Envelope}
\end{subfigure}%
\begin{subfigure}{0.5\textwidth}
\centering
\input{./Figures/Linear_FT}
\caption{Fourier transform}
\end{subfigure} \\
\begin{subfigure}{0.5\textwidth}
\centering
\input{./Figures/Linear_Chirp}
\caption{Chirp}
\end{subfigure}
\caption[Envelope, Fourier transform, and chirp of the pulse---linear case.]{Equilibrium state of the pulse for the parameters $s = 0.15$ (dispersion), $b = 0$ (nonlinearity), $a = 8000$ (gain), and $h = 0.04$ (loss) as defined by \eqref{eq:equilchirp} and \eqref{eq:equilvar}.}
\label{fig:linear}
\end{figure}

\subsection{Asymptotic Expansion of the Variance}
\begin{figure}[tbp]
\centering
\input{./Figures/Asympt}
\caption{Plot of \eqref{eq:asymp} and \eqref{eq:asymp2} to highlight the structure.}
\label{fig:asympt}
\end{figure}
While \eqref{eq:equilvar} is useful, so too is the asymptotic behaviour. The asymptotic expansion provides additional information about the structure that the full analytic expression cannot. Moveover, the expansion provides more detail about the limits, as well has the rate of convergence. Lastly, the expansion allows for simplified expressions within their domains that are much less unwieldy than those found in the previous subsection. \\

% !TeX root = ../Thesis.tex

\chapter{Asymptotic Expansions of the Variance}
\label{chap:asymp}
We shall investigate the nature of the solution to \eqref{eq:var} both when $s \rightarrow 0$, as well as when $s \rightarrow \infty$. For ease of notation, \eqref{eq:var} is rewritten as
\begin{align*}
\frac{1}{4} x^4 = \eps \left( -x^3 + 5x^2 - 8x + 4 \right),
\end{align*}
where $x = \sigma^2$, $\eps = s^4$, and $\eps \rightarrow 0$. Additionally, suppose $x$ can be expanded as a power series in $\eps$:
\begin{align*}
x = x_0 + x_1 \eps^\alpha + x_2 \eps^\beta + \dots
\end{align*}
with $0 < \alpha < \beta$. Then,
\begin{align*}
\bigO1 : \frac{1}{4} x_0^4 = 0,
\end{align*}
and so, $x_0 = 0$. Knowing this, up to the next order we have
\begin{align*}
\frac{1}{4} x_1^4 \eps^{4\alpha} = \eps \left( -x_1^3 \eps^{3\alpha} + 5x_1^2 \eps^{2\alpha} - 8x_1 \eps^\alpha + 4 \right)
\end{align*}
in order to have dominant balance it must be that the left hand side balances with the final term of the right hand side, that is, $4\alpha = 1$, or $\alpha = \frac{1}{4}$. Therefore,
\begin{align*}
\bigO\eps : \frac{1}{4} x_1^4 &= 4, \\
(x_1 - 2)(x_1 + 2)(x_1^2 + 4) &= 0,
\end{align*}
and so, $x_1 = \pm 2, \, \pm 2 i$. However, we wish $x$ to be positive and real---recall that $x = \sigma^2$---thus, we take $x_1 = 2$. Now, the next lowest order of the left hand side must be $\eps^{3\alpha + \beta} = \eps^{3/4 + \beta}$. As with the previous iteration, this must balance with the lowest order term of the $8x$. Hence, $\frac{3}{4} + \beta = 1 + \alpha = \frac{5}{4}$, thus, $\beta = \frac{1}{2}$. Now,
\begin{align*}
\bigO{\eps^{5/4}}: \frac{1}{4} 4 x_1^3 x_2 &= -8x_1, \\
x_2 &= -2,
\end{align*}
where the additional $4$ comes from the binomial expansion. Finally, up to second(?***) order
\begin{align*}
x &\approx 2 \eps^{1/4} - 2 \eps^{1/2}, \\
\sigma^2 &\approx 2 \left( s - s^2 \right),
\end{align*}
this is shown in Figure \ref{fig:lims0}.
\begin{figure}[tbp]
\input{./Figures/Lim_s_0}
\caption{Asymptotic expansion of the variance as $s \rightarrow 0$.}
\label{fig:lims0}
\end{figure}

We shall now consider the other limit, when $s \rightarrow \infty$. Using a similar substitution we instead write \eqref{eq:var} as
\begin{align*}
\frac{1}{4} \eps x^4 = -x^3 + 5x^2 - 8x + 4,
\end{align*}
where $x = \sigma^2$, and $\eps = s^{-4}$ instead so that we still have $\eps \rightarrow 0$. Furthermore, we must make a correction to the series expansion:
\begin{align*}
x = \eps^{-\xi} \left( x_0 + x_1 \eps^\alpha + x_2 \eps^\beta + \dots \right).
\end{align*}
The reason for this is because the equation is now singular---when $\eps=0$ the equation transforms from a quartic to a cubic, losing a root. As with before, we obtain dominant balance when the quartic term balances with the cubic term:
\begin{align*}
1 - 4 \xi &= -3 \xi, \\
\xi &= 1.
\end{align*}
Then,
\begin{align*}
\bigO{\eps^{-3}}: \frac{1}{4} x_0^4 &= -x_0^3, \\
x_0^3 (x_0 + 4) &= 0,
\end{align*}
we shall choose $x_0 = 0$---since $x \geq 0$. Unlike up until this point, we obtain dominant balance when the right hand side dominates the left hand side, this is achieved when $\alpha = 1$. Now,
\begin{align*}
\bigO1 : 0 &= -x_1^3 + 5x_1^2 - 8x_1 + 4, \\
&= -(x_1 - 1)(x_1 - 2)^2,
\end{align*}
so either $x_1 = 1$ or $x_1 = 2$, we shall see in the next step that we can eliminate one of these. The next term obtains dominant balance when the left hand side balances with the linear term of the right hand side, that is, when $\beta = 2$:
\begin{align*}
\bigO\eps : \frac{1}{4} x_1^4 &= -3x_1^2 x_2 + 5\cdot2 x_1 x_2 - 8x_2, \\
&= -x_2 (3x_1 - 4)(x_1 - 2),
\end{align*}
where again, the coefficients come from the binomial expansions;
from this it is clear that
\begin{align*}
x_2 &= \frac{-x_1^4}{4(3x_1 - 4)(x_1 - 2)},
\end{align*}
and that $x_1 \ne 2$. Thus, $x_1 = 1$, $x_2 = -\frac{1}{4}$, and
\begin{align*}
x = 1 - \frac{1}{4} \eps + x_2 \eps^{\gamma - 1} + \dots.
\end{align*}
Expanding one final term we find $\gamma = 3$, and
\begin{align*}
\bigO{\eps^2}: \frac{1}{4} 4x_2 x_1^3 &= -(3x_3 x_1^2 + 3x_2^2 x_1) + 5(2x_3 x_1 + x_2^2) - 8(x_3).
\end{align*}
Knowing the values of $x_1$, and $x_2$ this is simply an arithmetical calculation yielding $x_3 = \frac{3}{8}$. Finally,
\begin{align*}
x &\approx 1 - \frac{1}{4}\eps + \frac{3}{8}\eps^2, \\
\sigma^2 &\approx 1 - \frac{1}{4s^4} + \frac{3}{8s^8},
\end{align*}
this approximation is shown in Figure \ref{fig:limsinfty}.
\begin{figure}[tbp]
\input{./Figures/Lim_s_Infty}
\caption{Asymptotic expansion of the variance for $s \rightarrow \infty$.}
\label{fig:limsinfty}
\end{figure}











\subsection{Equilibrium Energy}
Up until this point the $\sqrt{P}$ factor has remained undetermined, however, it can be found once the equilibrium energy is known. From~\eqref{eq:energycond} the equilibrium energy can be found, as well as the equilibrium peak power. This relation can be simplified by squaring both sides and rearranging to give
\begin{align*}
\frac{1}{h^2 \zeta^2} E = W \left( a E \textrm{e}^E \right),
\end{align*}
where, once again, $W$ is the Lambert $W$ function. Then, by taking the exponential of each side, and multiplying by this expression, we obtain\footnote{By \eqref{eq:lambertw}.}
\begin{align*}
\frac{1}{h^2 \zeta^2} E \exp \left(\frac{1}{h^2 \zeta^2} E \right) &= W \left( a E \textrm{e}^E \right) \exp \left( W \left( a E \textrm{e}^E \right) \right) = a E \textrm{e}^E.
\end{align*}
Now, this can be written as 
\begin{align*}
a h^2 \zeta^2 &= \exp \left( \frac{1}{h^2 \zeta^2}E - E \right), \\
\log \left( a h^2 \zeta^2 \right) &= E \left( \frac{1}{h^2 \zeta^2} - 1 \right).
\end{align*}
The energy of the pulse entering the gain fibre at equilibrium is thus
\begin{align*}
E = \frac{h^2 \zeta^2}{1 - h^2 \zeta^2} \log \left( a h^2 \zeta^2 \right). \\
\end{align*}

This expression allows us to determine a restriction on the parameters for a solution of the form \eqref{eq:A0} to exist---in order for this energy to be positive, $a h^2 \zeta^2 > 1$. The energy of the pulses as it enters the optical coupler can now be found, recall from \eqref{eq:gainde} that the energy is defined as
\begin{align*}
E = \int_{-\infty}^\infty |A|^2 \, \df T;
\end{align*}
the energy entering the optical coupler is then
\begin{align}
E_* &= \int_{-\infty}^\infty |G(A)|^2 \, \df T = \frac{W(a E \textrm{e}^E)}{E} \int_{-\infty}^\infty |A|^2 \, \df T = W(a E \textrm{e}^E).
\label{eq:equilenergy}
\end{align}

We can now find the amplitude of the pulse as well now that the energy is known---as we have previously found the equilibrium shape. Again, from \eqref{eq:gainde}, it must be that $E_* = \sqrt{\pi} P \sigma$, or,
\begin{align}
P = \frac{W(a E \textrm{e}^E)}{\sqrt{\pi} \sigma}.
\label{eq:equilpower}
\end{align}
The asymptotic expansions when $s \rightarrow 0$ of the energy, and peak power are---for completeness---
\begin{align}
E_* &= \Lambda - \Theta \frac{\Lambda}{1 + \Lambda} s + \bigO{s^2}, & P &= \frac{\Lambda}{\sqrt{2 \pi s}} + \left( \frac{\Lambda}{2} -  \Theta \frac{\Lambda}{1 + \Lambda} \right) \sqrt{\frac{s}{2 \pi}} +  \bigO{s^{3/2}},
\label{eq:epasymp}
\end{align}
respectively, where
\begin{align*}
\Lambda &= W\left( \frac{a h^2 \ln(a h^2)}{1 - h^2} \exp \left( \frac{h^2 \ln(a h^2)}{1 - h^2} \right) \right),
\end{align*}
and
\begin{align*}
\Theta &= \frac{\left( h^2 \ln(a h^2) + 1 - h^2 \right) \left( \ln(a h^2) + 1 - h^2 \right)}{\left( 1 - h^2 \right)^2 \ln(a h^2)}.
\end{align*}

\section{Chapter Summary}
In this chapter we first derived the effect each of the five components---gain, nonlinearity, loss, dispersion, and modulation---has on the incoming pulse from \eqref{eq:nlse}. Furthermore, to highlight the structure of these effects the system was non-dimensionalized. This yielded the functional maps \eqref{eq:effects} for the components, with corresponding non-dimensional parameters \eqref{eq:ndparam}. With all of the operators defined, we proceeded to compose each process, in the order most representative of a tuneable laser, to give the effect of one complete circuit around the laser cavity. \\

We then sought a solution to the linearized model---having omitted the nonlinearity component. In this case, we were able to find a closed form analytic solution for the equilibrium state. To do this, we assumed the solution took the form of a linearly chirped Gaussian, \eqref{eq:A0}, which was then passed through each process map yielding \eqref{eq:A4} in the end. The fixed point of the linearized model was found by equating \eqref{eq:A0} and \eqref{eq:A4}, but, allowing a phase shift to have occured. The equilibrium variance, chirp, energy, and peak power were found to be \eqref{eq:equilvar}, \eqref{eq:equilchirp}, \eqref{eq:equilenergy}, and \eqref{eq:equilpower}, respectively. To gain a better understanding of the limiting behaviour, the asymptotic expansions were investigated. This yielded \eqref{eq:asympt} for the asymptotic expansions of the variance and chirp, and \eqref{eq:epasymp} for the energy and peak power. \\

With the linear model fully solved, we shall now turn our attention to the rich structure added by the full nonlinear model. \\
\documentclass[oneside]{UOIT_Thesis}

\usepackage{wrapfig}
\usepackage{lipsum}
\usepackage{amsthm}

% Load tikz and libraries
\usepackage{tikz}
\usetikzlibrary{patterns,decorations.pathreplacing}
\usetikzlibrary{arrows.meta}
\usetikzlibrary{external}
\tikzexternalize[prefix=Figures/]

% New dots
\pgfdeclarepatternformonly{mydots}{\pgfqpoint{-1pt}{-1pt}}{\pgfqpoint{5pt}{5pt}}{\pgfqpoint{6pt}{6pt}}%
{
    \pgfpathcircle{\pgfqpoint{0pt}{0pt}}{.5pt}
    \pgfpathcircle{\pgfqpoint{3pt}{3pt}}{.5pt}
    \pgfusepath{fill}
}

% Code Stuff
\usepackage{listings}

\definecolor{codegreen}{rgb}{0,0.6,0}
\definecolor{codegray}{rgb}{0.5,0.5,0.5}
\definecolor{codepurple}{rgb}{0.58,0,0.82}
\definecolor{backcolour}{rgb}{0.95,0.95,0.92}
 
\lstdefinestyle{mystyle}{
    emph={self, Equation, loop, BlackHole2D,__init__, initialize, BlackHole, Application, Group},
    emphstyle=\color{PineGreen},
    commentstyle=\color{blue},
    keywordstyle={\color{BrickRed}\bfseries},
    numberstyle=\color{codegray},
    stringstyle=\color{codepurple},
    breakatwhitespace=false,         
    breaklines=true,                 
    captionpos=b,                    
    keepspaces=true,                 
    numbers=left,                    
    numbersep=5pt,                
    showspaces=false,                
    showstringspaces=false,
    showtabs=false,                  
    tabsize=2
}

\lstset{style=mystyle}

\let\originalleft\left
\let\originalright\right
\renewcommand{\left}{\mathopen{}\mathclose\bgroup\originalleft}
\renewcommand{\right}{\aftergroup\egroup\originalright}

\providecommand{\df}{\textrm{d}}
\newcommand{\diff}[3][]{\frac{\textrm{d}^{#1}#2}{\textrm{d}{#3}^{#1}}}
\newcommand{\pdiff}[3][]{\frac{\partial^{#1}#2}{\partial{#3}^{#1}}}
\newcommand{\Es}{E_{\textrm{sat}}}
\newcommand{\FT}[1]{\mathcal{F}\left\{ #1 \right\}}
\newcommand{\FTi}[1]{\mathcal{F}^{-1}\left\{ #1 \right\}}
\newcommand{\Her}[2]{\widetilde{H}_{#1} \left( #2 \right)}
\newcommand{\eps}{\varepsilon}
\newcommand{\rect}[1]{\textrm{rect}\left( #1 \right)}

\providecommand{\bigO}[1]{\ensuremath{\mathop{}\mathopen{}\mathcal{O}\mathopen{}\left(#1\right)}}
\DeclareMathOperator{\sgn}{sgn}
\DeclareMathOperator{\sech}{sech}

\newtheorem{theorem}{Theorem}[chapter]

\title{A New Method of Modelling Tuneable Lasers with Functional Composition}
\author{Brady Metherall}
\degree{Master of Science}
\faculty{Science}
\program{Modelling and Computational Science}

%\let\originalcite\cite
%\renewcommand{\cite}[1]{\mbox{\originalcite{#1}}}

\begin{document}

\frontmatter
\pagenumbering{roman} 
\maketitle

\makeabstract{
A new nonlinear model is proposed for tuneable lasers. Using the generalized nonlinear Schr\"{o}dinger equation as a starting point, expressions for the transformations undergone by the pulse are derived for each component within the laser cavity. These transformations are then composed to give the overall effect of one trip around the cavity. The linear version of this model is solved analytically, and the nonlinear version numerically. A consequence of the nonlinear nature of this model is that it is able to exhibit wave breaking which prior models could not. We highlight the rich structure of the boundary of stability for a particular plane of the parameter space.
}

\makeacknowledgements{
First and most importantly, I would like to thank my supervisor and mentor, Dr. Sean Bohun, for his assistance, guidance, and inspiration over the past two years. I would also like to thank my fellow modelling and computational science students for helping make the last two years enjoyable. \\

Lastly, I would like to thank my mom and dad, and grandparents for their endless support and encouragement over the past two years.
}

\makedeclaration

{\hypersetup{linkcolor=black}
\maketableofcontents
}

\mainmatter

\pagenumbering{arabic}
\doublespacing

% !TeX root = ../Thesis.tex

\chapter{Introduction}

%Classical lasers, such as a laser pointer or a Helium-Neon gas laser, are limited to a single wavelength since the light is generated by stimulated emission. Tuneable lasers on the other hand, have the ability to operate within a range of wavelengths \cite{bohun, burgoyne2010, yamashita}. As a result, tuneable lasers have applications in spectroscopy and high resolution imaging such as coherent anti-Stokes Raman spectroscopy and optical coherence tomography \cite{bohun, burgoyne2014, yamashita}. This article is concerned with dispersion-tuned actively mode-locked (DTAML) lasers. The laser cavity consists of four elements: the dispersive element, the modulator, the gain fibre, and the optical coupler. The gain fibre consists of either an Erbium or Ytterbium doped fibre, and dispersion is generated by the highly dispersive chirped fibre Bragg grating (CFBG). \\

\section{Tuneable Lasers}
\begin{figure}[tbp]
\centering
% !TeX root = ./Tuneable_Lasers.tex

\begin{tikzpicture}
% Two laser loops
\draw [rounded corners=4mm] (0,0) rectangle ++(6,4);
\draw [rounded corners=4mm] (0,0) rectangle ++(-1.5,4);

% Gain
\draw (0.5,2.25) circle (0.5cm);
\draw (0.5,2) circle (0.5cm) node [anchor=west,xshift=0.5cm,align=center] {Er-doped \\ fibre};
\draw (0.5,1.75) circle (0.5cm);

% Modulator and pump
\filldraw[fill=white, draw=black] (2,-0.75) rectangle ++(2,1.5) node [midway] {Modulator};
\filldraw[fill=white, draw=black] (-2.1,1.5) rectangle ++(1.2,1) node [midway] {Pump};

% Coupler and output
\draw[-stealth] (3,4) -- (3,5.5) node [pos=0.75,anchor=west,xshift=0.25cm] {Laser output};
\draw[densely dashdotted] (2.5,3.5) -- (3.5,4.5) node [pos=1,anchor=north,yshift=-0.75cm,align=center] {Optical \\ coupler};

% Circulators
\filldraw[fill=white, draw=black] (6,2) circle (0.5cm);
\draw[->,>=stealth] (6,2.325) arc (90:360:0.325cm);

\filldraw[fill=white, draw=black] (0,0) circle (0.5cm);
\draw[->,>=stealth] (0,0.325) arc (90:360:0.325cm);

\filldraw[fill=white, draw=black] (0,4) circle (0.5cm) node [anchor=south,align=center,yshift=0.5cm] {Optical circulator};
\draw[->,>=stealth] (0,4.325) arc (90:360:0.325cm);

% Arrows
\draw [->,>=stealth,domain=20:70,blue] plot ({0.675*cos(\x)}, {0.675*sin(\x)});
\draw [->,>=stealth,domain=110:160,red] plot ({0.675*cos(\x)}, {0.675*sin(\x)});
\draw [->,>=stealth,domain=110:160,blue] plot ({6+0.675*cos(\x)}, {2+0.675*sin(\x)});
\draw [->,>=stealth,domain=200:250,blue] plot ({6+0.675*cos(\x)}, {2+0.675*sin(\x)});
\draw [->,>=stealth,domain=200:250,red] plot ({0.675*cos(\x)}, {4+0.675*sin(\x)});
\draw [->,>=stealth,domain=290:340,blue] plot ({0.675*cos(\x)}, {4+0.675*sin(\x)});

\draw [->,>=stealth,domain=270:360,blue] plot ({2.325+0.5*cos(\x)}, {4.675+0.5*sin(\x)});
\draw [->,>=stealth,blue] (2.125, 4.675-0.5) -- (2.325+0.6, 4.675-0.5);

% Grating
\draw (5.5,2) -- (3.5,2) node [pos=0.5,anchor=south,yshift=0.25cm,xshift=-0.15cm] {CFBG};
\foreach \i in {0,...,13}
	\draw (3.5 + \i*\i/100,1.75) -- (3.5 + \i*\i/100,2.25);

\end{tikzpicture}
\caption[Laser Cavity]{Laser cavity schematic.}
\label{fig:cavity}
\end{figure}

\subsection{Optical Coupler and Laser Output}
The optical coulper is a device that splits its input into two outputs---one continuing through the laser cavity, while the other exits the cavity to become the output of the laser. There are multiple devices that can accomplish this, however, the simplest is a partially reflecting mirror \cite{alazzawi}. These mirrors are characterized by their reflection coefficient, $R$. In the schematic shown in Figure \ref{fig:cavity}, the part of the signal that is reflected exits the cavity, whereas, the part of the signal transmitted remains within the cavity. \\

\subsection{Modulator}

\subsection{Fibre Bragg Grating}
A fibre Bragg grating (FBG) is an optical fibre where the refractive index varies periodically along its length \cite{ferreira}. To achieve this, typically silica fibres are doped with Germanium, which when exposed to intense ultraviolet (UV) light alters the refractive index of the core \cite{becker, starodoumov}. The photosensitivity of the optical fibres can be increased by more than an order of magnitude by the Germanium doping \cite{becker, ferreira}. \\

\begin{figure}[p]
\centering
\begin{subfigure}{\textwidth}
\centering
% !TeX root = ../Thesis.tex

\begin{tikzpicture}
\draw[pattern=north east lines, pattern color=blue] (-4,0.5) rectangle (4,1); % Top Insulation
\draw[pattern=north west lines, pattern color=blue] (-4,-0.5) rectangle (4,-1); % Bottom Insulation
\draw[pattern=mydots, pattern color=OliveGreen] (-4,-0.5) rectangle (4,0.5); % Core
\draw[fill=black] (-2,1.5) rectangle (2,1.75); % Phase Mask top

\foreach \i in {0,...,7}
  \draw[magenta, thick] (-1.75 + \i/2,5) -- (-1.75 + \i/2,1.5);

\foreach \i in {0,...,3}{
  % Phase mask teeth
  \draw[fill=black] (-2 + \i,1.25) rectangle (-1.75 + \i,1.5); 
  \draw[fill=black] (-1.25 + \i,1.25) rectangle (-1 + \i,1.5);
  % Rays
  \draw[->, >=stealth, magenta, thick] (-1.75 + \i,1.5) -- (-1.25 + \i,-1.5);
  \draw[->, >=stealth, magenta, thick] (-1.25 + \i,1.5) -- (-1.75 + \i,-1.5);
}
% Labels
\draw[->, >=stealth, thick] (3,-1.5) -- (2,0) node[pos=-0.2] {Ge-Doped Core};
\draw[->, >=stealth, thick] (-3,-1.5) -- (-2,-0.75) node[pos=-0.3] {Cladding};
\draw[->, >=stealth, thick] (-3,2.5) -- (-1.75,2) node[pos=-0.4] {UV Light};
\draw[->, >=stealth, thick] (3,2.5) -- (2,1.5) node[pos=-0.2] {Phase-Mask};
\end{tikzpicture}
\caption{\textbf{Phase-Mask:} The phase-mask refracts the UV light onto the core.}
\label{fig:phasemask}
\vspace{10mm}
\end{subfigure}
\begin{subfigure}{\textwidth}
\centering
% !TeX root = ../Thesis.tex

\begin{tikzpicture}

\draw[pattern=north east lines, pattern color=blue] (-4,0.5) rectangle (4,1); % Top Insulation
\draw[pattern=north west lines, pattern color=blue] (-4,-0.5) rectangle (4,-1); % Bottom Insulation
\draw[pattern=mydots, pattern color=OliveGreen] (-4,-0.5) rectangle (4,0.5); % Core

\foreach \i in {0,...,3}{
  % Rays
  \draw[->, >=stealth, magenta, thick] (-2.25 + \i,4.5) -- (-1.25 + \i,-1.5);
  \draw[->, >=stealth, magenta, thick] (-0.75 + \i,4.5) -- (-1.75 + \i,-1.5);
}
% Labels
\draw[->, >=stealth, thick] (3,-1.5) -- (2,0) node[pos=-0.2] {Ge Doped Core};
\draw[->, >=stealth, thick] (-3,-1.5) -- (-2,-0.75) node[pos=-0.3] {Cladding};
\draw[->, >=stealth, thick] (-3,2.5) -- (-1.82,2) node[pos=-0.4] {UV Light};
\end{tikzpicture}
\caption{\textbf{Holographic Side Exposure:} Two incident beams intersect at the core.}
\label{fig:holographic}
\end{subfigure}
\caption{Depictions of the two methods for manufacturing FBGs. The UV light is focused on the core causing periodic constructive and destructive interference.}
\label{fig:fbgmake}
\end{figure}

FBGs are manufactured using one of two methods---the phase-mask method \cite{agrawal2002,alazzawi,becker,starodoumov}, or the holographic side exposure method \cite{agrawal2002, alazzawi, becker, ferreira, starodoumov}---these are shown in Figure \ref{fig:fbgmake}. Both methods cause the periodic nature of the refractive index through interference. In the phase-mask method (Figure \ref{fig:phasemask}) a single beam of UV light passes through the phase-mask which acts as a series of lenses, focusing the light at the core---this causes a sinusoidal interference pattern. Similarly, in the holographic side exposure method (Figure \ref{fig:holographic}), two beams of UV light are instead used to create the interference pattern. \\

The purpose of FBGs is that they act as reflective filters \cite{agrawal2002, alazzawi, ferreira, starodoumov}. Due to the periodicity of the refractive index, light with the corresponding wavelength will be reflected with all others passing through. This wavelength is defined by the Bragg condition \cite{agrawal2002, alazzawi, becker, ferreira, silfvast, starodoumov}:
\begin{align}
\label{eq:bragg}
\lambda_B = 2 \Lambda \bar{n},
\end{align}
where $\lambda_B$ is the Bragg wavelength, $\Lambda$ is the period of the grating, and $\bar{n}$ is the average index of refraction. This reflects wavelengths close to $\lambda_B$, known as the stop-band. \\

\subsubsection{Chirped Fibre Bragg Grating}
\begin{figure}[tbp]
\begin{subfigure}{0.5\textwidth}
\input{./Figures/Unchirped}
\caption{Unchirped Gaussian pulse.}
\label{fig:unchirped}
\end{subfigure}
\begin{subfigure}{0.5\textwidth}
\input{./Figures/Chirped}
\caption{Linearly chirped Gaussian pulse.}
\label{fig:chirped}
\end{subfigure}
\caption{In a unchirped pulse the frequency is constant, however, in a chirped pulse the frequency varies along the envelope.}
\label{fig:chirp}
\end{figure}

Chirp is simply the term for a signal that has a non-constant frequency across it. Figure \ref{fig:chirp} shows examples of chirped and unchirped Gaussian pulses---the most common type of chirp is linear chirp, where the frequency varies linearly across the pulse. Because of this, the oscillations are characterized by $\exp \left( i C x^2 \right)$, where $C x$ is the linear variation of the wave number, and $C$ is the chirp parameter. By using a chirped phase-mask a chirped fibre Bragg grating (CFBG) can be created. Since the period of the refractive index varies across the CFBG, so does when the Bragg condition, \eqref{eq:bragg}, is satisfied. This causes most wavelengths to be reflected by a CFBG, but with each wavelength penetrating to a different depth. A consequence of this is that a time delay is created between wavelengths---this is depicted in Figure \ref{fig:cfbg} with the upper portion showing  the refractive index as a function of the depth. In this orientation, the red (dashed) wave is unable to penetrate as far as the blue (solid) wave since each wave is reflected where it matches the frequency of the refractive index.  \\

\begin{figure}[tbp]
\centering
\begin{tikzpicture}
\draw (4,0) -- (12,0) -- (12,4) -- (4,4) -- cycle;
\foreach \i in {1,...,12}
  \draw (12 - 1/18*\i*\i,0) -- (12 - 1/18*\i*\i,4);
%\foreach \i in {0,...,12}
%  \draw [dashed] (12 - 1/18*\i*\i,4) -- (12 - 1/18*\i*\i,8);
\draw [domain=0:(12-16/18), samples=1000, blue, thick] plot (\x, {-0.5*cos(4*pi*(\x-(12-25/18)) r) + 3});
\draw [domain=0:(12-81/18), samples=500, red, thick] plot (\x, {0.5*cos(2*pi*36/38*(\x-(12-100/18)) r) + 1});

\draw [thick, decoration={brace, mirror, raise=0.2cm}, decorate] (7.5,0) -- (100/9,0)
node [pos=0.5,anchor=north,yshift=-0.5cm] {Time Delay};
\draw [thick, decoration={brace, mirror, raise=0.2cm}, decorate] (12,0) -- (12,4)
node [pos=0.5,anchor=west,xshift=0.5cm] {CFBG};

\draw [->] (4,5) -- (12.5,5)
node [pos=0.47,anchor=north,yshift=-0.25cm] {$z$};
\draw [->] (4,5) -- (4,8)
node [pos=0.5,anchor=east,xshift=-0.2cm] {$n(z)$};
\foreach \i in {0,...,12}
  \draw [domain=(12-\i*\i/18):(12-(\i-1)*(\i-1)/18), samples=100, black, thick] plot (\x, {3/4*cos(2*pi*36/(4*\i-2)*(\x-(12-\i*\i/18)) r) + 6.5});
\end{tikzpicture}
\caption[CFBG]{CFBG}
\label{fig:cfbg}
\end{figure}

The speed of light in an optical fibre is slightly dependent on the wavelength---this causes light with a longer wavelength to travel faster, and is known as chromatic dispersion. This is a large problem in fibre optic communications, the signal can spread and potentially becomes uninterpretable after large distances. However, chromatic dispersion can be counteracted using a CFBG \cite{agrawal2002, alazzawi, becker, starodoumov} (in the opposite orientation of Figure \ref{fig:cfbg}). By forcing the longer wavelengths to travel farther the dispersion can be reversed, restoring the original signal. In one experiment \cite{dong}, a signal was successfully transmitted over $109$ km at $40$ Gb/s by compensating the dispersion with two  $40$ cm CFBGs. Over this distance, the pulse would have spread to about $55$ times its original width, and could only have been transmitted $4$ km at that bit rate. At reduced bit rates however, a $10$ cm CFBG can compensate the dispersion of $300$ km of fibre \cite{agrawal2002}. However, in our case, we wish to accelerate the dispersion, simulating hundreds of metres of fibre, so the CFBG is used in the orientation shown in Figure \ref{fig:cfbg}. \\

\subsection{Optical Circulator}
\begin{wrapfigure}{O}{0.5\textwidth}
\centering
% !TeX root = ../Thesis.tex

\begin{tikzpicture}
%Fibres
\draw (-2.25,0) -- (2.25,0);
\draw (0,2.25) -- (0,-2.25);

% Circulator
\filldraw[fill=white] (0,0) circle (1);
\draw[->,>=stealth] (0,0.65) arc (90:360:0.65);

%Labels
\node at (0.25, 1.5){$1$};
\node at (-1.5, 0.25){$2$};
\node at (-0.25, -1.5){$3$};
\node at (1.5, -0.25){$4$};
\end{tikzpicture}
\caption[Optical Circulator]{Symbol for a four port optical circulator.}
\label{fig:circulator}
\end{wrapfigure}
An optical circulator is a device that routes signals from port to port in a circular fashion \cite{agrawal2002, alazzawi, becker}, the symbol for a four port optical circulator is shown in Figure \ref{fig:circulator}. A signal entering from port 1 will be outputted from port 2; a signal entering from port 2 will exit from port 3; and so forth. Typically, optical circulators have three or four ports, with the first port being input only, and the final port being output only \cite{alazzawi}. Optical circulators are most commonly used with devices that reflect signals instead of transmit them. For example, a signal may enter through port 1, exit through port 2, be reflected by an FBG, re-enter port 2, and finally exit through port 3. \\

\subsection{Optical Amplifier and Pump Laser}
Optical amplifiers are of particular importance in fibre optic communications, they are used to restore the strength of a signal after it hass been attenuated over large distances, or when a signal is divided into multiple paths \cite{alazzawi, starodoumov}. They are also more efficient, and introduce less noise than an electrical repeater. In this context, the optical amplifier provides the energy of the laser \cite{alazzawi}. Most commonly, optical amplifiers are created  doping a length of fibre (called the gain fibre) with a rare-earth element which receives power from a pump laser \cite{agrawal2002, alazzawi, starodoumov}. The most common dopant is Erbium, however, Ytterbium and Neodymium are also used---Holmium, Samarium, Thulium, and Tellurium are infrequently used as well \cite{agrawal2002}. The amplification is achieved with stimulated emission---a similar process to spontaneous emission. \\

\begin{figure}[tbp]
\centering
% !TeX root = ../Thesis.tex

\begin{tikzpicture}
% Energy levels
\draw [dashed] (0,0) -- (12,0) node [pos = -0.05] {$E_1$};
\draw [dashed] (0,4) -- (12,4) node [pos = -0.05] {$E_2$};
\draw [dashed] (0,5) -- (4,5) node [pos = -0.15] {$E_3$};

% Transition arrows
\draw[->, >=stealth] (2,0.2) -- (2,4.8);
\draw[->, >=stealth] (2,5) -- (3-0.2*0.707,4+0.2*0.707);
\draw[<-, >=stealth] (6,0.2) -- (6,4);
\draw[<-, >=stealth] (10,0.2) -- (10,4);

% Circles
% Bottom
\filldraw[fill=white, draw=black, dashed] (2,0) circle (0.2);
\filldraw[fill=blue, draw=black] (6,0) circle (0.2);
\filldraw[fill=blue, draw=black] (10,0) circle (0.2);
% Top
\filldraw[fill=white, draw=black, dashed] (2,5) circle (0.2);
\filldraw[fill=blue, draw=black] (3,4) circle (0.2);
\filldraw[fill=white, draw=black, dashed] (6,4) circle (0.2);
\filldraw[fill=white, draw=black, dashed] (10,4) circle (0.2);

% Photons
\draw [domain=0.5:2, samples=250, red, thick, ->, >=stealth] plot (\x, {-0.5*sin(4*pi*deg(\x-2))});
\draw [domain=2:3.5, samples=250, red, thick, ->, >=stealth] plot (\x, {5-0.1*sin(4*pi*deg(\x-2))});
\draw [domain=6:7.5, samples=250, red, thick, ->, >=stealth] plot (\x, {4-0.5*sin(4*pi*deg(\x-6))});
\draw [domain=8.5:10, samples=250, red, thick, ->, >=stealth] plot (\x, {4-0.5*sin(4*pi*deg(\x-10))});
\draw [domain=10:11.5, samples=250, red, thick, ->, >=stealth] plot (\x, {4-sin(4*pi*deg(\x-10))});

% Labels
\node at (2, -1){Pumping};
\node [align=center] at (6, -1){Spontaneous \\ Emission};
\node [align=center] at (10, -1){Stimulated \\ Emission};
\end{tikzpicture}
\caption{In pumping, incident photons are absorbed by electrons which are then excited into a higher energy state. In spontaneous emission, an electron in an excited state emits a photon and returns to the lower energy state. Finally, in stimulated emission, an incident photon interacts with an electron in an excited state, the electron donates its energy to the photon and returns to the lower energy state.}
\label{fig:emission}
\end{figure}

Spontaneous emission is the process in which an electron of an atom that is in an excited state emits a photon as it transitions to a lower energy state. The energy of the photon is equal to the energy difference between the two levels. On the other hand, in stimulated emission an incident photon triggers this emission \cite{alazzawi}. The rare-earth metal ions are excited by the pump laser, and when they interact with a photon of the tuneable laser, a new photon is released with the same direction and phase as the incident one, maintaining coherence \cite{alazzawi}. These processes are highlighted in Figure \ref{fig:emission}. \\

Erbium-doped fibre amplifiers (EDFAs) are used most widely since Erbium has a band gap that corresponds to $1.54$--$1.57$ $\mu$m, which is the preferred band for fibre optics since this has the least power loss \cite{agrawal2002, alazzawi, starodoumov}. The pump laser typically operates at $980$ nm or $1.48$ $\mu$m because these are able to transfer the most power (up to $100$ mW) into the fibre while introducing minimal noise \cite{agrawal2002, alazzawi, becker, starodoumov}. The pump power can be applied either forwards (with the laser), backwards (against the laser), or both \cite{alazzawi}, with each configuration having similar performance \cite{agrawal2002}. However, backwards pumping has slightly better performance at high powers when the gain begins to saturate\footnote{This concept will be discussed in Section \ref{chap:gain}.} \cite{agrawal2002}. Additionally, with the configuration shown in Figure \ref{fig:cavity}, the optical circulators can be used to isolate the pump circuit from the rest of the laser cavity. \\

\section{Generalized Nonlinear Schr\"odinger Equation}
The standard equation for studying nonlinear optics is the generalized nonlinear Schr\"odinger equation (GNLSE) \cite{agrawal2013, burgoyne2007, ferreira, peng, shtyrina, yarutkina},
\begin{align}
\label{eq:nlse}
\pdiff{A}{z} &= - i \frac{\beta_2}{2}\pdiff[2]{A}{T} + \frac{\beta_3}{6}\pdiff[3]{A}{T} + i \gamma |A|^2 A + \frac{1}{2}g(A) A - \alpha A.
\end{align}
Here, $A$ is the complex pulse amplitude, $\beta_2$ and $\beta_3$ are the second-order (or group delay) and third-order dispersions, respectively. $\gamma$ is the coefficient of nonlinearity, $g(A)$ is an amplifying term due to the gain, and $\alpha$ is the loss due to scattering and absorption. \\

This equation can be derived from the nonlinear wave equation, the derivation is presented in detail in \cite{agrawal2013, ferreira}. Within the derivation comoving coordinates are used so that the reference frame propagates with the pulse at the group velocity. This is achieved with the substitution
\begin{align*}
T = t - \frac{z}{v_g}.
\end{align*}

The GNLSE takes the same form as the Schr\"odinger equation with the inclusion of the cubic nonlinear term, hence its name. For this reason, it is sometimes referred to as the cubic nonlinear Schr\"odinger equation. For intensities approaching $1 \text{ GW/cm}^2$, the $\gamma$ parameter must be replaced by $\gamma_0 (1 - b_s |A|^2)$, where $b_s$ is a saturation parameter \cite{agrawal2013}, this has the addition of a quintic term to incorporate nonlinearities associated with such large powers. Furthermore, the $\beta$ terms come from a Taylor expansion of the wavenumber, that is,
\begin{align*}
k(\omega) &= k_0 + \pdiff{k}{\omega}(\omega - \omega_0) + \frac{1}{2} \pdiff[2]{k}{\omega}(\omega - \omega_0)^2 + \frac{1}{6} \pdiff[3]{k}{\omega}(\omega - \omega_0)^3 + \dots, \\
&= \phi + \frac{1}{v_g}(\omega - \omega_0) + \frac{1}{2}\beta_2 (\omega - \omega_0)^2 + \frac{1}{6}\beta_3 (\omega - \omega_0)^3 + \dots,
\end{align*}
where $\phi$ is the phase shift. Typically, the third order effects must only be considered for ultrashort pulses---pulse widths less than $\sim5 \text{ ps}$---because of their large bandwidth \cite{agrawal2013}.

\section{Previous Modelling Efforts}
%To start off the discussion of the current modelling efforts for a tuneable laser, we begin with a review of the efforts to describe an `average' model. The idea is to capture some of the physical elements in the waveform described by an effective PDE, the solution of which gives the amplitude of the wave packet.


% !TeX root = ../Thesis.tex

\chapter{Previous Modelling Efforts}
In this chapter we shall explore the previous modelling efforts by first reviewing the classic equations used in nonlinear optics. We will then build upon this to the master equation of mode-locking. The solutions to this will briefly be studied before diving into `discretized' functional models. \\

\section{Generalized Nonlinear Schr\"odinger Equation}
The standard equation for studying nonlinear optics is the nonlinear Schr\"odinger equation (NLSE) \cite{agrawal2013, anderson, burgoyne2007, desurvire, ferreira, finot, rothenberg}
\begin{align*}
\pdiff{A}{z} &= - i \frac{\beta_2}{2}\pdiff[2]{A}{T} + i \gamma |A|^2 A.
\end{align*}
Here, $A$ is the complex pulse amplitude, $\beta_2$ is the second-order---or group delay---dispersion, and $\gamma$ is the coefficient of nonlinearity. In practice, this equation lacks a few key terms. Thus, it is often generalized by adding amplification, loss, and higher order terms. This gives the generalized nonlinear Schr\"odinger equation (GNLSE) \cite{agrawal2013, bohun, finot, peng, shtyrina, yarutkina},
\begin{align}
\label{eq:nlse}
\pdiff{A}{z} &= - i \frac{\beta_2}{2}\pdiff[2]{A}{T} + \frac{\beta_3}{6}\pdiff[3]{A}{T} + i \gamma |A|^2 A + \frac{1}{2}g(A) A - \alpha A,
\end{align}
where $\beta_3$ is the third order effect of dispersion, $g(A)$ is an amplifying term due to the gain, and $\alpha$ is the loss due to scattering and absorption. \\

This equation can be derived from the nonlinear wave equation, the derivation is presented in detail in \cite{agrawal2013, ferreira}. Within the derivation comoving coordinates are used so that the reference frame propagates with the pulse at the group velocity. This is achieved with the substitution
\begin{align*}
T = t - \frac{z}{v_g}.
\end{align*}

The GNLSE takes the same form as the Schr\"odinger equation with the inclusion of the cubic nonlinear term, hence its name. For this reason, it is sometimes referred to as the cubic nonlinear Schr\"odinger equation. For intensities approaching $1 \text{ GW/cm}^2$, the $\gamma$ parameter must be replaced by $\gamma_0 (1 - b_s |A|^2)$, where $b_s$ is a saturation parameter \cite{agrawal2013}, this has the addition of a quintic term to incorporate nonlinearities associated with such large powers. Furthermore, the $\beta$ terms come from a Taylor expansion of the wavenumber \cite{kartner}, that is,
\begin{align*}
k(\omega) &= k_0 + \pdiff{k}{\omega}(\omega - \omega_0) + \frac{1}{2} \pdiff[2]{k}{\omega}(\omega - \omega_0)^2 + \frac{1}{6} \pdiff[3]{k}{\omega}(\omega - \omega_0)^3 + \dots, \\
&= \phi + \frac{1}{v_g}(\omega - \omega_0) + \frac{1}{2}\beta_2 (\omega - \omega_0)^2 + \frac{1}{6}\beta_3 (\omega - \omega_0)^3 + \dots,
\end{align*}
where $\phi$ is the phase shift. Typically, the third order effects must only be considered for ultrashort pulses---pulse widths less than $\sim5 \text{ ps}$---because of their large bandwidth \cite{agrawal2013}.

\section{The Master Equation of Mode-Locking}
The GNLSE has many applications in nonlinear optics and fibre optic communications, however, in the context of lasers we wish to add a modulation term to ensure mode-locking, this yields the master equation of mode-locking, \cite{hausbook, haus1975, haus1986, haus1992, haus2000, kartner, tamura, usechak}
\begin{align*}
\pdiff{A}{z} &= - i \frac{\beta_2}{2}\pdiff[2]{A}{T} + \frac{\beta_3}{6}\pdiff[3]{A}{T} + i \gamma |A|^2 A + \frac{1}{2}g(A) A - \alpha A - M(T).
\end{align*}
The most common form of modulation is the sinusoid $M(T) = \frac{1}{2}\frac{M_s}{\omega_M} \left( 1 - \cos \left( \omega_M T \right) \right)$ \cite{hausbook, haus1975, haus1996, kartner}, and since $T$ is generally small, by it's definition, we expand via it's Taylor series so that $M(T) = \frac{1}{2}M_s T^2$. This brings us to the most common form of the master equation of mode-locking (neglecting the third order dispersion),
\begin{align}
\label{eq:meml}
\pdiff{A}{z} &= - i \frac{\beta_2}{2}\pdiff[2]{A}{T} + i \gamma |A|^2 A + \frac{1}{2}g(A) A - \alpha A - \frac{1}{2}M_s T^2.
\end{align}
Commonly, in an attempt to simplify the equation the gain assumed to be constant. As a whole no analytic solution is know for \eqref{eq:meml}, however, after additional simplifications there are three flavours of solutions. \\

In the least complicated case, the modulation and nonlinearity are omitted resulting in a solution in the form of a hyperbolic secant \cite{haus1975, haus1986, haus1992}. Now, with the inclusion of the nonlinearity a similar solution is found, however, it is now it the form of a chirped hyperbolic secant \cite{haus1991, usechak}. Both of these are unsurprising since this reduces to the soliton solution of the NLSE \cite{ferreira}. Lastly, by including the modulation term, and excluding the nonlinearity, \eqref{eq:meml} can be solved using separation of variables and one finds the solutions are the Gaussian--Hermite polynomials \cite{burgoyne2014, hausbook, haus1975, haus1996, haus2000, kartner, tamura, usechak}. However, in practice only the Gaussian is stable---the higher modes quickly decay within the laser \cite{hausbook, haus1975, haus1996, haus2000}. For a more comprehensive and exhaustive history see \cite{haus2000}. \\

\section{Discrete Functional Model}
While solution to the master equation yields reasonable results, it is not necessarily representative of what happens within the laser cavity. The issue with \eqref{eq:meml} is that it assumes each process effects the pulse continuously within the cavity. As highlighted by Figure \ref{fig:cavity}, this is a poor assumption. Within the cavity each effect is localized to its corresponding component: almost all of the dispersion happens within the CFBG, the pulse is only amplified within the Erbium-doped fibre, etc. Thus, perhaps a better model is one where \eqref{eq:meml} is broken down into the individual components giving the effect of each `block' of the cavity. Each of the blocks can then be composed together functionally to give the effect of one circuit around the cavity. This yields an algebraic equation instead of a differential one. \\

Such a method was first proposed in 1955 by Cutler \cite{cutler} while analyzing a microwave regenerative pulse generator. This method was adapted for mode-locked lasers in 1969 by Siegman and Kuizenga \cite{siegman}, which they then greatly added to the following year \cite{kuizenga1970a}. Kuizenga and Seigman also had success experimentally validating their model \cite{kuizenga1970b, kuizenga1970}. The effects of the nonlinearity would not be considered until Martinez, Fork, and Gordon \cite{martinez1984, martinez1985} tried to model passive mode-locking---mode-locking without the use of a modulator. In the absence of a modulator the nonlinearity becomes crucial to shaping the pulse. I don't know how to start this sentence*** when it was resurrected by Burgoyne \cite{burgoyne2014} in 2014 for tuneable lasers. \\

Despite these attempts, several short-comings exist. The clearest is that none of these models have contained each block--either the nonlinearity or the modulation have been omitted. In the context of tuneable lasers each component plays an important role. Another key drawback is that the functional operations of some of the components are somewhat phenomenological. While these functions are chosen based on the observed output, they are not necessarily correct. Finally, none of these previous models have been able to exhibit a phenomena called \emph{wave-breaking} in which the self-phase modulation (SPM) of the pulse becomes too strong distorting and damaging the wave until it ultimately becomes unstable and unsustainable. This notion will be explained in greater detail in Chapter \ref{chap:nl}. \\

% !TeX root = ../Thesis.tex

\chapter{A New Model}
Rather than using transfer functions with a linear PDE, we instead return to the generalized nonlinear Schr\"odinger equation~\cite{agrawal2013, ferreira, shtyrina, yarutkina} 
\begin{align}
\label{eq:nlse}
\pdiff{A}{z} &= - i \frac{\beta_2}{2}\pdiff[2]{A}{T} + \frac{\beta_3}{6}\pdiff[3]{A}{T} + i \gamma |A|^2 A + \frac{1}{2}g(A) A - \alpha A,
\end{align}
to represent the waveform. In this expression, $\beta_3$ is the third order dispersion coefficient, $\gamma$ is the coefficient of nonlinearity or self-phase modulation, $g(A)$ is the gain, and $\alpha$ is the loss of the fibre. Using this expression as a starting point, the laser cavity is assumed to be composed of five independent processes---gain, nonlinearity, loss, dispersion, and modulation. Within each component of the laser cavity the other four processes are assumed to be negligible, that is, each process is dominant only within one part of the laser, just as with the discrete model, but embracing any nonlinearities.

\section{Gain}
Considering the gain term as dominant as is expected with the Er-doped gain fibre, equation~\eqref{eq:nlse} reduces to
\begin{align}
\label{eq:gainde}
	\pdiff{A}{z} &= \frac{1}{2} \frac{g_0}{1 + E / \Es} A,& E &= \int_{-\infty}^\infty |A|^2 \, \df T,
\end{align}
where $g_0$ is a small signal gain, $E$ is the energy of the pulse and $\Es$ is the energy at which the gain begins to saturate~\cite{bohun, burgoyne2014, shtyrina, silfvast, yarutkina}.
%We shall first transform \eqref{eq:gainde} into an equation in terms of the energy. 
Multiplying~\eqref{eq:gainde} by $\bar{A}$, the complex conjugate of $A$, yields
\begin{align*}
	2\bar{A} \pdiff{A}{z} = \frac{g_0 |A|^2}{1 + E / \Es}.
\end{align*}
Adding this to its complex conjugate and integrating over $T$ gives
%\begin{align*}
%	\diff{|A|^2}{z} &= \frac{g_0 |A|^2}{1 + E / \Es}.
%\end{align*}
%After integrating this becomes
\begin{align}
\label{diffez}
	\diff{E}{z} &= \frac{g_0 E}{1 + E / \Es}.
\end{align}
For $E \ll \Es$ the energy grows exponentially, whereas for $E \gg \Es$ the gain has saturated and so the energy grows linearly. To obtain a closed form solution, \eqref{diffez} is integrated over a gain fibre of length $z$ and assume the energy increases from $E$ to $E_{\textrm{out}}$ so that
\[
	g_0 z = \log\frac{E_{\textrm{out}}}{E} + \frac{E_{\textrm{out}}-E}{\Es}
\]
and by exponentiating, rearranging, and applying $W_0$, the positive branch of the Lambert $W$ function,
\[
	 W_0\left(\frac{E}{\Es} \textrm{e}^{E/\Es} \textrm{e}^{g_0 z}\right) = 
	 W_0\left(\frac{E_{\textrm{out}}}{\Es} \textrm{e}^{E_{\textrm{out}}/\Es}\right) = \frac{E_{\textrm{out}}}{\Es}.
\]
This results in the closed form expression
\begin{align}
\label{eq:energy}
	E_{\textrm{out}}(z) = \Es W_0 \left( \frac{E}{\Es} \textrm{e}^{E/\Es} \textrm{e}^{g_0 z} \right)
\end{align}
with the desired property that $E_{\textrm{out}}(0)=E$. Since $E \sim |A|^2$, the gain in terms of the amplitude is given by
\begin{align}
	\label{eq:gain}
	G(A;E) &= \left(\frac{E_{\textrm{out}}(L_g)}{E}\right)^{1/2}A = \left( \frac{\Es}{E} W_0 \left( \frac{E}{\Es} \textrm{e}^{E/\Es} 
	\textrm{e}^{g_0 L_g} \right) \right)^{1/2} A,
\end{align}
where $L_g$ is the length of the gain fibre.

\section{Fibre Nonlinearity}
The nonlinearity of the fibre depends on the parameter $\gamma$. In regions where this affect is dominant expression~\eqref{eq:nlse} becomes
\begin{align}
\label{eq:fibrediff}
	\pdiff{A}{z} - i \gamma |A|^2 A = 0,
\end{align}
so that $\frac{\partial}{\partial z} |A|^2 = 0$ suggesting that $A(T,z) = A_0(T) e^{i \phi(T,z)}$. Substituting this representation into~\eqref{eq:fibrediff} and setting $\phi(T,0)=0$ gives $\phi(T,z) = \gamma |A|^2 z$. For a fibre of length $L_f$ the effect of the nonlinearity is therefore
\begin{align}
\label{eq:fibre}
	F(A) &= A e^{i \gamma |A|^2 L_f}.
\end{align}

\section{Loss}
Two sources of loss exist within the laser circuit: the loss due to the output coupler and the optical loss due to absorption and scattering. Combining these two effects give a loss that takes the form
\begin{align}
\label{eq:fibreloss}
	L(A) &= R e^{- \alpha L}A,
\end{align}
where $R$ is the reflectivity of the output coupler, and $L$ is the total length of the laser circuit.

\section{Dispersion}
Within the laser cavity, the dispersion is dominated by the chirped fibre Bragg grating (CFBG). In comparison, the dispersion due to the fibre is negligible\footnote{A $10$ cm chirped grating can provide as much dispersion as $300$ km of fibre~\cite{agrawal2002}.}. The dispersive terms of \eqref{eq:nlse} give
\begin{align}
\label{eq:disp}
	\pdiff{A}{z} = -i \frac{\beta_2}{2} \pdiff[2]{A}{T} + \frac{\beta_3}{6} \pdiff[3]{A}{T}
\end{align}
and since dispersion acts in the frequency domain, it is convenient to use the Fourier transform of~\eqref{eq:disp}, giving the result that
\begin{align*}
	\pdiff{}{z}\FT{A} &= i\frac{\omega^2}{2}\left(\beta_2 - \frac{\beta_3}{3} \omega\right) \FT{A}.
\end{align*}
The effect of dispersion is then
\begin{align}
\label{eq:dispersion}
	D(A) &= \FTi{\textrm{e}^{i \omega^2 L_D(\beta_2 - \beta_3 \omega/3)/2} \FT{A}}.
\end{align}
For a highly dispersive media the third order effects may need to be considered~\cite{agrawal2013, litchinitser}. However, for simplicity in the basic model and because of the nature of the grating, the third order effect will be neglected so we set $\beta_3=0$ for the subsequent analysis~\cite{agrawal2013, ferreira}.

\section{Modulation}
In the average model, the amount of modulation is characterized by the parameter $\epsilon$ through the term $\frac{\epsilon}{2}T^2 A$. In the new model, the modulation is considered to be applied externally through its action on the spectrum and for simplicity the representation is taken as the Gaussian
\begin{align}
	M(A) &= \textrm{e}^{-T^2 / 2 T_M^2} A,
\end{align}
where $T_M$ is a characteristic width of the modulation.%\footnote{Taking the Fourier transform, $\FT{\textrm{e}^{-T^2/ 2 T_M^2}} = \frac{1}{\sqrt{2\pi}}T_M\textrm{e}^{-\omega^2T_M^2/2}$.}

\section{Non-Dimensionalization}

\begin{table}
\begin{center}
\begin{tabular}{lclc}
Parameter & Symbol & Value & Source \\
\noalign{\global\arrayrulewidth=1.5pt}\hline
Saturation Energy & $\Es$ & $10^3$--$10^4 \text{ pJ}$ & \cite{burgoyneemail} \\
Fibre Nonlinearity & $\gamma$ & $0.001$--$0.01 \text{ W}^{-1} \text{m}^{-1}$ & \cite{agrawal2013} \\
Small Signal Gain & $g_0$ & $1$--$10 \text{ m}^{-1}$ & \cite{burgoyneemail} \\
Grating Dispersion & $\beta_2^g L_D$ & $10$--$2000 \text{ ps}^2$ & \cite{burgoyne2014, agrawal2013, litchinitser} \\
Fibre Dispersion & $\beta_2^f$ & $-50$--$50 \text{ ps}^2/ \text{km}$ & \cite{burgoyne2014, agrawal2013} \\
Modulation Time & $T_M$ & $15$--$150 \text{ ps}$ & \cite{burgoyne2014, burgoyneemail} \\
Length of Cavity & $L$ & $10$--$100 \text{ m}$ & \cite{burgoyneemail} \\
Length of Gain Fibre & $L_g$ & $1$--$4 \text{ m}$ & \cite{burgoyne2014, shtyrina, yarutkina} \\
Length of Fibre & $L_f$ & $0.15$--$1 \text{ m}$ & \cite{burgoyneemail} \\
\hline
\end{tabular}
\caption{Orders of magnitude of various parameters.}
\label{tab:values}
\end{center}
\end{table}

%\begin{figure}[tbp]
%\centering
%\begin{subfigure}[]{\columnwidth}
%\input{Shape}
%\caption{Envelope}
%\label{fig:envelope}
%\end{subfigure} \\
%\begin{subfigure}[]{\columnwidth}
%\input{Transform}
%\caption{Fourier Transform}
%\label{fig:fourier}
%\end{subfigure} \\
%\begin{subfigure}[]{\columnwidth}
%\input{Chirp}
%\caption{Chirp}
%\end{subfigure}
%\caption{Simulation with $s = 0.09$, $a = 8 \times 10^3$, $h = 0.04$, and $E_0 = 0.1$ with $A_0 = \Gamma \sech{(2T)} \textrm{e}^{i \pi / 4}$. $\Gamma$ is chosen such that $\int_{-\infty}^\infty |A|^2 \, \textrm{d}T = E_0$, and hyperbolic secant is chosen since it is a common soliton. In the stable case $b = 1.32$, whereas for the broken case $b = 1.25$. The pulses are shown after 25 iterations.}
%\label{fig:pulse}
%\end{figure}

The structure of each process of the laser can be better understood by re-scaling the time, energy, and amplitude. Specifically, the time shall be scaled by the characteristic modulation time which is proportional to the pulse duration, the energy by the saturation energy, and the amplitude will be scaled so that it is consistent:
\begin{align*}
	T &= T_M \widetilde{T},& E &= \Es \widetilde{E},& A &= \left( \frac{\Es}{T_M} \right)^{1/2} \widetilde{A}.
\end{align*}
The new process maps, after dropping the tildes, become
\begin{align*}
	G(A) &= \left(E^{-1} W_0 \left( a E \textrm{e}^{E}\right) \right)^{1/2} A,&
	F(A) &= A \textrm{e}^{i b |A|^2},&
	M(A) &= \textrm{e}^{-T^2 / 2} A, \\
	D(A) &= \FTi{\textrm{e}^{i s^2 \omega^2} \FT{A}},&
	L(A) &= h A,
\end{align*}
with four dimensionless parameter groups (see Table~\ref{tab:values})
\begin{align*}
	a &= \textrm{e}^{g_0 L_g} \sim 8 \times 10^3,& s &= \sqrt{\frac{\beta_2 L_D}{2 T_M^2}} \sim 0.2,&
	b &= \gamma L_f \frac{\Es}{T_M} \sim 1,& h &= R \textrm{e}^{-\alpha L} \sim 0.04,
\end{align*}
which control the behaviour of the laser.

\section{Combining the Effects}
In this model the pulse is iteratively passed through each process, the order of which is now important. In this first realization, the pulse is first amplified by the gain fibre, then since the pulse's magnitude is greatest the nonlinearity needs to be considered. The pulse is then tapped off by the output coupler, and then passes through the grating and is modulated. The pulse after one complete circuit of the laser cavity is then passed back in to restart the process. Functionally this can be denoted as
\begin{align*}
	\mathcal{L}(A) = M(D(L(F(G(A))))),
\end{align*}
where $\mathcal{L}$ is one loop of the laser. A solution to this model is one in which the envelope and chirp are unchanged after traversing every component in the cavity, that is, such that $\mathcal{L}(A) = A$---potentially with a constant phase shift.

% !TeX root = ../Thesis.tex

\chapter{Solution of the Nonlinear Model}
\label{chap:nl}
We shall now consider the nonlinear case---when $b > 0$. In this case it becomes much too difficult to obtain an analytic result. Recall from \eqref{eq:fibre} that the nonlinearity takes the form
\begin{align*}
F(A) &= A \textrm{e}^{i b |A|^2}.
\end{align*}
This is a highly nonlinear operator, and attempting to take Fourier transform of a pulse that's under gone this transformation quickly becomes futile. Instead we must resort to a numerical solution. \\

\section{Code}
Finding the solution numerically will be done in a similar manner as with the analytic linear solution. Using Python a function is written for each component of the laser cavity given by \eqref{eq:effects}, an initial pulse is iteratively passed from function to function in the hopes that a `fixed point' is found. The full Python code can be found in Appendix \ref{chap:code}. \\

\subsection{Validation}
Before delving into the nonlinear behaviour of the model, we wish to validate that the code is working as expected---and as a sanity test for the linear solution---by comparing the results of the simulations with $b=0$ to the results of the linear model in Chapter \ref{chap:linear}. In the case of the numerical solution our initial conditions become somewhat important. In all of the following analysis the initial waveform is $\Gamma \sech \left( 2T \right) \textrm{e}^{i \pi / 4}$ normalized so that the initial energy is $E_0 = 0.1$; additionally, $a = 8000$, and $h = 0.04$. \\
\begin{figure}[tbp]
\centering
\input{./Figures/Variance}
\caption{Simulation and analytic equilibrium variance, chirp, and phase shift as a function of $s$.}
\label{fig:var}
\end{figure}

In the first experiment the pulse is allowed to equilibrate for 40 loops of the circuit, we then compare the variance, chirp, and phase shift of the two methods. The results of this are shown in Figure \ref{fig:var}. For the most part, we see exceptional agreement between the analytic solution and numerical solution for the variance, and phase shift. However, for $s > 2.4$ the chirp from the simulations seems to erratically vary from the analytic solution. The reason for this is quite a simple one. \\
\begin{figure}[tbp]
\centering
\input{./Figures/LinearPlot}
\caption{Equilibrium energy, and peak power of the pulse as a function of $s$. The thin black line is to highlight that the energy is \emph{not} linearly related to $s$.}
\label{fig:valenergy}
\end{figure}

To investigate this behaviour, we instead turn our attention to the energy, and amplitude of the pulse at equilibrium as shown in Figure \ref{fig:valenergy}. As with Figure \ref{fig:var}, there is very good agreement between the two solutions. It is also now clear that at approximately $s = 2.5$ there is too much dispersion---the pulse is no longer sustainable. This is of course a consequence of the condition \eqref{eq:energycond}. Since the pulse effectively vanishes after this point, the chirp calculation becomes numerically unstable leading to the wild oscillations. \\

With the numerical solution yielding the expected results, we are now ready to dive into the rich structure the nonlinearity adds.

\section{Nonlinear Model}
\begin{figure}[tbp]
\centering
\begin{subfigure}{0.5\textwidth}
\centering
\input{./Figures/Stable_Shape}
\caption{Envelope}
\end{subfigure}%
\begin{subfigure}{0.5\textwidth}
\centering
\input{./Figures/Stable_FT}
\caption{Fourier transform}
\end{subfigure} \\
\begin{subfigure}{0.5\textwidth}
\centering
\input{./Figures/Stable_Chirp}
\caption{Chirp}
\end{subfigure}
\caption{Simulation with $s = 0.15$, and $b = 2.1$ after 15 circuits.}
\label{fig:nlstable}
\end{figure}
With the inclusion of the nonlinearity we generally find a similar solution to the linear case. An example of this is shown in Figure \ref{fig:nlstable}. The envelope of the pulse is unsurprisingly Gaussian-esque, however, it is \emph{not} precisely Gaussian and more closely resembles a generalized Gaussian\footnote{A generalized Gaussian has the form $\exp \left(-t^\alpha \right)$, with $\alpha > 2$.}. The fact that the pulse envelope is not Gaussian is further emphasized while examining the Fourier transform of the pulse. If the pulse were a Gaussian, we would expect the Fourier transform to also be a Gaussian \cite{debnath, gradshteyn}. Instead, the magnitude of the Fourier transform has a unique Batman-like shape. This deviation suggests the nonlinearity implants higher frequency oscillations into the pulse---this will be a key observation in the coming subsections. Finally, we shall examine the derivative of the phase---essentially the chirp. Recall in Chapter \ref{chap:linear} the chirp was defined as coefficient of $-\frac{1}{2}iT^2$ in the exponential, by taking the negative of the derivative we would expect a linear function with a slope equal to the chirp. In the nonlinear case, this is what we find for moderate values of $T$. On the other hand, for $|T| > 1$ the relation begins to level off. This is in agreement with experimental results \cite{chen, rothenberg, tomlinson}. \\

We shall now take a look at perhaps the most important feature of the pulse---the energy. \\

\subsection{Energy}
\label{chap:energy}

\begin{figure}[p]
\centering
\begin{subfigure}{\textwidth}
\input{./Figures/StabilityZoom}
\caption{Centralized for more typical $s$, and $b$ values.}
\label{fig:energyzoom}
\end{subfigure} \\
\begin{subfigure}{\textwidth}
\input{./Figures/Stability}
\caption{Energy of the pulse within the $s$-$b$ plane.}
\label{fig:energybig}
\end{subfigure}
\caption{Energy of the pulse at equilibrium. The energy is constant along the black lines.}
\label{fig:energy}
\end{figure}

The energy of the pulse is directly related to the output power of the laser, and since this is not as controllable as in a regular laser, it is of great interest. The energy of the pulse at equilibrium\footnote{The pulse is passed through each component of the laser 40 times before the energy is computed.} is shown in Figure \ref{fig:energy}. In Figure \ref{fig:energyzoom} we generally find that the energy is a smooth surface that slowly decays as $s$, and $b$ increase. In the upper left hand region of the plot this is clearly not the case. The contours show that the energy is very noisy and discontinuous. Perhaps more surprising, as shown in Figure \ref{fig:energybig}, is that this boundary appears to become periodic as $b$ increases around $s = 0.3$. \\

% I couldn't get this to work without having blank space on like 4 pages

%\begin{wrapfigure}{O}{0.5\textwidth}
%%\vspace{-10mm}
%\centering
%\input{./Figures/EffNL}
%\caption{Approximate effective nonlinearity.}
%\label{fig:effnl}
%\end{wrapfigure}
\begin{figure}[tbp]
\centering
\input{./Figures/EffNL}
\caption{Approximate effective nonlinearity.}
\label{fig:effnl}
\end{figure}
This periodic strip can be explained from the functional nature of the nonlinearity. Recall once again that from \eqref{eq:fibre} that the nonlinearity takes the form
\begin{align*}
F(A) &= A \textrm{e}^{i b |A|^2}.
\end{align*}
The strength of the nonlinearity is thus controlled by $b |A|^2$, where $|A|^2$ is the power of the pulse. Although not the greatest, we can approximate the power of the pulse by the energy so that $b |A|^2 \sim b E$, this gives an estimate for what shall be referred to as the effective nonlinearity. Figure \ref{fig:effnl} shows the effective nonlinearity at equilibrium. Notice that as $b$ increases along the strip $0.2 < s < 0.4$, the contours become steeper and steeper. In addition, the contours are at multiples of $\pi$, and exit the boundary at approximately the same places. Thus, the reason for this periodicity is that the strength of the nonlinearity is proportional to the modulus of the effective nonlinearity---the effective nonlinearity passes through the same points repeatedly. \\

\begin{figure}[p]
\centering
\begin{subfigure}{\textwidth}
\centering
\input{./Figures/Unstable_Shape}%
\input{./Figures/Unstable_Bad_Shape}
\caption{Envelope}
\end{subfigure} \\
\begin{subfigure}{\textwidth}
\centering
\input{./Figures/Unstable_FT}%
\input{./Figures/Unstable_Bad_FT}
\caption{Fourier transform}
\end{subfigure} \\
\begin{subfigure}{\textwidth}
\centering
\input{./Figures/Unstable_Chirp}%
\input{./Figures/Unstable_Bad_Chirp}
\caption{Chirp}
\end{subfigure}
\caption{$s = 0.15$, $b = 2.15$, \textbf{Left:} 11 loops, \textbf{Right:} 16 loops}
\label{fig:break}
\end{figure}

\subsubsection{Self-Phase Modulation and Wave Breaking}

The noise in the energy exhibited for moderate to large values of $b$, and small values of $s$ is a phenomenon called \emph{wave breaking} \cite{agrawal2013, anderson, finot, rothenberg, tomlinson}. Wave breaking is not limited to just optics, wave breaking occurs in areas such as plasmas, transmission lines, and fluid dynamics \cite{rothenberg}. Wave breaking occurs because the pulse begins to interfere with itself in a way called self-phase modulation \cite{agrawal2002, agrawal2013, becker}. Self-phase modulation occurs because the index of refraction is intensity dependent \cite{agrawal2002, becker, rothenberg, silfvast}, which leads to additional chirp across the pulse \cite{agrawal2013, anderson, rothenberg, silfvast}. This in turn causes higher order frequencies to be injected into the pulse \cite{agrawal2013, anderson}. These high frequencies compound with each trip around the cavity becoming parasitic very quickly---Figure \ref{fig:break} highlights this. Notice that the difference between this figure and Figure \ref{fig:nlstable} is a difference in $b$ of $0.05$---this difference could be as small as adding a few centimetres more of fibre between the gain and output coupler. The left figures show the pulse after 11 trips around the cavity, in the Fourier transform it is clear that the contributions from higher frequencies has increased---we obtain similar results as in \cite{anderson, rothenberg}. Additionally, the chirp starts losing its linearity causing it to start becoming unstable; the nature of this instability is in agreement with \cite{anderson, rothenberg}. The parasitic nature of the high frequency contributions is evident by examining the right figures. After 5 additional trips around the cavity, the envelope of the pulse is much more rippled, and the real and imaginary parts become incoherent. Moreover, the Fourier transform has no clear defined shape and has essentially become noise, and the chirp has grown to be highly oscillatory and unstable. Once the pulse has reached a state such as this, there is no turning back---the envelope, Fourier transform, and chirp never reach a steady equilibrium state. \\

\subsection{Convergence}
\begin{figure}[tbp]
\input{./Figures/Step1}
\caption{Error of the pulse (given by \eqref{eq:error}) between iterations $99$ and $100$.}
\label{fig:error}
\end{figure}

\begin{figure}[p]
\begin{subfigure}{0.5\textwidth}
\input{./Figures/Step2}
\caption{$\Delta = 2$}
\end{subfigure}
\begin{subfigure}{0.5\textwidth}
\input{./Figures/Step3}
\caption{$\Delta = 3$}
\end{subfigure} \\
\begin{subfigure}{0.5\textwidth}
\input{./Figures/Step5}
\caption{$\Delta = 5$}
\end{subfigure}
\begin{subfigure}{0.5\textwidth}
\input{./Figures/Step7}
\caption{$\Delta = 7$}
\end{subfigure} \\
\begin{subfigure}{0.5\textwidth}
\input{./Figures/Step11}
\caption{$\Delta = 11$}
\end{subfigure}
\begin{subfigure}{0.5\textwidth}
\input{./Figures/Step13}
\caption{$\Delta = 13$}
\end{subfigure} \\
\begin{subfigure}{0.5\textwidth}
\input{./Figures/Step16}
\caption{$\Delta = 16$}
\end{subfigure}
\begin{subfigure}{0.5\textwidth}
\input{./Figures/Step60}
\caption{$\Delta = 60$}
\end{subfigure} \\
\caption{Error for various values of $\Delta$, the colour bar and axis labels have been omitted for clarity, the scales and ranges are the same as in Figure \ref{fig:error}.}
\label{fig:deltaerror}
\end{figure}

To obtain a better understanding of how the pulse either converges to equilibrium, or diverges to wave breaking, we shall examine the difference between the envelopes of consecutive iterations. More precisely, we compute the error by
\begin{align}
\textrm{E} = \frac{\| |A_i| - |A_{i-1}| \|_2}{\| A_{i-1} \|_2},
\label{eq:error}
\end{align}
where $\| \cdot \|_2$ denotes the $L^2$ norm, which is computed numerically using trapezoid rule. Notice as well that in the numerator we use the modulus of the pulses, again this is because we are uninterested about the phase shift between iterations. A plot of the error can be found in Figure \ref{fig:error}, with $i = 100$. \\

Unsurprisingly, the error is largest in the region where the wave breaks. As mentioned in the previous subsection, the pulse does not reach a stable state. As a consequence the envelope varies violently, which leads to this large error. On the other hand, the region where the energy appeared to be stable does not have a non-zero error---there are a few reasons for this. First, because of the numerical calculation, for all intents and purposes an error less than $10^{-10}$ can be considered $0$. Understanding the other sources of these small errors requires a deeper examination. \\

The second cause of these errors is because some equilibrium states have a periodicity greater than $1$, we redefine \eqref{eq:error} to be instead
\begin{align}
\textrm{E}_\Delta = \frac{\| |A_i| - |A_{i-\Delta}| \|_2}{\| A_{i-\Delta} \|_2},
\label{eq:deltaerror}
\end{align}
to pick up equilibrium states with period $\Delta$, again with $i = 100$. Figure \ref{fig:deltaerror} shows the error for an assortment of $\Delta$ values. There are several key observations to make, first, for $\Delta = 2$ the kidney-shaped blob in the lower right corner has vanished, thus, this entire region has a periodicity of $2$. Second, for $\Delta = 3$ in the upper right corner a negative triangular-shaped structure emerges. Lastly, $\Delta \in \left\{ 5, 7, 11, 13 \right\}$ may not seem to extract any higher order periods, however, there are in fact very small striations in the lower left. Up until now the $\Delta$ values have been prime since they are the building blocks for composite numbers. However, there are two composite $\Delta$ values of interest---the first being $16$. Since $16 = 2^4$ it will pick up any periodic behaviour with orders of $2, 4, 8, 16$, which are frequently found in bifurcations. Lastly, we chose $\Delta = 60$ for a similar reason: because $60$ is highly divisible---it is an abundant number. This extracts orders such as $12$ or $30$ without having to explicitly run the calculations\footnote{In the case of $\Delta = 60$ a value of $i = 150$ is used to ensure sufficient iterations to converge.}. \\

\begin{figure}[tbp]
\input{./Figures/Min}
\caption{Composite error, \eqref{eq:errorcomp}, of Figure \ref{fig:deltaerror}.}
\label{fig:errorcomp}
\end{figure}

We now have the data to make a more meaningful estimate of the error between iterations. To compute this composite error we take the minimum of the error from each of these calculations:
\begin{align}
\textrm{E}_c = \min_\Delta \left\{ \textrm{E}_\Delta \right\}.
\label{eq:errorcomp}
\end{align}
The intriguing structure of the composite error is shown in Figure \ref{fig:errorcomp}. Compared with the error from Figure \ref{fig:error} the lower right region is much more well behaved, that is, the error is a few orders of magnitude lower. A peculiar trait of the band between the unstable and stable regions is that within this area the pulse is in a quasi-stable state---the pulse is stable and has reach a sort of equilibrium, however, the envelope of the pulse has minute variations with no clear period. Additionally, there appears to be no periodicity of the envelope within the wave breaking region, further supporting the claim that the pulse here is completely unstable. \\

\begin{figure}[p]
\centering
\begin{subfigure}{\textwidth}
\input{./Figures/StabilitySwitchZoom}
\caption{Centralized for more typical $s$, and $b$ values.}
\label{fig:switchzoom}
\end{subfigure} \\
\begin{subfigure}{\textwidth}
\input{./Figures/StabilitySwitch}
\caption{Energy of the pulse within the $s$-$b$ plane.}
\label{fig:switchbig}
\end{subfigure}
\caption{Energy of the pulse at equilibrium, with the modulation and dispersion blocks switched.}
\label{fig:switch}
\end{figure}

\subsection{Permutation of Components}
The last item we wish to consider is the order in which the components are placed. In Section \ref{sec:effects} a brief description for the choice of the order was given. We start with the loss component since this coincides with the output; the fibre nonlinearity follows the gain since this is where it has the largest impact; and the loss follows the nonlinearity in an attempt to mitigate its effect. Therefore, the loss is first, and the gain followed by the nonlinearity are last---leaving dispersion and modulation in the middle. We chose to put the dispersion block ahead of the modulator. However, there was no real reason behind this---modulation before dispersion is equally as valid---and in this subsection we explore the effect of modulating the pulse before it passes through the CFBG. \\

The result of this switch is shown in Figure \ref{fig:switch}. As a whole, unsurprisingly, we find the same behaviour and structure, however, there are some intriguing differences. Perhaps the most interesting is the small island of instability in Figure \ref{fig:switchzoom} at around $s = 0.075$, $b = 1.1$ which was not present in the other ordering. Within this island the wave is unstable and breaks, but, curiously there is a small gap between this island and the main unstable region. We find more distinctions when considering the larger area in Figure \ref{fig:switchbig}. Again, the structure and periodic nature of the boundary is similar to before, however, this boundary has shifted rightwards to a larger $s$ value. Additionally, within the unstable region the density of the contour lines is much greater---suggesting it is in some sense more chaotic and random than with the components in their original permutation. The final main difference between the two orderings, is that in this case, the energy contours are no longer monotonic functions of $s$. Instead we find a parabolic shape on the top contour, and two lobes on the second contour.






% !TeX root = ../Thesis.tex

\chapter{Conclusion}

\phantomsection 
\addcontentsline{toc}{chapter}{References} 
\bibliography{Ref}

\appendix

% !TeX root = ../Thesis.tex

\chapter{The Lambert $W$ Function}
\label{chap:lambertw}
The Lambert $W$ function is defined to be the inverse of the function $f(x) = x \textrm{e}^x$ and its graph is shown in Figure \ref{fig:lambertw}. In other words, if $z = x \textrm{e}^x$ then $x = W(z)$. Notice that by combining these relations we obtain the identities
\begin{align}
\label{eq:lambertw}
z &= W(z) \textrm{e}^{W(z)}, & x &= W(x \textrm{e}^x).
\end{align}
This function is called the Lambert $W$ function because it is the logarithm of a special instance of Lambert's series---the letter $W$ is used because of the work done by E. M. Wright \cite{lambertw}.

Notice that the original function, $f(x) = x \textrm{e}^x$, is \emph{not} injective, and as a consequence, the $W$ function is multi-valued on the interval $[-1/\textrm{e},0)$. To alleviate this, sometimes the branch $W(x) \geq -1$ is denoted $W_0$ and is called the upper or principal branch, whereas the branch $W(x) < -1$ is denoted $W_{-1}$ and is called the lower branch. However, in this work the $W$ function will only take positive real values and so this distinction is not needed.

\begin{figure}[htbp]
\centering
\input{./Figures/LambertPlot}
\caption{The two branches of the Lambert $W$ function.}
\label{fig:lambertw}
\end{figure}

The Lambert $W$ function has applications in various areas of math and physics \cite{lambertw} including:
\begin{itemize}
\item Jet fuel problems
\item Combustion problems
\item Enzyme kinetics problems
\item Linear constant coefficient differential delay equations
\item Volterra equations.
\end{itemize}
Primarily, the $W$ function is used when solving iterated exponentiation, and solving certain algebraic equations. For example, consider the equation $z = x^x$. By taking the logarithm of each side we have
\begin{align*}
\log z &= x \log x, \\
&= \log x \textrm{e}^{\log x},
\end{align*}
which after applying the $W$ function reduces to $W(\log z) = \log x$ by \eqref{eq:lambertw}. Finally, the answer can be written as $x = \exp(W(\log z))$.


%% !TeX root = ../Thesis.tex

\chapter{Spread Due to Dispersion}

\cite{integrals}

%% !TeX root = ../Thesis.tex

\chapter{Asymptotic Expansions of the Variance}
\label{chap:asymp}
We shall investigate the nature of the solution to \eqref{eq:var} both when $s \rightarrow 0$, as well as when $s \rightarrow \infty$. For ease of notation, \eqref{eq:var} is rewritten as
\begin{align*}
\frac{1}{4} x^4 = \eps \left( -x^3 + 5x^2 - 8x + 4 \right),
\end{align*}
where $x = \sigma^2$, $\eps = s^4$, and $\eps \rightarrow 0$. Additionally, suppose $x$ can be expanded as a power series in $\eps$:
\begin{align*}
x = x_0 + x_1 \eps^\alpha + x_2 \eps^\beta + \dots
\end{align*}
with $0 < \alpha < \beta$. Then,
\begin{align*}
\bigO1 : \frac{1}{4} x_0^4 = 0,
\end{align*}
and so, $x_0 = 0$. Knowing this, up to the next order we have
\begin{align*}
\frac{1}{4} x_1^4 \eps^{4\alpha} = \eps \left( -x_1^3 \eps^{3\alpha} + 5x_1^2 \eps^{2\alpha} - 8x_1 \eps^\alpha + 4 \right)
\end{align*}
in order to have dominant balance it must be that the left hand side balances with the final term of the right hand side, that is, $4\alpha = 1$, or $\alpha = \frac{1}{4}$. Therefore,
\begin{align*}
\bigO\eps : \frac{1}{4} x_1^4 &= 4, \\
(x_1 - 2)(x_1 + 2)(x_1^2 + 4) &= 0,
\end{align*}
and so, $x_1 = \pm 2, \, \pm 2 i$. However, we wish $x$ to be positive and real---recall that $x = \sigma^2$---thus, we take $x_1 = 2$. Now, the next lowest order of the left hand side must be $\eps^{3\alpha + \beta} = \eps^{3/4 + \beta}$. As with the previous iteration, this must balance with the lowest order term of the $8x$. Hence, $\frac{3}{4} + \beta = 1 + \alpha = \frac{5}{4}$, thus, $\beta = \frac{1}{2}$. Now,
\begin{align*}
\bigO{\eps^{5/4}}: \frac{1}{4} 4 x_1^3 x_2 &= -8x_1, \\
x_2 &= -2,
\end{align*}
where the additional $4$ comes from the binomial expansion. Finally, up to second(?***) order
\begin{align*}
x &\approx 2 \eps^{1/4} - 2 \eps^{1/2}, \\
\sigma^2 &\approx 2 \left( s - s^2 \right),
\end{align*}
this is shown in Figure \ref{fig:lims0}.
\begin{figure}[tbp]
\input{./Figures/Lim_s_0}
\caption{Asymptotic expansion of the variance as $s \rightarrow 0$.}
\label{fig:lims0}
\end{figure}

We shall now consider the other limit, when $s \rightarrow \infty$. Using a similar substitution we instead write \eqref{eq:var} as
\begin{align*}
\frac{1}{4} \eps x^4 = -x^3 + 5x^2 - 8x + 4,
\end{align*}
where $x = \sigma^2$, and $\eps = s^{-4}$ instead so that we still have $\eps \rightarrow 0$. Furthermore, we must make a correction to the series expansion:
\begin{align*}
x = \eps^{-\xi} \left( x_0 + x_1 \eps^\alpha + x_2 \eps^\beta + \dots \right).
\end{align*}
The reason for this is because the equation is now singular---when $\eps=0$ the equation transforms from a quartic to a cubic, losing a root. As with before, we obtain dominant balance when the quartic term balances with the cubic term:
\begin{align*}
1 - 4 \xi &= -3 \xi, \\
\xi &= 1.
\end{align*}
Then,
\begin{align*}
\bigO{\eps^{-3}}: \frac{1}{4} x_0^4 &= -x_0^3, \\
x_0^3 (x_0 + 4) &= 0,
\end{align*}
we shall choose $x_0 = 0$---since $x \geq 0$. Unlike up until this point, we obtain dominant balance when the right hand side dominates the left hand side, this is achieved when $\alpha = 1$. Now,
\begin{align*}
\bigO1 : 0 &= -x_1^3 + 5x_1^2 - 8x_1 + 4, \\
&= -(x_1 - 1)(x_1 - 2)^2,
\end{align*}
so either $x_1 = 1$ or $x_1 = 2$, we shall see in the next step that we can eliminate one of these. The next term obtains dominant balance when the left hand side balances with the linear term of the right hand side, that is, when $\beta = 2$:
\begin{align*}
\bigO\eps : \frac{1}{4} x_1^4 &= -3x_1^2 x_2 + 5\cdot2 x_1 x_2 - 8x_2, \\
&= -x_2 (3x_1 - 4)(x_1 - 2),
\end{align*}
where again, the coefficients come from the binomial expansions;
from this it is clear that
\begin{align*}
x_2 &= \frac{-x_1^4}{4(3x_1 - 4)(x_1 - 2)},
\end{align*}
and that $x_1 \ne 2$. Thus, $x_1 = 1$, $x_2 = -\frac{1}{4}$, and
\begin{align*}
x = 1 - \frac{1}{4} \eps + x_2 \eps^{\gamma - 1} + \dots.
\end{align*}
Expanding one final term we find $\gamma = 3$, and
\begin{align*}
\bigO{\eps^2}: \frac{1}{4} 4x_2 x_1^3 &= -(3x_3 x_1^2 + 3x_2^2 x_1) + 5(2x_3 x_1 + x_2^2) - 8(x_3).
\end{align*}
Knowing the values of $x_1$, and $x_2$ this is simply an arithmetical calculation yielding $x_3 = \frac{3}{8}$. Finally,
\begin{align*}
x &\approx 1 - \frac{1}{4}\eps + \frac{3}{8}\eps^2, \\
\sigma^2 &\approx 1 - \frac{1}{4s^4} + \frac{3}{8s^8},
\end{align*}
this approximation is shown in Figure \ref{fig:limsinfty}.
\begin{figure}[tbp]
\input{./Figures/Lim_s_Infty}
\caption{Asymptotic expansion of the variance for $s \rightarrow \infty$.}
\label{fig:limsinfty}
\end{figure}










% !TeX root = ../Thesis.tex
%\chapter{Span of Gaussians in $L^2$}
%\label{chap:gauss}
\section{Span of Gaussians in $L^2$}
In order to show that Gaussians span $L^2$, we shall start our analysis with the span of the Hermite polynomials in $L^2$. Typically the Hermite polynomials are recursively defined as \cite{conway, courant, teuwen}
\begin{align*}
H_n(x) := (-1)^n \textrm{e}^{x^2} \diff[n]{}{x} \textrm{e}^{-x^2},
\end{align*}
with the inner product\footnote{The complex conjugate of $g$ is omitted since the functions dealt with are real.}
\begin{align*}
\left< f, g \right> = \int_\mathbb{R} f(x) g(x) \textrm{e}^{-x^2} \df x,
\end{align*}
where $\textrm{e}^{-x^2}$ is the weighting function. This is so that 
\begin{align*}
\left< H_m, H_n \right> = \sqrt{\pi} 2^n n! \delta_{mn}
\end{align*}
and, therefore, the Hermite polynomials form an orthogonal set \cite{courant, hochstrasser, kreyszig, szego, teuwen}. Consider instead, the Gaussian--Hermite polynomials
\begin{align*}
\Her{n}{x} &:= \textrm{e}^{-x^2/2} H_n(x),
\end{align*}
with the inner product
\begin{align*}
\left< f, g \right> = \int_\mathbb{R} f(x) g(x) \, \df x,
\end{align*}
notice that the weighting function has been absorbed into the Hermite polynomials. We shall now show that the set of Gaussian--Hermite polynomials spans $L^2$. \\

\begin{theorem}
\label{thm:hermite}
The set of Gaussian--Hermite polynomials, $\mathcal{G}(\mathbb{R}) = \{ \Her{n}{x} : x \in \mathbb{R}, n \in \mathbb{N} \}$, spans $L^2(\mathbb{R})$.
\end{theorem}
\begin{proof}
The Gaussian--Hermite polynomials are an orthogonalization of the set $\Xi = \{x^n \textrm{e}^{-x^2/2} : x \in \mathbb{R}, n \in \mathbb{N}\}$, and therefore, have the same span \cite{kreyszig, szego}. Thus, it is sufficient to show that $\Xi$ spans $L^2$.
Suppose there exists an $f \in L^2$ such that $\left< f, \widetilde{H}_n \right> = 0$ for all $n$, or equivalently, that $\left< f, g_n \right> = 0$ for $g_n(x) = x^n \textrm{e}^{-x^2/2}$ for all $n$ so that it is not in the span of $\mathcal{G}(\mathbb{R})$. \\

Let us now consider the function \cite{courant, szego, teuwen}
\begin{align*}
F(z) = \frac{1}{\sqrt{2\pi}}\int_\mathbb{R} f(x) \textrm{e}^{zx} \textrm{e}^{-x^2/2} \, \df x,
\end{align*}
noting that $f(x) \textrm{e}^{zx} \textrm{e}^{-x^2/2} \in L^2(\mathbb{R})$ by the Cauchy--Schwarz inequality \cite{griffel, kolmogorov}, so $F$ converges, and that $F$ is holomorphic. The first exponential can be expanded into its Maclaurin series, then with Fubini's theorem we obtain
\begin{align*}
F(z) = \frac{1}{\sqrt{2\pi}} \sum_{n = 0}^{\infty} \frac{z^n}{n!} \int_\mathbb{R} f(x) x^n \textrm{e}^{-x^2/2} \, \df x.
\end{align*}
Recall, that by assumption
\begin{align*}
\int_\mathbb{R} f(x) x^n \textrm{e}^{-x^2/2} \, \df x &= \left< f, g_n \right> = 0,
\end{align*}
for all $n$ and so $F(z) \equiv 0$. We now notice that
\begin{align*}
0 &= F(z) = F(i \omega) = \FT{f \textrm{e}^{-x^2/2}},
\end{align*}
and now it is clear that $f(x) \textrm{e}^{-x^2/2} = 0$ almost everywhere, therefore, $f(x) = 0$ almost everywhere. \\
\end{proof}

Alternatively, without much difficulty it can be shown that the Gaussian--Hermite polynomials satisfy
\begin{align}
\label{eq:paracyl}
\diff[2]{\Her{n}{x}}{x} - \left( x^2 - 2n - 1 \right) \Her{n}{x} = 0.
\end{align}
This has the form of the parabolic cylinder functions \cite{miller}, and indeed the Gaussian--Hermite polynomials can be expressed in terms of parabolic cylinder functions \cite{hochstrasser, miller}. Moreover, \ref{eq:paracyl} can also be written in Sturm--Liouville form as
\begin{align*}
\diff{}{x} \left( 1 \cdot \diff{\Her{n}{x}}{x} \right) + (1 - x^2) \Her{n}{x} = -2n \Her{n}{x}.
\end{align*}
By the spectral theorem \cite{griffel, higson, kreyszig}, this suggests that the Gaussian--Hermite polynomials form an orthogonal set with the weighting function $w(x) = 1$, and are complete in $L^2(\mathbb{R})$. Using this result, we can now show that Gaussian functions also span $L^2(\mathbb{R})$ with the following theorem. \\

\begin{figure}[tbp]
\begin{subfigure}{0.5\textwidth}
\input{./Figures/GS1}
\caption{$N = 20$, $h = 0.1$}
\end{subfigure}
\begin{subfigure}{0.5\textwidth}
\input{./Figures/GS2}
\caption{$N = 10$, $h = 0.1$}
\end{subfigure}
\caption{Two examples of Gaussian series.}
\label{fig:gs}
\end{figure}

\begin{theorem}
\label{thm:gaussian}
Gaussians of a single variance span $L^2(\mathbb{R})$.
\end{theorem}
\begin{proof}
Theorem \ref{thm:hermite} showed that any square integrable function can be expressed as
\begin{align*}
f(x) = \sum_{n = 0}^\infty a_n \Her{n}{x}.
\end{align*}
Using a similar idea to \cite{calcaterra2, calcaterra}, we shall now expand the Gaussian--Hermite polynomials using their definition
\begin{align*}
f(x) = \sum_{n = 0}^\infty a_n (-1)^n \textrm{e}^{x^2/2} \diff[n]{}{x} \textrm{e}^{-x^2}.
\end{align*}
Finally, we can rewrite the derivatives using central differences \cite{zwillinger} so that
\begin{align}
f(x) = \sum_{n = 0}^\infty a_n (-1)^n \textrm{e}^{x^2/2} \left[ \lim_{h \rightarrow 0} \frac{1}{h^n} \sum_{i = 0}^n (-1)^i \binom{n}{i} \exp\left( -\left( x + h\left( \frac{n}{2} - i \right) \right)^2 \right) \right].
\label{eq:gs}
\end{align}
\end{proof}

\begin{figure}[tbp]
\input{./Figures/GaussMod}
\caption[Gaussian series for a particular modulation function.]{$N = 20$, $h = 0.1$}
\label{fig:gaussmod}
\end{figure}

Two examples of Gaussian series from \eqref{eq:gs} are shown in Figure \ref{fig:gs}. These examples are taken from \cite{calcaterra2, calcaterra}, however, we achieve similar or better approximations with either fewer terms, or with an $h$ value an order of magnitude larger. There are two reasons for this, first, we use central difference as opposed to backwards difference yielding a convergence of $\bigO{h^2}$ instead of $\bigO{h}$. Furthermore, because \cite{calcaterra2, calcaterra} used backwards differences, the means of the Gaussians are all non-negative, whereas with central differences, our Gaussians' means are both positive and negative---this leads to smaller coefficients, and better numerical stability. Moreover, Figure \ref{fig:gaussmod} shows the Gaussian series for the modulation function in \cite{bohun, burgoyneemail}.



\chapter{Code}
\label{chap:code}
\lstinputlisting[language=Python]{../Lasers.py}

% Cite collections so they show up in the references
\nocite{abramowitz, gowers, hernandez}

\end{document}

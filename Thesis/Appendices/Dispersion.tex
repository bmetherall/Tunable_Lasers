% !TeX root = ../Thesis.tex
%\chapter{Spread Due to Dispersion}
%\label{chap:disp}
The effect of dispersion can be computed analytically for input pulses using \eqref{eq:disp}. In the case of a Gaussian pulse, we make use of the transforms \cite{debnath, gradshteyn}
\begin{align*}
\FT{\textrm{e}^{-\eta T^2}} &= (2 \eta)^{-1/2}\textrm{e}^{-\omega^2/4\eta},& \FTi{\textrm{e}^{-\eta \omega^2}} &= (2 \eta)^{-1/2}\textrm{e}^{-T^2/4\eta}.
\end{align*}
From \eqref{eq:disp} we have that
\begin{align*}
D \left( \textrm{e}^{-\eta T^2} \right) &=\FTi{\textrm{e}^{i s^2 \omega^2} \FT{\textrm{e}^{-\eta T^2}}}, \\
&= (2 \eta)^{-1/2} \FTi{\exp \left( -\omega^2 \left( \frac{1}{4\eta} - i s^2 \right) \right)}, \\
&= (1 - 4 i \eta s^2)^{-1/2} \exp \left( -T^2 \frac{a}{1 - 4 i s^2 \eta} \right).
\end{align*}
For us, $\displaystyle \eta = \frac{1}{2} \frac{1 + i C}{\sigma^2}$; making this substitution yields
\begin{align*}
D(A_1) &= \left( 1 + \frac{2C s^2}{\sigma^2} - \frac{2s^2}{\sigma^2}i \right)^{-1/2} \exp \left( -T^2 \frac{1 + i C}{2\sigma^2 - 4 i s^2 (1 + i C)} \right).
\end{align*}
This can be greatly simplified by first rationalizing the denominators to give
\begin{align*}
D(A_1) &= \left( \frac{1 + \frac{2C s^2}{\sigma^2} + \frac{2s^2}{\sigma^2}i}{\left( 1 + \frac{2C s^2}{\sigma^2} \right)^2 + \left( \frac{2s^2}{\sigma^2} \right)^2} \right)^{1/2} \exp \left( -T^2 \frac{(1 + i C)(2\sigma^2 + 4 C s^2 + 4 i s^2)}{(2\sigma^2 + 4C s^2)^2 + 16s^4} \right),
\end{align*}
and then by writing in polar coordinates:
\begin{multline*}
D(A_1) = \left( \left( 1 + \frac{2C s^2}{\sigma^2} \right)^2 + \left( \frac{2s^2}{\sigma^2} \right)^2 \right)^{-1/4} \exp \left( \frac{1}{2} i \arctan \left( \frac{\frac{2s^2}{\sigma^2}}{1 + \frac{2C s^2}{\sigma^2}} \right) \right) \\
\times \exp \left( -T^2 \frac{\sigma^2 \left[ 1 + i \left( C + (1 + C^2) \frac{2s^2}{\sigma^2} \right) \right]}{2 \left[ (\sigma^2 + 2C s^2)^2 + 4s^4 \right]} \right).
\end{multline*}
Finally, this can be simplified further to
\begin{multline*}
D(A_1) = \sigma \left( \left( \sigma^2 + 2C s^2 \right)^2 + 4s^4 \right)^{-1/4} \exp \left( \frac{1}{2} i \arctan \left( \frac{2s^2}{\sigma^2 + 2C s^2} \right) \right) \\
\times \exp \left( -T^2 \frac{\sigma^2 \left[ 1 + i \left( C + (1 + C^2) \frac{2s^2}{\sigma^2} \right) \right]}{2 \left[ (\sigma^2 + 2C s^2)^2 + 4s^4 \right]} \right).
\end{multline*}
From this expression it is clear that at equilibrium
\begin{subequations}
\begin{align}
\label{eq:varout}
\widetilde{\sigma}^2 \sigma^2 &= \left( \sigma^2 + 2 C s^2 \right)^2 + 4s^4, \\
\label{eq:chirpout}
\widetilde{C} &= C + (1+C^2) \frac{2 s^2}{\sigma^2}, \\
\label{eq:phase}
\phi &= \frac{1}{2} \arctan \left( \frac{2s^2}{\sigma^2 + 2C s^2} \right), \\
\zeta &= \left( \frac{\sigma}{\widetilde{\sigma}} \right)^{1/2}.
\end{align}
\end{subequations}

The expressions for the chirp and variance can be verified with the well known relations \cite{agrawal2013, anderson, ferreira, silfvast}
\begin{align*}
\left( \frac{T_1}{T_0} \right)^2 &= \left( 1 + \frac{C \beta_2 z}{T_0^2} \right)^2 + \left( \frac{ \beta_2 z}{T_0^2} \right)^2,&
\widetilde{C} &= C + (1+C^2) \frac{\beta_2 z}{T_0^2}.
\end{align*}
Where, within our non-dimensionalization, $T_0 = \sigma T_M$, $T_1 = \widetilde{\sigma} T_M$, $z = L_D$, and $\beta_2 z T_0^{-2} = 2s^2\sigma^{-2}$. In addition to this, the expression for $\zeta$ can be validated using conservation of energy. Within the CFBG energy is conserved, therefore,
\begin{align*}
\int_{-\infty}^\infty |A_1|^2 \, \df T &= \int_{-\infty}^\infty |A_2|^2 \, \df T.
\end{align*}
Which, after substituting the expressions from Section \ref{sec:linear}, reduces to
\begin{align*}
h^2 P \int_{-\infty}^\infty \textrm{e}^{-T^2/\sigma^2} \, \df T &= h^2 P \zeta \int_{-\infty}^\infty \textrm{e}^{-T^2/\widetilde{\sigma}^2} \, \df T.
\end{align*}
These expressions are easily integrated to show that $\sqrt{\pi \sigma^2} = \sqrt{\pi \widetilde{\sigma}^2} \zeta^2$, and finally that
\begin{align*}
\zeta &= \left( \frac{\sigma}{\widetilde{\sigma}} \right)^{1/2}.
\end{align*}







% !TeX root = ../Thesis.tex

\chapter{The Lambert $W$ Function}
\label{chap:lambertw}
The Lambert $W$ function is defined to be the inverse of the function $f(x) = x \textrm{e}^x$; its graph is shown in Figure \ref{fig:lambertw}. In other words, if $z = x \textrm{e}^x$ then $x = W(z)$. Notice that by combining these relations we obtain the identities
\begin{align}
\label{eq:lambertw}
z &= W(z) \textrm{e}^{W(z)}, & x &= W(x \textrm{e}^x).
\end{align}
This function is called the Lambert $W$ function because it is the logarithm of a special instance of Lambert's series---the letter $W$ is used because of the work done by E. M. Wright \cite{corless}. \\

Notice that the original function, $f(x) = x \textrm{e}^x$, is \emph{not} injective, and as a consequence, the $W$ function is multi-valued on the interval $[-1/\textrm{e},0)$. To alleviate this, occasionally the branch $W(x) \geq -1$ is denoted $W_0$ and is called the principal or upper branch, whereas the branch $W(x) < -1$ is denoted $W_{-1}$ and is called the lower branch. However, in this work the $W$ function will only take positive real values and so this distinction is not needed. \\

\begin{figure}[htbp]
\centering
\input{./Figures/LambertPlot}
\caption{The two branches of the Lambert $W$ function.}
\label{fig:lambertw}
\end{figure}

The Lambert $W$ function has applications in various areas of math and physics \cite{corless} including:
\begin{itemize}
\item Jet fuel problems
\item Combustion problems
\item Enzyme kinetics problems
\item Linear constant coefficient differential delay equations
\item Volterra equations.
\end{itemize}
Primarily, the $W$ function arises when solving iterated exponentiation or certain algebraic equations. For example, consider the equation $z = x^x$. By taking the logarithm of each side we have
\begin{align*}
\log z &= x \log x, \\
&= \log x \textrm{e}^{\log x},
\end{align*}
which after applying the $W$ function reduces to $W(\log z) = \log x$ by \eqref{eq:lambertw}. Finally, $x$ as a function of $z$ can be written as $x = \exp(W(\log z))$.


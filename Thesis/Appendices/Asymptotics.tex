% !TeX root = ../Thesis.tex
%\chapter{Asymptotic Expansions of the Variance}
%\label{chap:asymp}
We shall investigate the nature of the solution to \eqref{eq:var} both when $s \rightarrow 0$ (a very short \gls{cfbg}), as well as when $s \rightarrow \infty$ (a very long \gls{cfbg}). For ease of notation, \eqref{eq:var} is rewritten as
\begin{align}
\label{eq:asymp}
\frac{1}{4} x^4 = - \eps (x - 1)(x - 2)^2,
\end{align}
where $x = \sigma^2$, $\eps = s^4$, and $\eps \rightarrow 0$. Figure \ref{fig:asympt} shows \eqref{eq:asymp}. Additionally, suppose $x$ can be expanded as a power series in $\eps$:
\begin{align}
\label{eq:series}
x = x_0 + x_1 \eps^\alpha + x_2 \eps^\beta + \cdots
\end{align}
with $0 < \alpha < \beta$. Then, expanding \eqref{eq:asymp} using \eqref{eq:series} and collecting terms, at $\bigO{1}$,
\begin{align*}
\frac{1}{4} x_0^4 = 0,
\end{align*}
and so, $x_0 = 0$. Knowing this, we consider how to balance
\begin{align*}
\frac{1}{4} x_1^4 \eps^{4\alpha} = \eps \left( -x_1^3 \eps^{3\alpha} + 5x_1^2 \eps^{2\alpha} - 8x_1 \eps^\alpha + 4 \right).
\end{align*}
In order to have dominant balance it must be that the left hand side balances with the final term of the right hand side, that is, $4\alpha = 1$, or $\alpha = \frac{1}{4}$. Therefore, at $\bigO{\eps}$ $x_1^4 = 16$, or
\begin{align*}
(x_1 - 2)(x_1 + 2)(x_1^2 + 4) &= 0,
\end{align*}
and so, $x_1 = \pm 2, \, \pm 2 i$. However, we require $x$ to be positive and real---recall that $x = \sigma^2$---thus, we take $x_1 = 2$. Now, the next lowest order of the left hand side must be $\eps^{3\alpha + \beta} = \eps^{3/4 + \beta}$. As with the previous iteration, this must balance with the lowest order term of the $8x$. Hence, $\frac{3}{4} + \beta = 1 + \alpha = \frac{5}{4}$, thus, $\beta = \frac{1}{2}$. Now, at $\bigO{\eps^{5/4}}$,
\begin{align*}
\frac{1}{4} 4 x_1^3 x_2 &= -8x_1,
\end{align*}
and with the choice of $x_1$, $x_2 = 2$. Combining these results gives $x = 2 \eps^{1/4} - 2 \eps^{1/2} + \bigO{\eps^{3/4}}$, or $\sigma^2 = 2 s \left( 1 - s \right) + \bigO{s^3}$,
this is shown in Figure \ref{fig:lims0}. The choice $\eps = s^4$ was for convenience, but in retrospect, $\eps = s$ yields a cleaner, but identical, solution. \\
\begin{figure}[tbp]
\input{./Figures/Lim_s_0}
\caption[Asymptotic expansion of the variance as the dispersion parameter approaches 0.]{Asymptotic expansion of the variance as the dispersion parameter, $s$, approaches $0$.}
\label{fig:lims0}
\end{figure}

We shall now consider the other limit, when the dispersion parameter, $s$, approaches $\infty$. Using a similar substitution we instead write \eqref{eq:var} as
\begin{align}
\label{eq:asymp2}
\frac{1}{4} \eps x^4 = -x^3 + 5x^2 - 8x + 4,
\end{align}
where $x = \sigma^2$, and $\eps = s^{-4}$ instead so we still have $\eps \rightarrow 0$, this is also shown in Figure~\ref{fig:asympt}. Furthermore, we must make a correction to the series expansion:
\begin{align}
\label{eq:series2}
x = \eps^{-\xi} \left( x_0 + x_1 \eps^\alpha + x_2 \eps^\beta + \cdots \right).
\end{align}
The reason for this is because the equation is now singular---when $\eps=0$ the equation transforms from a quartic to a cubic, losing a root. As with before, we obtain dominant balance when the quartic term balances with the cubic term which occurs when $1 - 4 \xi = -3 \xi$, or $\xi = 1$. Then to $\bigO{\eps^{-3}}$, expression \eqref{eq:asymp2} gives
\begin{align*}
x_0^3 (x_0 + 4) &= 0.
\end{align*}
The non-trivial solution $x_0 = -4$ gives rise to a boundary layer structure. However, this is unphysical because $x \geq 0$, leaving $x_0^3 = 0$, so this singular perturbation does not generate the boundary layer expected. Therefore, \eqref{eq:series2} reduces to a regular perturbation problem, and $\alpha$ must equal 1. The next condition occurs at $\bigO{1}$ where
\begin{align*}
0 = -x_1^3 + 5x_1^2 - 8x_1 + 4 = -(x_1 - 1)(x_1 - 2)^2,
\end{align*}
so either $x_1 = 1$ or $x_1 = 2$. The next term obtains dominant balance when the left hand side balances with the linear term of the right hand side, that is, when $\beta = 2$ provided\footnote{In the case of $x_1 = 2$ we find $\beta = \frac{3}{2}$ which leads to complex roots.} $x_1 \ne 2$. Now at $\bigO{\eps}$,
\begin{align*}
\frac{1}{4} x_1^4 = -3x_1^2 x_2 + 5\cdot2 x_1 x_2 - 8x_2 = -x_2 (3x_1 - 4)(x_1 - 2),
\end{align*}
where again, the coefficients come from the binomial expansions. From this it is clear that
\begin{align*}
x_2 &= \frac{-x_1^4}{4(3x_1 - 4)(x_1 - 2)}.
\end{align*}
Thus, $x_1 = 1$, $x_2 = -\frac{1}{4}$, and
\begin{align*}
x = 1 - \frac{1}{4} \eps + x_2 \eps^{\gamma - 1} + \cdots.
\end{align*}
Expanding one final term we find $\gamma = 3$, and to $\bigO{\eps^2}$,
\begin{align*}
\frac{1}{4} 4x_2 x_1^3 &= -(3x_3 x_1^2 + 3x_2^2 x_1) + 5(2x_3 x_1 + x_2^2) - 8x_3.
\end{align*}
Knowing the values of $x_1$, and $x_2$ this is simply an arithmetical calculation yielding $x_3 = \frac{3}{8}$. Finally,
\begin{align*}
x &= 1 - \frac{1}{4}\eps + \frac{3}{8}\eps^2 + \bigO{\eps^3},
\end{align*}
or,
\begin{align*}
\sigma^2 &= 1 - \frac{1}{4s^4} + \frac{3}{8s^8} + \bigO{s^{-12}}
\end{align*}
this approximation is shown in Figure \ref{fig:limsinfty}. \\
\begin{figure}[tbp]
\input{./Figures/Lim_s_Infty}
\caption[Asymptotic expansion of the variance as the dispersion parameter approaches $\infty$.]{Asymptotic expansion of the variance as the dispersion parameter, $s$, approaches $\infty$.}
\label{fig:limsinfty}
\end{figure}

In conclusion, the asymptotic expansions of the observables are
\begin{equation}
\begin{aligned}
\sigma^2 &=
\begin{cases*}
2s(1 - s) + \bigO{s^3} & $s \rightarrow 0$ \\
1 - \frac{1}{4s^4} + \frac{3}{8s^8} + \bigO{s^{-12}} & $s \rightarrow \infty$
\end{cases*}
& C &=
\begin{cases*}
1 - s + \frac{1}{2}s^2 + \bigO{s^3} & $s \rightarrow 0$ \\
\frac{1}{2s^2} - \frac{3}{8s^6} + \bigO{s^{-10}} & $s \rightarrow \infty$
\end{cases*} \\
\frac{C}{\sigma^2} &=
\begin{cases*}
\frac{1}{2s} - \frac{1}{8}s + \bigO{s^3} & $s \rightarrow 0$ \\
\frac{1}{2s^2} - \frac{1}{4s^6} + \bigO{s^{-10}} & $s \rightarrow \infty$
\end{cases*}
& \zeta &=
\begin{cases*}
1 - \frac{1}{2}s + \frac{1}{8}s^2 + \bigO{s^3} & $s \rightarrow 0$ \\
\frac{1}{4^{1/4}s} - \frac{3 \cdot 4^{3/4}}{32 s^5} + \bigO{s^{-9}} & $s \rightarrow \infty$.
\end{cases*}
\label{eq:asympt}
\end{aligned}
\end{equation}
The general form of these is as expected. When $s \rightarrow 0$, we effectively remove the \gls{cfbg} from the cavity. This causes the pulse to be perpetually modulated, and so, the pulse's shape approaches a $\delta$ function, and therefore, $\sigma^2 \rightarrow 0$. Moreover, since $\zeta$ is the ratio of the amplitude of the pulse before and after dispersion, $\zeta \rightarrow 1$. In the opposite extreme, $s \rightarrow \infty$, the length / strength of the \gls{cfbg} approaches infinity---this causes the pulse to broaden greatly, while conserving its energy. Because of this, the variance approaches unity because the shape is entirely determined by the modulation function. Additionally, $\zeta \rightarrow 0$ because the pulse becomes infinitely wide, and thus, the amplitude becomes infinitesimally small. Due to the expansive nature, the chirp approaches zero because any chirp will be expanded away. \\

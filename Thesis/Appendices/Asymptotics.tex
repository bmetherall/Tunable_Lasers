% !TeX root = ../Thesis.tex

\chapter{Asymptotic Expansions of the Variance}
\label{chap:asymp}
We shall investigate the nature of the solution to \eqref{eq:var} both when $s \rightarrow 0$, as well as when $s \rightarrow \infty$. For ease of notation, \eqref{eq:var} is rewritten as
\begin{align*}
\frac{1}{4} x^4 = \eps \left( -x^3 + 5x^2 - 8x + 4 \right),
\end{align*}
where $x = \sigma^2$, $\eps = s^4$, and $\eps \rightarrow 0$. Additionally, suppose $x$ can be expanded as a power series in $\eps$:
\begin{align*}
x = x_0 + x_1 \eps^\alpha + x_2 \eps^\beta + \dots
\end{align*}
with $0 < \alpha < \beta$. Then,
\begin{align*}
\bigO1 : \frac{1}{4} x_0^4 = 0,
\end{align*}
and so, $x_0 = 0$. Knowing this, up to the next order we have
\begin{align*}
\frac{1}{4} x_1^4 \eps^{4\alpha} = \eps \left( -x_1^3 \eps^{3\alpha} + 5x_1^2 \eps^{2\alpha} - 8x_1 \eps^\alpha + 4 \right)
\end{align*}
in order to have dominant balance it must be that the left hand side balances with the final term of the right hand side, that is, $4\alpha = 1$, or $\alpha = \frac{1}{4}$. Therefore,
\begin{align*}
\bigO\eps : \frac{1}{4} x_1^4 &= 4, \\
(x_1 - 2)(x_1 + 2)(x_1^2 + 4) &= 0,
\end{align*}
and so, $x_1 = \pm 2, \, \pm 2 i$. However, we wish $x$ to be positive and real---recall that $x = \sigma^2$---thus, we take $x_1 = 2$. Now, the next lowest order of the left hand side must be $\eps^{3\alpha + \beta} = \eps^{3/4 + \beta}$. As with the previous iteration, this must balance with the lowest order term of the $8x$. Hence, $\frac{3}{4} + \beta = 1 + \alpha = \frac{5}{4}$, thus, $\beta = \frac{1}{2}$. Now,
\begin{align*}
\bigO{\eps^{5/4}}: \frac{1}{4} 4 x_1^3 x_2 &= -8x_1, \\
x_2 &= -2,
\end{align*}
where the additional $4$ comes from the binomial expansion. Finally, up to second(?***) order
\begin{align*}
x &\approx 2 \eps^{1/4} - 2 \eps^{1/2}, \\
\sigma^2 &\approx 2 \left( s - s^2 \right),
\end{align*}
this is shown in Figure \ref{fig:lims0}.
\begin{figure}[tbp]
\input{./Figures/Lim_s_0}
\caption{Asymptotic expansion of the variance as $s \rightarrow 0$.}
\label{fig:lims0}
\end{figure}

We shall now consider the other limit, when $s \rightarrow \infty$. Using a similar substitution we instead write \eqref{eq:var} as
\begin{align*}
\frac{1}{4} \eps x^4 = -x^3 + 5x^2 - 8x + 4,
\end{align*}
where $x = \sigma^2$, and $\eps = s^{-4}$ instead so that we still have $\eps \rightarrow 0$. Furthermore, we must make a correction to the series expansion:
\begin{align*}
x = \eps^{-\xi} \left( x_0 + x_1 \eps^\alpha + x_2 \eps^\beta + \dots \right).
\end{align*}
The reason for this is because the equation is now singular---when $\eps=0$ the equation transforms from a quartic to a cubic, losing a root. As with before, we obtain dominant balance when the quartic term balances with the cubic term:
\begin{align*}
1 - 4 \xi &= -3 \xi, \\
\xi &= 1.
\end{align*}
Then,
\begin{align*}
\bigO{\eps^{-3}}: \frac{1}{4} x_0^4 &= -x_0^3, \\
x_0^3 (x_0 + 4) &= 0,
\end{align*}
we shall choose $x_0 = 0$---since $x \geq 0$. Unlike up until this point, we obtain dominant balance when the right hand side dominates the left hand side, this is achieved when $\alpha = 1$. Now,
\begin{align*}
\bigO1 : 0 &= -x_1^3 + 5x_1^2 - 8x_1 + 4, \\
&= -(x_1 - 1)(x_1 - 2)^2,
\end{align*}
so either $x_1 = 1$ or $x_1 = 2$, we shall see in the next step that we can eliminate one of these. The next term obtains dominant balance when the left hand side balances with the linear term of the right hand side, that is, when $\beta = 2$:
\begin{align*}
\bigO\eps : \frac{1}{4} x_1^4 &= -3x_1^2 x_2 + 5\cdot2 x_1 x_2 - 8x_2, \\
&= -x_2 (3x_1 - 4)(x_1 - 2),
\end{align*}
where again, the coefficients come from the binomial expansions;
from this it is clear that
\begin{align*}
x_2 &= \frac{-x_1^4}{4(3x_1 - 4)(x_1 - 2)},
\end{align*}
and that $x_1 \ne 2$. Thus, $x_1 = 1$, $x_2 = -\frac{1}{4}$, and
\begin{align*}
x = 1 - \frac{1}{4} \eps + x_2 \eps^{\gamma - 1} + \dots.
\end{align*}
Expanding one final term we find $\gamma = 3$, and
\begin{align*}
\bigO{\eps^2}: \frac{1}{4} 4x_2 x_1^3 &= -(3x_3 x_1^2 + 3x_2^2 x_1) + 5(2x_3 x_1 + x_2^2) - 8(x_3).
\end{align*}
Knowing the values of $x_1$, and $x_2$ this is simply an arithmetical calculation yielding $x_3 = \frac{3}{8}$. Finally,
\begin{align*}
x &\approx 1 - \frac{1}{4}\eps + \frac{3}{8}\eps^2, \\
\sigma^2 &\approx 1 - \frac{1}{4s^4} + \frac{3}{8s^8},
\end{align*}
this approximation is shown in Figure \ref{fig:limsinfty}.
\begin{figure}[tbp]
\input{./Figures/Lim_s_Infty}
\caption{Asymptotic expansion of the variance for $s \rightarrow \infty$.}
\label{fig:limsinfty}
\end{figure}










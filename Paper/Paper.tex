\documentclass[12pt]{article}

\usepackage{amsfonts}
\usepackage{amsmath}
\usepackage{geometry}
\usepackage{xcolor,graphicx}
\usepackage[subfolder,cleanup]{gnuplottex}
%\usepackage{amsthm}
%\usepackage{enumitem}
%\usepackage{wrapfig}
%\usepackage{subcaption}
\usepackage{hyperref}
\usepackage{biblatex}

\bibliography{Ref.bib}

\providecommand{\df}{\textrm{d}}
\newcommand{\diff}[3][]{\frac{\textrm{d}^{#1}#2}{\textrm{d}{#3}^{#1}}}
\newcommand{\pdiff}[3][]{\frac{\partial^{#1}#2}{\partial{#3}^{#1}}}
\newcommand{\Es}{E_{\textrm{sat}}}
\newcommand{\FT}[1]{\mathcal{F}\left\{ #1 \right\}}
\newcommand{\FTi}[1]{\mathcal{F}^{-1}\left\{ #1 \right\}}

\providecommand{\bigO}[1]{\ensuremath{\mathop{}\mathopen{}\mathcal{O}\mathopen{}\left(#1\right)}}
\DeclareMathOperator{\sgn}{sgn}



%Impetutus: Want every student to have at least one experiential learning opportunity by the time they graduate.
%Stake from province: 780K received to do this
%
%Examples of experiential learning:
%industry or community agency sponsored research projects
%interactive simulations
%capstone projects
%on-campus work/teaching labs
%performance based learning
%
%Checkmark 1: Student is in a workplace or a simulated workplace.
%This could be a lab or a workstation.  Use of industry standard software.  Case study application.
%Field trips provided that it meets all of the criteria.
%
%Checkmark 2: Student is exposed to authentic demands that improve employability.
%Initiative, communication, conflict resolution, making them use their own initiative to solve
%problems that may arise.
%
%Checkmark 3: Experience is structured, with purposeful and meaningful activities.
%Planning, supervision, guidance, check-in.  There is resource from Reyerson that helps break this down
%This is not just photocopying or filing.
%
%Checkmark 4: Student applies program knowledge.
%Relevance to course learning and encourage students to be flexible.
%
%Checkmark 5: Experience includes self-evaluation and assessment of learning outcomes.
%This is a typical stumbling block and requires that the student is reflective on what was learned and
%how this applies to future learning.  Assessment of the student's work is tied to to the learning outcomes.
%Student's need to be introduced to meaningful reflection.
%
%Checkmark 6: Experience includes formal recognition.
%Recognition has to acknowledge the experience meets the previous five criteria.
%
% HIGH LEVEL IDEAS:
%Case studies - partnership with a pharmaceutical firm to develop in-office posters to explain how a 
%drug works for use by a physician. (Check out RIIPEN website)
%Field trips - design and incorporate activity, conceptualization, and reflection.  A tour of an R&D facility.
%Hackathons - Student-led intensive sessions to solve real-world problems.  Interdisciplinary teams
%to create ideas for a community healthcare initiative.
%
%
%
%Promoting a tapestry of science.


\title{}
\author{Brady Metherall \& C.\ Sean Bohun}

\newgeometry{margin=1in}
%\setlength\parindent{0pt}

\begin{document}
\maketitle


\noindent
To start off the discussion of the current modelling efforts for a tuneable laser, we begin with a review 
of the efforts to describe an `average' model.  The idea is to capture some of the physical the elements 
in the waveform (dispersion, modulation and gain/loss) described by an effective PDE, the solution of
which gives the amplitude of the wave packet.

\section{Average Model}

The model for the amplitude in the `average' model is presupposed to have the form
\begin{equation}
\label{average}
	\pdiff{A}{z} = -i\frac{\beta_2}{2}\pdiff[2]{A}{T} - \frac{\epsilon}{2}T^2 A + \frac{g}{2}A
\end{equation}
where $A=A(T,z)$ is the complex amplitude of the pulse
with $\beta_2 \in \mathbb{R}$ defining the dispersion, $\epsilon \in \mathbb{R}$, $\epsilon > 0$, determining the
modulation and $g \in \mathbb{R}$, $g > 0$ giving the gain.  Expression~(\ref{average}) is reminiscent of the 
nonlinear Schr\"odinger equation but linear in $A$ so that the solution can take the form of a Gaussian wavepacket.
Consequently, the ansatz for the amplitude is taken to be a function of the form
\begin{equation}
\label{ansatz}
	A(T,z) = \left(\frac{P_0}{1-i C}\right)^{1/2}\exp
	\left(-\frac{\delta \Omega^2 T^2}{2(1-i C)}\right)\textrm{e}^{i\psi z}
\end{equation}
so that the modulus, $|A|$, and the phase, $A/|A|$, are given by
\begin{align*}
	\left|A\right| &= \left(\frac{P_0}{\sqrt{1 + C^2}}\right)^{1/2}
	\exp\left(-\frac{\delta \Omega^2 T^2}{2(1 + C^2)}\right),&
	\frac{A}{|A|} &= \left(\frac{1+iC}{\sqrt{1+C^2}}\right)^{1/2}\exp
	\left(-i\frac{\delta \Omega^2 T^2 C}{2(1+C^2)}\right)\textrm{e}^{i\psi z}.
\end{align*}
The quantity $C \in \mathbb{R}$ is known as the chirp and contributes a constant phase of $\theta$ where
$2\theta = \arctan C$, $P_0$ is the maximum value of $|A|^2$ at $C=0$ (zero chirp), $\psi \in \mathbb{R}$
is the accumulated phase, and $\delta\Omega$ is the spectral half-width of $|A|^2$ 
since\footnote{If $f(t) = \textrm{e}^{-\alpha t^2}$ then $\hat{f}(\omega) = \frac{1}{2\pi}\int_{\mathbb{R}}
f(t)\textrm{e}^{-i\omega t}\, \df t = \frac{1}{2\pi} (\frac{\pi}{\alpha})^{1/2}\textrm{e}^{-\omega^2/4\alpha}$.}
\[
	\hat{A}(\omega,z) = \frac{1}{2\pi}\int_{-\infty}^\infty A(t,z) \textrm{e}^{-i\omega t}\, \df t
	= \left(\frac{P_0}{2\pi\delta \Omega^2}\right)^{1/2}
	\exp\left(-\frac{(1-i C)\omega^2}{2\delta \Omega^2}\right) \textrm{e}^{i\psi z},
\]
and $|\hat{A}(\delta\Omega,z)|^2 = \textrm{e}^{-1}|\hat{A}(0,z)|^2$.  The corresponding half-width of the
pulse duration comes from the expression for $|A|$ and gives
\begin{equation}
\label{duration}
	\delta T = \frac{\sqrt{1+C^2}}{\delta \Omega}.
\end{equation}

Applying the form~(\ref{ansatz}) to expression~(\ref{average}) provides the algebraic condition
\begin{align*}
	i\psi &= -i\frac{\beta_2}{2}\delta\Omega^2(1-iC)^{-1}
	\left(-1 + \delta\Omega^2(1-iC)^{-1}T^2\right)-\frac{\epsilon}{2}T^2 + \frac{g}{2} \\
	&= \frac{\beta_2\delta\Omega^2}{2(1+C^2)^2}(-C(1+C^2)+2\delta\Omega^2 T^2C) -\frac{\epsilon}{2}T^2 + \frac{g}{2}\\
	& \quad + i\frac{\beta_2\delta\Omega^2}{2(1+C^2)^2}(1+C^2 + \delta\Omega^2 T^2(C^2-1)).
\end{align*}
Treating this condition as a complex valued quadratic in $T$ gives four conditions.  Assuming that $\psi\in\mathbb{R}$,
the real and imaginary components at $\bigO{1}$ and $\bigO{T^2}$ yield
\begin{align*}
	&\bigO{T^2}_{\textrm{Im}}: 0 = C^2 - 1, &
	&\bigO{1}_{\textrm{Im}}: \psi = \frac{\beta_2\delta\Omega^2}{2(1+C^2)}, \\
	&\bigO{T^2}_{\textrm{Re}}: \epsilon = \frac{2\beta_2\delta\Omega^4C}{(1+C^2)^2}, &
	&\bigO{1}_{\textrm{Re}}: g = \frac{\beta_2\delta\Omega^2C}{(1+C^2)}.
\end{align*}
Starting with $\bigO{1}_{\textrm{Re}}$ we note that $g > 0$ implies that 
$\sgn(\beta_2 C) = \sgn(\beta_2)\sgn(C) = 1$. 
From $\bigO{T^2}_{\textrm{Im}}$, $C = \pm 1$ and therefore $C = \sgn(\beta_2)$, 
$\beta_2 C = |\beta_2|$ and $\epsilon > 0$, consistent with~(\ref{average}).  
The result is a two parameter family of solutions
of the form~(\ref{ansatz}) with
\begin{align*}
	C &= \sgn(\beta_2),&
	\delta\Omega^2 &= \left(\frac{2\epsilon}{|\beta_2|}\right)^{1/2},&
	\psi &= \sgn(\beta_2)\left(\frac{\epsilon|\beta_2|}{8}\right)^{1/2},&
	g &= \left(\frac{\epsilon|\beta_2|}{2}\right)^{1/2}.
\end{align*}
Moreover, we also see that the representation~(\ref{ansatz})
as a classical solution of~(\ref{average}) imposes a subclass of solutions with 
$g = \left(\epsilon|\beta_2|/2\right)^{1/2}$.  
A useful pulse characterization is the half-width of the pulse duration, $\delta T$, which satisfies
\begin{align*}
	\delta T^2 &= \left(\frac{2|\beta_2|}{\epsilon}\right)^{1/2}.
\end{align*}

Since~(\ref{average}) is linear in $A$, any value of the peak power, $P_0$, is admissible.  In practice, 
at high power levels, the gain drops as the fibre saturates.  One can model this with a gain term that
takes the form
\begin{equation}
\label{newg}
	g(P_0) = \frac{g_0}{1+\frac{\textrm{power in fibre}}{\textrm{saturation power}}} - \alpha
\end{equation}
where $g_0$ is the low-power gain and $\alpha$ represents the net losses in the laser cavity.
The power in the fibre depends on the frequency $f$ and modulus of the pulse so that
\begin{align*}
	f \int_{-\infty}^\infty |A(s)|^2\, \df s &=
	\frac{\sqrt{\pi}f}{\delta\Omega}P_0 = \Delta P_0, &
	\Delta &= \frac{\sqrt{\pi}f}{\delta\Omega} = \frac{\sqrt{\pi}f\delta T}{\sqrt{1+C^2}}=
	\sqrt{\pi}f \left(\frac{|\beta_2|}{2\epsilon}\right)^{1/4}
\end{align*}
where $\Delta$ is the duty cycle of the pulse.  Denoting the saturation power as 
$P_{\textrm{sat}}$,~(\ref{newg}) can be inverted to give
\[
	P_0 = \frac{P_{\textrm{sat}}}{\sqrt{\pi}f}  \left(\frac{2\epsilon}{|\beta_2|}\right)^{1/4}
	\left(g_0\left(\left(\frac{\epsilon|\beta_2|}{2}\right)^{1/2} + \alpha\right)^{-1} - 1\right).
\]

Figure~\ref{fig:average} shows a select set of possible waveforms in $(\epsilon,|\beta_2|)$ parameter space.  
The representative hyperbola, $\epsilon|\beta_2| = 2g^2$ correspond to constant gain while
the rays through the origin, $|\beta_2|/\epsilon = \delta T^4/4$, correspond to constant width.
Curves of constant power level are also indicated.  One of the main drawbacks of this model is 
the fixed chirp and its symmetric behaviour with respect to $T$.  One way to alleviate some of these
restrictions is to construct a discrete model whereby each of the modules in the tuneable laser
are given by a transfer function that is motivated by the PDE~(\ref{average}).  In particular, the
dispersion and the modulation would correspond to transfer functions of the form
\begin{align}
	\hat{A}_{\textrm{out}}(\omega) &= \hat{A}_{\textrm{in}}(\omega)\textrm{e}^{i\beta_2\omega^2/2},&
	A_{\textrm{out}}(T) &= A_{\textrm{in}}(T)\textrm{e}^{-\epsilon T^2/2},
\end{align}
respectively where the dispersion is naturally defined in the frequency domain.  Between the modules, 
the pulse is assumed to propagate according to the representation~(\ref{ansatz}) chosen to be consistent
with the transfer functions.  Details of this technique and its ability of predict a number of experimental
effects can be found in~\cite{burgoyne2014}.



\section{A New Model}
Rather than using a transfer function with a linear PDE, we instead return to generalized nonlinear 
Schr\"odinger equation~\cite{agrawal2013, ferreira, shtyrina, yarutkina} 
\begin{align}
\label{eq:nlse}
\pdiff{A}{z} = - i \frac{\beta_2}{2}\pdiff[2]{A}{T} + \frac{\beta_3}{6}\pdiff[3]{A}{T} 
+ i \gamma |A|^2 A + \frac{1}{2}g(A) A - \alpha A,
\end{align}
to represent the waveform.  In this expression, $\beta_3$ is the third order dispersion coefficient, 
$\gamma$ is the coefficient of nonlinearity, $g(A)$ is the gain, and $\alpha$ is the loss of the fibre.
Using this expression as a starting point, the laser cavity is assumed to be composed of five independent 
processes---gain, nonlinearity, loss, dispersion, and modulation. 
Within each component of the laser cavity the other four process are assumed to be negligible, that is, 
each process is dominant only within one part of the laser, just as with the discrete model, but embracing
any nonlinearities.

\subsection{Gain}
Considering the gain term as dominant as is expected with the 
the Er-doped gain fibre, equation~\eqref{eq:nlse} reduces to
\begin{align}
\label{eq:gainde}
	\pdiff{A}{z} &= \frac{1}{2} g(A) A,& g(A) &= \frac{g_0}{1 + E / \Es},& E &= \int_{-\infty}^\infty |A|^2 \, \df T,
\end{align}
where $g_0$ is a small signal gain, $E$ is the energy of the 
pulse~\cite{bohun, burgoyne2014, shtyrina, silfvast, yarutkina} and $\Es$ is the energy at which the gain 
begins to saturate.
%We shall first transform \eqref{eq:gainde} into an equation in terms of the energy. 
Multiplying~\eqref{eq:gainde} by $\bar{A}$, the complex conjugate of $A$, yields
\begin{align*}
	2\bar{A} \pdiff{A}{z} = \frac{g_0 |A|^2}{1 + E / \Es}.
\end{align*}
Adding this to its complex conjugate and integrating over $T$ gives
%\begin{align*}
%	\diff{|A|^2}{z} &= \frac{g_0 |A|^2}{1 + E / \Es}.
%\end{align*}
%After integrating this becomes
\begin{align}
\label{diffez}
	\diff{E}{z} &= \frac{g_0 E}{1 + E / \Es}.
\end{align}
For $E \ll \Es$ the energy grows exponentially, whereas for $E \gg \Es$ the gain has saturated and so the energy grows linearly.  To obtain a closed form, integrate~(\ref{diffez}) over a gain fibre of length $z$ and assume the energy 
increases from $E$ to $E_{\textrm{out}}$ so that
\[
	g_0 z = \log\frac{E_{\textrm{out}}}{E} + \frac{E_{\textrm{out}}-E}{\Es}
\]
and by exponentiating, rearranging, and applying $W_0$, the positive branch of the Lambert $W$ function,
\[
	 W_0\left(\frac{E}{\Es} \textrm{e}^{E/\Es} \textrm{e}^{g_0 z}\right) = 
	 W_0\left(\frac{E_{\textrm{out}}}{\Es} \textrm{e}^{E_{\textrm{out}}/\Es}\right) = \frac{E_{\textrm{out}}}{\Es}.
\]
This results in the closed form expression
\begin{align}
\label{eq:energy}
	E_{\textrm{out}}(z) = \Es W_0 \left( \frac{E}{\Es} \textrm{e}^{E/\Es} \textrm{e}^{g_0 z} \right)
\end{align}
with the desired property that $E_{\textrm{out}}(0)=E$.  Since $E \sim |A|^2$, the gain in terms of 
the amplitude is given by
\begin{align}
\label{eq:gain}
	G(A;E) &= \left(\frac{E_{\textrm{out}}(L_g)}{E}\right)^{1/2}A = 
	\left( \frac{\Es}{E} W_0 \left( \frac{E}{\Es} \textrm{e}^{E/\Es} 
	\textrm{e}^{g_0 L_g} \right) \right)^{1/2} A,
\end{align}
where $L_g$ is the length of the gain fibre.


\subsection{Fibre Nonlinearity}
The nonlinearity of the fibre depends on the parameter $\gamma$.  In regions where this effect is dominant 
expression~\eqref{eq:nlse} gives
\begin{align}
\label{eq:fibrediff}
	\pdiff{A}{z} - i \gamma |A|^2 A = 0,
\end{align}
so that $\frac{\partial}{\partial z} |A|^2 = 0$ suggesting that $A(T,z) = A_0(T) e^{i \varphi(T,z)}$.
Substituting this representation into~\eqref{eq:fibrediff} and setting $\varphi(T,0)=0$ gives 
$\varphi(T,z) = \gamma |A|^2 z$. For a fibre of length $L_f$ the effect of the nonlinearity is therefore
\begin{align}
\label{eq:fibre}
	F(A) &= A e^{i \gamma |A|^2 L_f}.
\end{align}

\subsection{Loss}
Two sources of loss exist within the laser circuit: the loss due to the output coupler and the optical loss 
due to absorption and scattering. Combining these two effect gives a loss that takes the form
\begin{align}
\label{eq:fibreloss}
	L(A) &= C e^{- \alpha L}A,
\end{align}
where $C$ is the percentage lost due to the output coupler, and $L$ is the length of the laser circuit.

\subsection{Dispersion}
Within the laser cavity, the dispersion is dominated by the chirped fibre Bragg grating (CFBG).  In comparision,
the dispersion due to the fibre is negligible.\footnote{A $10$ cm chirped grating can provide as much dispersion 
as $300$ km of fibre~\cite{agrawal2002}.} The dispersive terms of \eqref{eq:nlse} give
\begin{align}
\label{eq:disp}
	\pdiff{A}{z} = -i \frac{\beta_2}{2} \pdiff[2]{A}{T} + \frac{\beta_3}{6} \pdiff[3]{A}{T}
\end{align}
and since dispersion acts in the frequency domain, it is convenient to use the Fourier transform 
of~\eqref{eq:disp}, giving the result that
\begin{align*}
	\pdiff{\FT{A}}{z} &= i\frac{\omega^2}{2}\left(\beta_2 + \frac{\beta_3}{3} \omega\right) \FT{A}.
\end{align*}
The affect of dispersion is then
\begin{align}
\label{eq:dispersion}
	D(A) &= \FTi{\textrm{e}^{i \omega^2 L_D(\beta_2 + \beta_3 \omega/3)/2} \FT{A}}.
\end{align}
For a highly dispersive media the third order effects may need to be considered~\cite{agrawal2013, litchinitser}.
However, for simplicity in the basic model, the third order effect will be neglected so we set $\beta_3=0$ for 
the subsequent analysis~\cite{agrawal2013, ferreira}.

\subsection{Modulation}
In the average model the amount of modulation is characterized by the parameter $\epsilon$ through
the term $\frac{\epsilon}{2}T^2 A$.  In the new model, the modulation is considered to be applied
externally through its action on the spectrum and for simplicity the representation is taken as a
Gaussian pulse  
\begin{align}
	M(A) &= \textrm{e}^{-T^2 / 2 T_M^2} A,
\end{align}
where $T_M$ is a characteristic width of the modulation.\footnote{Taking the Fourier transform, 
$\FT{\textrm{e}^{-T^2/ 2 T_M^2}} = \frac{1}{\sqrt{2\pi}}T_M\textrm{e}^{-\omega^2T_M^2/2}$.}



\subsection{Non-Dimensionalization}
The structure of each process of the laser can be better understood by re-scaling the time, energy, and amplitude. Specifically, the time shall be scaled by the characteristic modulation time, the energy by the saturation energy, and the amplitude will be scaled so that it is consistent:
\begin{align*}
	T &= T_M \widetilde{T},& E &= \Es \widetilde{E},& A &= \left( \frac{\Es}{T_M} \right)^{1/2} \widetilde{A}.
\end{align*}
The new process maps, after dropping the tildes, become
\begin{align*}
	G(A) &= \left(E^{-1} W_0 \left( a E e^{E}\right) \right)^{1/2} A,&
	F(A) &= A \textrm{e}^{i b |A|^2},&
	L(A) &= h A,\\
	D(A) &= \FTi{\textrm{e}^{i s^2 \omega^2} \FT{A}},&
	M(A) &= \textrm{e}^{-T^2 / 2} A,
\end{align*}
with four dimensionless parameter groups (see Table~\ref{tab:values})
\begin{align*}
	a &= \textrm{e}^{g_0 L_g} \sim 30,& s &= \sqrt{\frac{\beta_2 L_D}{2 T_M^2}} \sim 0.1,&
	b &= \gamma L_F \frac{\Es}{T_M} \sim 0.1,& h &= C \textrm{e}^{-\alpha L} \sim 0.25,
\end{align*}
which control the behaviour of the laser.

\subsection{Combining the Pieces}
In this model the pulse is iteratively passed through each process, the order of which is now
important.  In this first realization, the pulse is first
amplified by the gain fibre, then since the pulse's magnitude is greatest the nonlinearity 
needs to be considered. The pulse is then tapped off by the output coupler, and then passes 
through the grating and is modulated. The pulse after one complete circuit of the laser 
cavity is then passed back in to restart the process. Functionally this can be denoted as
\begin{align*}
	\mathcal{L}(A) = M(D(L(F(G(A))))),
\end{align*}
where $\mathcal{L}$ is one loop of the circuit. A solution to this model is one in which the 
envelope is unchanged after traversing every component in the cavity, that is, such 
that $|\mathcal{L}(A)| = |A|$.




\section{Results}
The model described is solved numerically and somewhat surprisingly, its convergence as a function
of the characteristic parameter groups is highly complex.  To gain some insight into this process
we consider an linear analysis that ignores the nonlinear fibre effects.

\subsection{Linear Solution}
Since the pulse is modulated by a Gaussian, it is expected that the envelope will also converge 
to a Gaussian.  Consider an initial pulse with envelope
\begin{align*}
	|A_0| = \sqrt{P} \textrm{e}^{-T^2 / 2 \sigma^2},
\end{align*}
after passing through the gain and loss pieces, and neglecting the fibre nonlinearity, the envelope 
will have the form
\begin{align*}
	|A| &= \sqrt{P} g(E) h \textrm{e}^{-T^2 / 2 \sigma^2},&
	g(E) &= \left( \frac{W_0(a E \textrm{e}^E)}{E} \right)^{1/2},
\end{align*}
where $g(E)$ is the gain component. After the pulse travels through the grating, the envelope will 
maintain its Gaussian shape, however, it will have spread. This can be represented as
\begin{align}
\label{linear1}
	|A| = \sqrt{P} g(E) h \zeta(s, 0) \textrm{e}^{-T^2 / 2 \tilde{\sigma}^2}
\end{align}
where $\zeta(s, b)$ is the decrease in the amplitude and $\widetilde{\sigma}^2$ is the new variance. 
Because the dispersive element conserves energy,
\begin{align*}
	\int_{-\infty}^{\infty} P g(E)^2 h^2 \textrm{e}^{-T^2 / \sigma^2} \, \df T = 
	\int_{-\infty}^{\infty} P g(E)^2 h^2 \zeta^2(s, 0) \textrm{e}^{-T^2 / \widetilde{\sigma}^2} \, \df T,
\end{align*}
or more simply $\sigma = \zeta^2(s, 0) \widetilde{\sigma}$.
Therefore, the envelope as the pulse exits the grating given by~(\ref{linear1}) is
\begin{align*}
	|A| = \sqrt{P} g(E) h \zeta(s, 0) \exp \left( -\zeta^4(s, 0) \frac{T^2}{2 \sigma^2} \right).
\end{align*}
Finally, the pulse is modulated and so that after one loop the total effect is given by
\begin{align}
\label{linear2}
	|A| = \sqrt{P} g(E) h \zeta(s, 0) \exp \left( -\zeta^4(s, 0) \frac{T^2}{2 \sigma^2} - \frac{T^2}{2} \right).
\end{align}

In equilibrium the envelope of the pulse is unchanged after completing a full circuit, in other words,
\begin{align*}
	\sqrt{P} \textrm{e}^{-T^2 / 2 \sigma^2} =  
	\sqrt{P} g(E) h \zeta(s, 0) \exp \left( -\zeta^4(s, 0) \frac{T^2}{2 \sigma^2} - \frac{T^2}{2} \right),
\end{align*}
or more simply,
\begin{align}
\label{eq:energycond}
	\left( \frac{W_0(a E \textrm{e}^E)}{E} \right)^{1/2} h \zeta(s, 0) &= 1,&
	-\frac{T^2}{2 \sigma^2} &= -\zeta^4(s, 0) \frac{T^2}{2 \sigma^2} - \frac{T^2}{2}.
\end{align}
The second condition simplifies to
\begin{align*}
	\sigma = \sqrt{1 - \zeta^4(s, 0)}
\end{align*}
and the peak of the pulse can be found by integration to give
\begin{align*}
	P = \frac{E}{\sqrt{\pi} \sigma}.
\end{align*}
Furthermore, from~\eqref{eq:energycond} the equilibrium energy is found to be
\begin{align*}
	E = \frac{\zeta^2(s, 0) h^2}{1 - \zeta^2(s, 0) h^2} \ln \left( a \zeta^2(s, 0) h^2 \right).
\end{align*}


\begin{figure}[htbp]
\centering
\begin{gnuplot}[terminal=epslatex, terminaloptions={color size 6in,3.7in lw 2}]
# Constant power things
Ps = 1
f = 1
g = 1
a = 0.1
f(x,y) = Ps / (sqrt(pi) * f) * (y / (2 * x))**(1/4) * (g / (sqrt(x * y / 2) + a) -1)

set grid
set xr [0:10]
set yr [0:10]
set xl '$\epsilon$'
set yl '$|\beta_2|$'

# Put contours in file
set contour base  
set cntrparam levels incremental 0, 0.1
unset surface  
set table 'curve.dat'
set isosamples 1000
splot f(x,y)  
unset table  

# Plot everything
p 2 / x t 'Constant Gain' lc 1 dt 5, \
for [g=1:5] 2 * g**2 / x not lc 1 dt 5, \
0 t 'Constant Duration' lc 4 dt 5, \
for [T=0:20:2] x / 2 * (T/10.0)**4 not lc 4 dt 5, \
'curve.dat' w l t 'Constant Power' lw 2

\end{gnuplot}
\caption{}
\label{fig:}
\end{figure}





\begin{figure}[htbp]
\centering
\begin{gnuplot}[terminal=epslatex, terminaloptions={color size 6in,3.7in lw 3}]
f(x) = m * x + c
fit f(x) '../Zeta.dat' u 1:(log($2)) via m, c
set grid
set ytics 0.1
set xl '$s$'
set yl '$\zeta(s, 0)$'
set yr [0.75:1.05]
plot '../Zeta.dat' u 1:2 w l t 'Data', \
exp(f(x)) dt 4 t '$\exp(-0.511444s + 0.00174128)$'
\end{gnuplot}
\caption{}
\label{fig:zeta}
\end{figure}



\begin{figure}[htbp]
\centering
\begin{gnuplot}[terminal=epslatex, terminaloptions={color size 6in,3.7in lw 3}]
set grid
set xl '$s$'
set yl '$b$'
set cbl 'Error'
set xr [0:1.5]
set view map

set palette defined ( 0 '#D73027',\
    	    	      1 '#F46D43',\
		      2 '#FDAE61',\
		      3 '#FEE090',\
		      4 '#E0F3F8',\
		      5 '#ABD9E9',\
		      6 '#74ADD1',\
7 '#4575B4' )

set cbr [0:0.5]
sp '../abig.dat' using 1:2:3 w image not
\end{gnuplot}
\caption{Error is calculated by $\displaystyle \frac{\|A(\text{iteration } 75) - A(\text{iteration } 74) \|_2}{\|A(\text{iteration } 74) \|_2}$}
\label{fig:}
\end{figure}

\begin{figure}[htbp]
\centering
\begin{gnuplot}[terminal=epslatex, terminaloptions={color size 6in,3.7in lw 3}]
set grid
set xl '$s$'
set yl '$b$'
set cbl 'Energy'
set xr [0:1.5]
set view map

set palette defined ( 0 '#D73027',\
    	    	      1 '#F46D43',\
		      2 '#FDAE61',\
		      3 '#FEE090',\
		      4 '#E0F3F8',\
		      5 '#ABD9E9',\
		      6 '#74ADD1',\
7 '#4575B4' )

set cbr [0:0.25]
sp '../abig.dat' using 1:2:4 w image not
\end{gnuplot}
\caption{}
\label{fig:}
\end{figure}



\begin{figure}[htbp]
\centering
\begin{gnuplot}[terminal=epslatex, terminaloptions={color size 6in,3.7in lw 3}]
S(x) = exp(-0.511444 * x + 0.0017412)
E(x, y) = -y**2 * log(a * y**2 * S(x)**2) * S(x)**2 / (y**2 * S(x)**2 - 1)
a = 30

set grid
set pm3d map
set isosamples 100, 100
set hidden3d
set contour surface
set cntrparam levels incremental 0, 0.1, 1.5
set palette rgbformulae 22, 13, -31

set palette defined ( 0 '#D73027',\
    	    	      1 '#F46D43',\
		      2 '#FDAE61',\
		      3 '#FEE090',\
		      4 '#E0F3F8',\
		      5 '#ABD9E9',\
		      6 '#74ADD1',\
7 '#4575B4' )

set ytics 0.1
set xl '$s$'
set yl rotate by 0 '$h$'
set cbl rotate by 0 '$E$'
set xr [0:0.5]
set yr [0.25:0.6]
set cbr [0:1.4]

unset clabel

splot E(x, y) not
\end{gnuplot}
\caption{Equation \eqref{eq:energycond} for $a = 30$}
\label{fig:}
\end{figure}


\begin{table}
\begin{center}
\begin{tabular}{|l|l|}
\hline
Parameter & Value \\
\hline
$\beta_2^g L_D$ & $10$--$2000 \text{ps}^2$ \\
$g_0$ & $1$--$10 \text{m}^{-1}$ \\
$\beta_2^f$ & $20$--$50 \text{ps}^2/ \text{km}$ \\
$\gamma$ & $0.001$--$0.01 \text{W}^{-1} \text{m}^{-1}$ \cite{agrawal2013} \\
$\Es$ & $10^4 \text{pJ}$ \\
$L_G$ & $2$--$3 \text{m}$ \cite{burgoyne2014, shtyrina, yarutkina} \\
\hline
\end{tabular}
\caption{}
\label{tab:values}
\end{center}
\end{table}


\newpage
\printbibliography
\end{document}

\documentclass[12pt]{article}

\usepackage{amsfonts}
\usepackage{amsmath}
\usepackage{geometry}
\usepackage{xcolor,graphicx}
\usepackage[subfolder,cleanup]{gnuplottex}
%\usepackage{amsthm}
%\usepackage{enumitem}
%\usepackage{wrapfig}
%\usepackage{subcaption}
\usepackage{hyperref}
\usepackage{biblatex}

\bibliography{Ref.bib}

\newcommand{\diff}[3][]{\frac{d^{#1}#2}{d{#3}^{#1}}}
\newcommand{\pdiff}[3][]{\frac{\partial^{#1}#2}{\partial{#3}^{#1}}}
\newcommand{\Es}{E_{sat}}
\newcommand{\FT}[1]{\mathcal{F}\left\{ #1 \right\}}
\newcommand{\FTi}[1]{\mathcal{F}^{-1}\left\{ #1 \right\}}

\title{}
\author{Brady Metherall}

\newgeometry{margin=1in}
\setlength\parindent{0pt}

\begin{document}
\maketitle

\section{Model}
The laser cavity is treated as the composition of five independent process---gain, nonlinearity, loss, dispersion, and modulation. Expressions for each component will be derived from the  generalized nonlinear Schr\"{o}dinger equation \cite{agrawal2013, ferreira, shtyrina, yarutkina},
\begin{align}
\label{eq:nlse}
\pdiff{A}{z} = - i \frac{1}{2} \beta_2 \pdiff[2]{A}{T} + \frac{1}{6} \beta_3 \pdiff[3]{A}{T} + i \gamma |A|^2 A + \frac{1}{2}g(A) A - \alpha A,
\end{align}
where $A$ is the complex amplitude of the pulse, $\beta_2$ and $\beta_3$ are the second and third order dispersion coefficients respectively, $\gamma$ is the coefficient of nonlinearity, $g(A)$ is the gain, and $\alpha$ is the loss of the fiber.

\subsection{Gain}
Within the Er-doped gain fiber the other four processes are assumed to be negligible, and so \eqref{eq:nlse} reduces to
\begin{align}
\label{eq:gainde}
\pdiff{A}{z} = \frac{1}{2} g(A) A
\end{align}
with
\begin{align*}
g(A) &= \frac{g_0}{1 + E / \Es},
\end{align*}
where $g_0$ is a small signal gain, $\Es$ is the energy at which the gain begins to saturate, and \begin{align*}
E = \int_{-\infty}^\infty |A|^2 \, dT,
\end{align*}
is the energy of the pulse \cite{bohun, burgoyne2014, shtyrina, silfvast, yarutkina}. We shall first transform \eqref{eq:gainde} into an equation in terms of the energy. Taking
\begin{align*}
\pdiff{A}{z} = \frac{1}{2} \cdot \frac{g_0 A}{1 + E / \Es},
\end{align*}
multiplying by $\bar{A}$, the complex conjugate of $A$, and then adding the complex conjugate gives
\begin{align*}
\diff{|A|^2}{z} &= \frac{g_0 |A|^2}{1 + E / \Es},
\end{align*}
which after integrating becomes
\begin{align*}
\diff{E}{z} &= \frac{g_0 E}{1 + E / \Es}.
\end{align*}
For $E \ll \Es$ the energy grows exponentially, whereas for $E \gg \Es$ the gain has saturated and so the energy grows linearly. A closed form of the energy can be found by separating and integrating yielding
\begin{align}
\label{eq:energy}
E(z) = \Es W_0 \left( \frac{E_0}{\Es} e^{E_0 / \Es} e^{g_0 z} \right),
\end{align}
where $W_0$ is the Lambert $W$ function. However, only the exiting energy is of interest, thus \eqref{eq:energy} can be written as
\begin{align*}
E' = \Es W_0 \left( \frac{E}{\Es} e^{E / \Es} e^{g_0 L_g} \right),
\end{align*}
where $E$ is the energy of the incoming pulse, and $E'$ is the energy after traveling through the length of the gain fiber. Since $E \sim |A|^2$ the gain in terms of the amplitude is given by
\begin{align}
\label{eq:gain}
G(A) &= \left[ \frac{\Es}{E} W_0 \left( \frac{E}{\Es} e^{E / \Es} e^{g_0 L_g} \right) \right]^{1/2} A.
\end{align}

\subsection{Nonlinearity of Fiber}
The effect of the nonlinearity of the fiber can also be found from \eqref{eq:nlse}:
\begin{align*}
\pdiff{A}{z} - i \gamma |A|^2 A = 0,
\end{align*}
which similarly to the gain, can be transformed by multiplying by the complex conjugate of $A$, and adding its complex conjugate gives
\begin{align}
\label{eq:fibre}
\pdiff{|A|^2}{z} &= 0.
\end{align}
This suggests the envelope of the pulse does not change as it travels through the fiber---a solution of the form $A = A_0 e^{i \varphi}$ can be assumed. Substituting this expression into \eqref{eq:fibre} gives $\varphi = \gamma |A|^2 z$ therefore
\begin{align*}
F(A) &= A e^{i \gamma |A|^2 L_f},
\end{align*}
where $L_f$ is the length of the fiber.

\subsection{Loss}
Two sources of loss exist within the laser circuit: the loss due to the output coupler and the optical loss due to the circuit. The loss is then given as
\begin{align*}
L(A) &= C e^{- \alpha L}A,
\end{align*}
where $C$ is the loss due to the output coupler.

\subsection{Dispersion}
The dominant dispersion is due to the chirped fiber Bragg grating (CFBG). The dispersion due to the fiber is negligible in comparison---a $10$cm chirped grating can provide as much dispersion as $300$km of fiber \cite{agrawal2002}. The dispersive terms of \eqref{eq:nlse} give
\begin{align}
\label{eq:disp}
\pdiff{A}{z} = -i \frac{1}{2} \beta_2 \pdiff[2]{A}{T} + \frac{1}{6} \beta_3 \pdiff[3]{A}{T},
\end{align}
however, the third order effects of dispersion will be neglected \cite{agrawal2013, ferreira}. Although, under some circumstances they may need to be considered since the grating is highly dispersive \cite{agrawal2013, litchinitser}. Taking the Fourier transform of \eqref{eq:disp} gives
\begin{align*}
\pdiff{\FT{A}}{z} &= \frac{1}{2} i \beta_2 \omega^2 \FT{A}.
\end{align*}
The Fourier transform of $A$ can be found by integrating and thus,
\begin{align*}
D(A) &= \FTi{e^{i \frac{1}{2} \beta_2 \omega^2 L_D} \FT{A}}
\end{align*}
is the effect of the grating on the pulse.

\subsection{Modulation}
Pick a function that is bump-like, infinitely differentiable (I don't know how important that is), and has a maximum of $1$. We shall pick
\begin{align*}
M(A) &= e^{-T^2 / 2 T_M^2} A,
\end{align*}
where $T_M$ is a characteristic width of the modulation, since its Fourier transform is itself. ***I don't know how to write this.

\subsection{Combining the Pieces}

\begin{align*}
\mathcal{L}(A) = M(D(L(F(G(A)))))
\end{align*}

A solution such that $|\mathcal{L}(A)|^2 = |A|^2$ is sought.

\section{Non-Dimensionalization}

\begin{align*}
T = T_M \widetilde{T}, \quad E = \Es \widetilde{E}, \quad A = \left( \frac{\Es}{T_M} \right)^{1/2} \widetilde{A}, \quad \omega = \frac{\widetilde{\omega}}{T_M}
\end{align*}

\begin{align*}
G(A) &= \left( \frac{W_0 \left( a E e^{E}\right)}{E} \right)^{1/2} A.
\end{align*}

\begin{align*}
D(A) &= \FTi{e^{i s^2 \omega^2} \FT{A}}
\end{align*}

\begin{align*}
F(A) &= A e^{i b |A|^2},
\end{align*}

\begin{align*}
L(A) &= h A,
\end{align*}

\begin{align*}
M(A) &= e^{-T^2 / 2} A
\end{align*}

\begin{align*}
a = e^{g_0 L_g} \quad s^2 = \frac{\beta_2 L_D}{2 T_M^2}, \quad b = \gamma L_F \frac{\Es}{T_M}, \quad h = C e^{-\alpha L}
\end{align*}


\section{Results}

\begin{align*}
\left( \frac{W(a E e^E)}{E} \right)^{1/2} S(s) h = 1
\end{align*}

\begin{align*}
E = \frac{S(s)^2 h^2}{1 - S(s)^2 h^2} \ln \left( a S(s)^2 h^2 \right)
\end{align*}


\begin{figure}[htbp]
\centering
\begin{gnuplot}[terminal=epslatex, terminaloptions={color size 6in,3.7in lw 3}]
f(x) = m * x + c
fit f(x) '../sdata.dat' u 1:(log($2)) via m, c
set grid
set ytics 0.1
set xl '$s$'
set yl 'Spread'
plot '../sdata.dat' u 1:2 w p pt 7 ps 0.4 t 'Data', \
exp(f(x)) t '$\exp(-0.521294s + 0.00365752)$'
\end{gnuplot}
\caption{}
\label{fig:}
\end{figure}

\begin{figure}[htbp]
\centering
\begin{gnuplot}[terminal=epslatex, terminaloptions={color size 6in,3.7in lw 3}]
S(x) = exp(-0.5131 * x + 0.00174)
E(x, y) = -y**2 * log(a * y**2 * S(x)**2) * S(x)**2 / (y**2 * S(x)**2 - 1)
a = 30

set grid

set pm3d map

set isosamples 100, 100
set hidden3d

set contour surface
set cntrparam levels incremental 0, 0.1, 1.5

set palette rgbformulae 22, 13, -31

set ytics 0.1
set xl '$s$'
set yl rotate by 0 '$h$'
set cbl rotate by 0 '$E$'
set xr [0:0.5]
set yr [0.3:0.6]

unset clabel

splot E(x, y) not

\end{gnuplot}
\caption{a=30}
\label{fig:}
\end{figure}


\begin{table}
\begin{tabular}{|l|l|}
\hline
Parameter & Value \\
\hline
$\beta_2^g L_D$ & $10$--$2000 \text{ps}^2$ \\
$g_0$ & $1$--$10 \text{m}^{-1}$ \\
$\beta_2^f$ & $20$--$50 \text{ps}^2/ \text{km}$ \\
$\gamma$ & $0.001$--$0.01 \text{W}^{-1} \text{m}^{-1}$ \\
$\Es$ & $10^4 \text{pJ}$ \\
$L_G$ & $3 \text{m}$ \\
\hline
\end{tabular}
\end{table}


\newpage
\printbibliography
\end{document}

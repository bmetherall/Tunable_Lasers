\begin{figure}[htbp]
\centering
\begin{subfigure}[]{\textwidth}
\begin{gnuplot}[terminal=epslatex, terminaloptions={color size 6.5in,3in lw 3}]
set grid
set key above
set xr [-2:2]
set yr [-0.75:0.75]
set xl '$T$'
set xtics 1

unset key
set multiplot layout 1, 2

set size 0.5, 1
p '../Stable/Envelope8.dat' u 1:4 w l t 'Magnitude', \
 '../Stable/Envelope8.dat' u 1:2 w l dt 5 lc 7 t 'Real', \
'../Stable/Envelope8.dat' u 1:3 w l dt 4 t 'Imaginary'

set size 0.5, 1
p '../Stable/Envelope15.dat' u 1:4 w l t 'Magnitude', \
 '../Stable/Envelope15.dat' u 1:2 w l dt 5 lc 7 t 'Real', \
'../Stable/Envelope15.dat' u 1:3 w l dt 4 t 'Imaginary'

set key at 0, 0.8
unset ytics
unset xtics
unset border
unset xl
set size 1, 0.1
set yr [0:0.01]

p 2 t 'Magnitude', \
2 dt 5 lc 7 t 'Real', \
2 dt 4 t 'Imaginary'
\end{gnuplot}
\caption{Wave is stable for $b = 1.5$.}
\label{fig:stable}
\end{subfigure} \\
\vspace{4mm}
\begin{subfigure}[]{\textwidth}
\begin{gnuplot}[terminal=epslatex, terminaloptions={color size 6.5in,3in lw 3}]
set grid
set key above
set xr [-2:2]
set yr [-0.75:0.75]
set xl '$T$'
set xtics 1

unset key
set multiplot layout 1, 2

set size 0.5, 1
p '../Break/Envelope8.dat' u 1:4 w l t 'Magnitude', \
 '../Break/Envelope8.dat' u 1:2 w l dt 5 lc 7 t 'Real', \
'../Break/Envelope8.dat' u 1:3 w l dt 4 t 'Imaginary'

set size 0.5, 1
p '../Break/Envelope15.dat' u 1:4 w l t 'Magnitude', \
 '../Break/Envelope15.dat' u 1:2 w l dt 5 lc 7 t 'Real', \
'../Break/Envelope15.dat' u 1:3 w l dt 4 t 'Imaginary'

set key at 0, 0.8
unset ytics
unset xtics
unset border
unset xl
set size 1, 0.1
set yr [0:0.01]
p 2 t 'Magnitude', \
2 dt 5 lc 7 t 'Real', \
2 dt 4 t 'Imaginary'

\end{gnuplot}
\caption{Wave breaks for $b = 1.4$.}
\label{fig:break}
\end{subfigure}
\caption{Simulation with $s = 0.11$, $a = 30$, $h = 0.6$, and $E_0 = 0.1$ with $A_0 = \Gamma \sech{(2T)} \textrm{e}^{i \pi / 4}$. \textbf{Left:} Pulse after 8 iterations. \textbf{Right:} Pulse after 15 iterations.}
\label{fig:envelope}
\end{figure}

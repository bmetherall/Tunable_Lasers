\begin{figure*}[htbp]
\centering
\begin{subfigure}[]{\textwidth}
\begin{gnuplot}[terminal=epslatex, terminaloptions={color size 6.5in,2in lw 3}]
set grid
set xr [-2:2]
set xl '$T$'
set xtics 0.5
set ytics 1
set yl '$|A|^2$ (AU)'

p '../Stable/Envelope25.dat' u 1:(10**3*$4**2) w l t 'Stable', \
'../Break/Envelope25.dat' u 1:(10**3*$4**2) w l dt 5 t 'Broken'
\end{gnuplot}
\caption{Envelope}
\label{fig:envelope}
\end{subfigure} \\

\begin{subfigure}[]{\textwidth}
\begin{gnuplot}[terminal=epslatex, terminaloptions={color size 6.5in,2in lw 3}]
set grid
set xl '$\omega$'
set yl '$|\widehat{A}|$'
set xr [-40:40]
set ytics 3

p '../Stable/Envelope25.dat' u 6:($7**2) w l t 'Stable', \
'../Break/Envelope25.dat' u 6:($7**2) w l dt 5 t 'Broken'

\end{gnuplot}
\caption{Fourier Transform}
\label{fig:fourier}
\end{subfigure} \\

\begin{subfigure}[]{\textwidth}
\begin{gnuplot}[terminal=epslatex, terminaloptions={color size 6.5in,2in lw 3}]
#d2(x,y) = ($0 == 0) ? (x1 = x, y1 = y, 1/0) : (x2 = x1, x1 = x, y2 = y1, y1 = y, if (x1 > x2 + pi) (x2 = x2 + 2 * pi), (y1-y2)/(x1-x2))

x0=NaN
y0=NaN

d(x,y) = (dx=x-x0, x0=x, dy=y-y0, y0=y, (dy > pi) ? (dy = dy-2*pi) : (dy), ((dy < -pi)) ? (dy = dy+2*pi) : (dy), dy/dx)

set grid
set xr [-2:2]
set yr [-20:20]
set xl '$T$'
set xtics 0.5
set ytics 10
set yl '$-\diff{\varphi}{T}$'
set key bottom right opaque

p '../Stable/Envelope25.dat' u 1:(-d($1,$5)) w l t 'Stable', \
'../Break/Envelope25.dat' u 1:(-d($1,$5)) w l dt 5 t 'Broken'
\end{gnuplot}
\caption{Chirp}
\label{fig:chirp}
\end{subfigure}

\caption{Simulation with $s = 0.09$, $a = 8 \times 10^3$, $h = 0.04$, and $E_0 = 0.1$ with $A_0 = \Gamma \sech{(2T)} \textrm{e}^{i \pi / 4}$. $\Gamma$ is chosen such that $\int_{-\infty}^\infty |A|^2 \, \textrm{d}T = E_0$, and hyperbolic secant is chosen since it is a common soliton. In the stable case $b = 1.32$, whereas for the broken case $b = 1.25$. The pulse is shown after 25 iterations.}
\label{fig:pulse}
\end{figure*}

\documentclass[12pt,twoside,letterpaper]{article}

\usepackage{fancyhdr}
\usepackage{graphicx,epsfig,color,pstricks}
\usepackage{amsmath,esint}
\usepackage{latexsym,amssymb}
\usepackage{gensymb}
\usepackage{units}

%\usepackage[latin1]{inputenc}
%\usefonttheme{professionalfonts}
%\usepackage{times}
%\usepackage{tikz}
%\usepackage{amsmath, esint}


% sget page layout parameters (two sides)
\newlength{\defaultheadheight}
\setlength{\defaultheadheight}{\headheight}
\newlength{\defaultheadsep}
\setlength{\defaultheadsep}{\headsep}

% set page layout parameters (two sides)
\textwidth 7.0in
\textheight 9.75in
\hoffset -0.25in 
\voffset -0.75in
\topmargin 0in
\headheight \defaultheadheight
\headsep 0.16in
\footskip=0in
\oddsidemargin 0in 
\evensidemargin 0in %22pt default
%\evensidemargin 0.34in %22pt default
\raggedbottom

\DeclareMathOperator{\prob}{Prob}
\DeclareMathOperator{\pv}{PV}
\DeclareMathOperator{\sech}{sech}
\DeclareMathOperator{\sgn}{sgn}

%differentiation
\makeatletter
\providecommand*{\diff}%
    {\@ifnextchar^{\DIfF}{\DIfF^{}}}
\def\DIfF^#1{%
    \mathop{\mathrm{\mathstrut d}}%
        \nolimits^{#1}\gobblespace}
\def\gobblespace{%
    \futurelet\diffarg\opspace}
\def\opspace{%
    \let\DiffSpace\!%
    \ifx\diffarg(%
        \let\DiffSpace\relax
    \else
        \ifx\diffarg[%
            \let\DiffSpace\relax
    \else
        \ifx\diffarg\{%
            \let\DiffSpace\relax
        \fi\fi\fi\DiffSpace}

\providecommand*{\deriv}[3][]{%                   % \deriv[]{u}{x}
   \frac{\diff^{#1}#2}{\diff #3^{#1}}}
%\providecommand*{\deriv}[3][]{\frac{\diff^{#1}}{\diff #3^{#1}}#2}
\providecommand*{\pd}[3][]{\frac{\partial^{#1}#2}{\partial #3^{#1}}}
\providecommand{\bigO}[1]{\ensuremath{\mathop{}\mathopen{}\mathcal{O}\mathopen{}\left(#1\right)}}
\providecommand{\littleO}[1]{\ensuremath{\mathop{}\mathopen{}o\mathopen{}\left(#1\right)}}

% Special code to use the double bracket without loading the whole stmry package
\DeclareSymbolFont{stmry}{U}{stmry}{m}{n}
\SetSymbolFont{stmry}{bold}{U}{stmry}{b}{n}
\DeclareMathDelimiter\llbracket{\mathopen}{stmry}{"4A}{stmry}{"71}
\DeclareMathDelimiter\rrbracket{\mathclose}{stmry}{"4B}{stmry}{"79}


\begin{document}

\newcounter{mychapter}
\setcounter{mychapter}{1}
\renewcommand{\labelenumi}{\textbf{\arabic{mychapter}.\arabic{enumi}}}

\pagenumbering{arabic}
%\setcounter{page}{1}
\pagestyle{fancy}
\fancyhf{} % clear all header and footer fields
\renewcommand{\footrulewidth}{0pt}
\renewcommand{\headrulewidth}{0.1pt}

\fancyhead[L]{\textbf{Tunable Lasers}: Dr. C. Sean Bohun}
\fancyhead[R]{Brady -- 2018 \ Page \thepage}


The model for the amplitude in the `average' model is given by
\begin{equation}
\label{average}
	\pd{A}{z} = -i\frac{\beta_2}{2}\pd[2]{A}{T} - \frac{\epsilon}{2}T^2 A + \frac{g}{2}A
\end{equation}
with $\beta_2 \in \mathbb{R}$ defining the dispersion, $\epsilon \in \mathbb{R}$, $\epsilon > 0$, determining the
modulation and $g \in \mathbb{R}$, $g > 0$ giving the gain.
The ansatz for the amplitude is a function of the form
\begin{equation}
\label{ansatz}
	A(T,z) = \left(\frac{P_0}{1-i C}\right)^{1/2}\exp
	\left(-\frac{\delta \Omega^2 T^2}{2(1-i C)}\right)\textrm{e}^{i\psi z}
\end{equation}
so that the modulus $|A|$ and the phase $A/|A|$ are given by
\begin{align*}
	\left|A\right| &= \left(\frac{P_0}{\sqrt{1 + C^2}}\right)^{1/2}
	\exp\left(-\frac{\delta \Omega^2 T^2}{2(1 + C^2)}\right),&
	\frac{A}{|A|} &= \left(\frac{1+iC}{\sqrt{1+C^2}}\right)^{1/2}\exp
	\left(-i\frac{\delta \Omega^2 T^2 C}{2(1+C^2)}\right)\textrm{e}^{i\psi z}.
\end{align*}
The quantity $C \in \mathbb{R}$ is known as the chirp and contributes a constant phase of $\theta$ where
$2\theta = \arctan C$, $P_0$ is the maximum value of $|A|^2$ at $C=0$ (zero chirp), $\psi \in \mathbb{R}$
is the accumulated phase, and $\delta\Omega^2$ is the spectral half-width of $|A|^2$ 
since\footnote{If $f(t) = \textrm{e}^{-\alpha t^2}$ then $\hat{f}(\omega) = \frac{1}{2\pi}\int_{\mathbb{R}}
f(t)\textrm{e}^{i\omega t}\,\diff t = \frac{1}{2\pi} (\frac{\pi}{\alpha})^{1/2}\textrm{e}^{-\omega^2/4\alpha}$.}
\[
	\hat{A}(\omega,z) = \frac{1}{2\pi}\int_{\infty}^\infty A(t,z) \textrm{e}^{i\omega t}\, \diff t
	= \left(\frac{P_0}{2\pi\delta \Omega^2}\right)^{1/2}
	\exp\left(-\frac{(1-i C)\omega^2}{2\delta \Omega^2}\right) \textrm{e}^{i\psi z},
\]
and $|\hat{A}|^2(\delta\Omega,z) = \textrm{e}^{-1}|\hat{A}|^2(0,z)$.  The corresponding half-width of the
pulse duration comes from the expression for $|A|$ and gives
\begin{equation}
\label{duration}
	\delta T = \frac{\sqrt{1+C^2}}{\delta \Omega}.
\end{equation}

In the case of the expression~(\ref{average}), applying~(\ref{ansatz}) gives the condition
\begin{align*}
	i\psi &= -i\frac{\beta_2}{2}\delta\Omega^2(1-iC)^{-1}
	\left(-1 + \delta\Omega^2(1-iC)^{-1}T^2\right)-\frac{\epsilon}{2}T^2 + \frac{g}{2} \\
	&= \frac{\beta_2\delta\Omega^2}{2(1+C^2)^2}(-C(1+C^2)+2\delta\Omega^2 T^2C) -\frac{\epsilon}{2}T^2 + \frac{g}{2}\\
	& \quad + i\frac{\beta_2\delta\Omega^2}{2(1+C^2)^2}(1+C^2 + \delta\Omega^2 T^2(C^2-1)).
\end{align*}
This gives four conditions by utilizing that $\psi\in\mathbb{R}$.  In detail,
\begin{align*}
	&\bigO{T^2}_{\textrm{Im}}: 0 = C^2 - 1, &
	&\bigO{1}_{\textrm{Im}}: \psi = \frac{\beta_2\delta\Omega^2}{2(1+C^2)}, \\
	&\bigO{T^2}_{\textrm{Re}}: \epsilon = \frac{2\beta_2\delta\Omega^4C}{(1+C^2)^2}, &
	&\bigO{1}_{\textrm{Re}}: g = \frac{\beta_2\delta\Omega^2C}{(1+C^2)}.
\end{align*}
Starting with $\bigO{1}_{\textrm{Re}}$ we note that $g > 0$ implies that 
$\sgn(\beta_2 C) = \sgn(\beta_2)\sgn(C) = 1$. 
From $\bigO{T^2}_{\textrm{Im}}$, $C = \pm 1$ and therefore $C = \sgn(\beta_2)$, 
$\beta_2 C = |\beta_2|$ and $\epsilon > 0$.  We also see that the representation~(\ref{ansatz})
as a classical solution of~(\ref{average}) imposes a subclass of solutions whereby 
$g = \left(\epsilon|\beta_2|/2\right)^{1/2}$.  One is left with a two parameter family of solutions
to~(\ref{ansatz}) with
\begin{align*}
	\delta\Omega^2 &= \left(\frac{2\epsilon}{|\beta_2|}\right)^{1/2},&
	\psi &= \sgn(\beta_2)\left(\frac{\epsilon|\beta_2|}{8}\right)^{1/2},&
	\delta T^2 &= \left(\frac{2|\beta_2|}{\epsilon}\right)^{1/2}.&
\end{align*}
Since~(\ref{average}) is linear in $A$, any value of the peak power, $P_0$, is admissible.  In practice however, 
at high power levels the gain drops as the fibre saturates.  One can model this with
\begin{equation}
\label{newg}
	g(P_0) = \frac{g_0}{1+\frac{\textrm{power in fibre}}{\textrm{saturation power}}} - \alpha
\end{equation}
where $g_0$ is the low-power gain and $\alpha$ represents the net losses in the laser cavity.
The power in the fibre depends on the frequency $f$ and modulus of the pulse so that
\begin{align*}
	f \int_{-\infty}^\infty |A(s)|^2\,\diff s &=
	\frac{\sqrt{\pi}f}{\delta\Omega}P_0 = \Delta P_0, &
	\Delta &= \frac{\sqrt{\pi}f}{\delta\Omega} = \frac{\sqrt{\pi}f\delta T}{\sqrt{1+C^2}}
\end{align*}
where $\Delta$ is the duty cycle of the pulse.  Denoting the saturation power as 
$P_{\textrm{sat}}$,~(\ref{newg}) can be inverted to give
\[
	P_0 = \frac{P_{\textrm{sat}}}{\sqrt{\pi}f}  \left(\frac{|\beta_2|}{2\epsilon}\right)^{1/4}
	\left(g_0\left(\left(\frac{\epsilon|\beta_2|}{2}\right)^{1/2} + \alpha\right)^{-1} - 1\right).
\]



\end{document}